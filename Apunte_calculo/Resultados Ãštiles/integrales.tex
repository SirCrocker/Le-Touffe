\subsection{Primitivas}
\textit{Se ignora la constante de integración para facilitar la visualización.}\newline
\begin{minipage}{0.55\textwidth}
\begin{equation}
\begin{split}
    &\int\frac{1}{x}dx  = \ln{|x|} \\
    &\int e^{f(x)}f'(x)dx  = e^{f(x)} \\
    &\int \ln{x}dx  = x\ln{x}-x \\
    &\int x\ln{x}dx  = \frac{1}{4}x^2(2\ln{x}-1) \\
    &\int\frac{1}{x^2+a^2}\,dx  = \frac{1}{a}\arctan{\frac{1}{a}}\\
    &\int\frac{1}{\sqrt{x^2-a^2}}\,dx  = \sinh^{-1}{\frac{x}{a}}\\
    &\int\frac{1}{\sqrt{a^2-x^2}}\,dx  = \arcsin{\frac{x}{a}}\\
    &\int \frac{1}{\sqrt{x^2+a^2}}\,dx  = \cosh^{-1}{\frac{x}{a}}\\
    &\int\frac{1}{x\sqrt{x^2-a^2}}\,dx  = \frac{1}{a}\arctan{\left(\frac{\sqrt{x^2-a^2}}{a}\right)}\\
    &\int\frac{1}{x\sqrt{x^2+a^2}}\,dx  = -\frac{1}{a}\ln{\left(\frac{
    a+\sqrt{x^2+a^2}}{x}\right)}\\
    &\int\frac{1}{x\sqrt{a^2-x^2}}\,dx  = -\frac{1}{a}\ln{\left(\frac{a+\sqrt{a^2-x^2}}{x}\right)}\\
    &\int\frac{1}{x^2-a^2}\,dx  = \frac{\ln{|x-a|}-\ln{|x+a|}}{2a}\\
    &\int\frac{1}{a^2-x^2}\,dx  = \frac{\ln{|x+a|}-\ln{|x-a|}}{2a}\\
    &\int\frac{1}{(1+x^2)^2}\,dx  = \frac{x}{2(1+x^2)}+\frac{1}{2}\arctan{x}\\
    &\int\cos{(\ln{x})}\,dx= \frac{x}{2}(\cos{(\ln{x})}+\sin{(\ln{x})})\\
\end{split}
\nonumber
\end{equation}
\end{minipage}
\begin{minipage}{0.55\textwidth}
\begin{equation}
\begin{split}
    &\int \sin{x}\,dx = -\cos{x} \\
    &\int \cos{x}\,dx = \sin{x} \\
    &\int\tan{x}\,dx  = -\ln{|\cos{x}|} \\
    &\int\sec{x}\,dx  = \ln{|\sec{x}+\tan{x}|}\\
    &\int\csc{x}\,dx  = \ln{|\csc{x}-\cot{x}|}\\
    &\int\tan^2{x}\,dx  = \tan{x}-x \\
    &\int\cot^2{x}\,dx  = -\cot{x}-x \\
    &\int\cos^2{(ax)}\,dx  = \frac{x}{2}+\frac{\sin{(2ax)}}{4a}\\
    &\int\sin^2{(ax)}\,dx  = \frac{x}{2}-\frac{\sin{(2ax)}}{4a}\\
    &\int\cosh^2{(ax)}\,dx  = \frac{x}{2}+\frac{\sinh{(2ax)}}{4a}\\
    &\int\sinh^2{(ax)}\,dx  = \frac{\sinh{(2ax)}}{4a}-\frac{x}{2}\\
    &\int\frac{1}{\cosh{x}}\,dx  = 2\arctan{e^x}\\
    &\int\frac{1}{\sinh{x}}\,dx  = \ln{|e^x-1|}-\ln{|e^x+1|}\\
    &\int\sqrt{a^2-x^2}\,dx  = \frac{x}{2}\sqrt{a^2-x^2}+\frac{a^2}{2}\arcsin{\frac{x}{a}}\\
    &\int\sin{(bx)}e^{ax}\,dx=\frac{a\sin{(bx)}-b\cos{(bx)}}{a^2+b^2}e^{ax}\\
    &\int\frac{x}{\sqrt{a-bx}}\,dx= \frac{-2a}{b^2}\sqrt{a-bx}+\frac{2}{3b^2}(a-bx)^{3/2}\\
\end{split}
\nonumber
\end{equation}
\end{minipage}
\newpage
\begin{minipage}{0.55\textwidth}
\begin{equation}
\begin{split}
    &\int|x|\,dx  = \frac{x\abs{x}}{2}\\
    &\int \sin^3{x}\,dx = \frac{1}{4}\left(\frac{\cos{3x}}{3}-3\cos{x}\right)\\
    &\int \cos^3{x}\,dx = \frac{1}{4}\left(\frac{\sin{3x}}{3}+3\sin{x}\right)\\
\end{split}
\nonumber
\end{equation}
\end{minipage}
\begin{minipage}{0.55\textwidth}
\begin{equation}
\begin{split}
    &\int\sqrt{a^2+x^2}\,dx = \frac{x}{2}\sqrt{x^2+a^2}+\frac{a^2}{2}\ln{(x~\sqrt{x^2+a^2})}\\
    &\int\sqrt{x^2-a^2}\,dx = \frac{x}{2}\sqrt{x^2-a^2}-\frac{a^2}{2}\ln{(x+\sqrt{x^2-a^2})}\\
    &\int\frac{x}{(x^2+a^2)^{3/2}}\,dx = -\frac{1}{\sqrt{x^2 + a^2}}\\
\end{split}
\nonumber
\end{equation}
\end{minipage}

\[\int\frac{x}{\sqrt{ax^2 + bx +c}}\,dx = \frac{1}{a}\sqrt{ax^2 +bx +c} - \frac{b}{2a^{3/2}}\ln{\abs{2ax +b+2\sqrt{a(ax^2+bx+c)}}}\]
\bigbreak
Para funciones de forma $\frac{Ax+B}{ax^2+bx+c}$, se tiene que:

\begin{itemize}
    \item Si $4ac>b^2$
    \[\int\frac{Ax+B}{ax^2+bx+c}\,dx = \frac{A}{2a}\ln{(ax^2+bx+c)}+\frac{2aB-bA}{a\sqrt{4ac-b^2}}\arctan{\left(\frac{2ax+b}{\sqrt{4ac-b^2}}\right)}\]
    \item Si $4ac<b^2$ %este puede que haya que revisarlo
    \[\int\frac{Ax+B}{ax^2+bx+c}\,dx = \frac{1}{2\sqrt{b^2-4ac}}
    \left(\Omega_{-}+\,\Omega_{+}\right)\]
    donde
    \[\Omega_{-} = (2B-A(b-\sqrt{b^2-4ac}))\ln{(2ax+b-\sqrt{b^2-4ac})}\]
    \[\Omega_{+} = (A(b+\sqrt{b^2-4ac})-2B)\ln{(2ax+b+\sqrt{b^2-4ac})}\]
\end{itemize}

\subsection{De Suma a Integral}

Dado un intervalo $[a, b]$ y una partición $\{x_0=a, x_1, x_2, ..., x_n=b\}$ tal que $\Delta x_i = x_{i+1}-x_i$ es una diferencia infinitesimal, considerando $y_i\in [x_i, x_{i+1}]$, por la definición de integral de Riemann, se verifica que

\[\sum f(y_i)\Delta x_i = \int^b_af(x)\,dx\]

Algunos ejemplos son:

\begin{equation}
\begin{split}
    &\lim_{n\to\infty}\frac{1}{n}\sum^n_{k=1}f(k/n)=\int^1_0f(x)\,dx\\
    &\lim_{n\to\infty}\sum^n_{k=1}f(r^{\frac{k}{n}})
    (r^{\frac{k}{n}}-r^{\frac{k-1}{n}})=\int^r_1f(x)\,dx\\
\end{split}
\nonumber
\end{equation}

\subsection{Métodos de Integración}

\subsubsection{Integración por Partes}

Dado que la derivada del producto de dos funciones $f(x)h(x)$ es $f'(x)h(x)+f(x)h'(x)$, la integral de una función de forma $f'(x)h(x)$ se puede calcular como

\[\int f'(x)h(x)\,dx = f(x)h(x) - \int f(x)h'(x)\,dx\]

\subsubsection{Cambio de Variable}

Sean $h \in \mathcal{C}^1$, $f\in\mathcal{C}^0$, $F$ una primitiva de $f$ y $H = F \circ h$, se verifica que

\[\int^b_af(h)h'=H(b)-H(a)=F(h(b))-F(h(a))=\int^{h(b)}_{h(a)}f\]

esto es un cambio de variable. Hay dos formas de notarlo:

\begin{equation}
\begin{split}
    &\int^{h(b)}_{h(a)}f(x)\,dx =
    \begin{bmatrix}
    x = h(t)\\
    dx= h'(t)dt
    \end{bmatrix} = \int^b_af(h(t))h'(t)\,dt\\
    &\int^b_af(h(t))h'(t)\,dt = \int^{h(b)}_{h(a)} f(h)\,dh\\
\end{split}
\nonumber
\end{equation}

La segunda forma es especialmente útil para ecuaciones diferenciales.

\subsubsection{F(sin x, cos x)}

Cuando se tiene integrales de forma

\[\int F(\sin{x}, \cos{x})\,dx\]

puede convenir utilizar el cambio de variable $x = 2\arctan{t}$, para el cual se cumple que

\begin{minipage}{0.55\textwidth}
\begin{equation}
    \sin{x}=\frac{2t}{1+t^2}\,\,\,\,\,\,\,\,
    \cos{x}=\frac{1-t^2}{1+t^2}\,\,\,\,\,\,\,\,
    t=\tan{\frac{x}{2}}\,\,\,\,\,\,\,\,
    dx=\frac{2}{1+t^2}dt
\nonumber
\end{equation}
\end{minipage}

\[\int F(\sin{x}, \cos{x})\,dx = 
\begin{bmatrix}
x = 2\arctan{t}\\
dx=\frac{2}{1+t^2}dt
\end{bmatrix}
= \int F\left(\frac{2t}{1+t^2},\frac{1-t^2}{1+t^2}\right)
\frac{2}{1+t^2}\,dt\]

Casos particulares:

\begin{itemize}
    \item si $F(-\sin{x}, \cos{x}) = -F(\sin{x}, \cos{x})$, el cambio $t = \cos{x}$ puede funcionar.
    \item si $F(\sin{x}, -\cos{x}) = -F(\sin{x}, \cos{x})$, el cambio $t = \sin{x}$ puede funcionar.
    \item si $F(-\sin{x}, -\cos{x}) = F(\sin{x}, \cos{x})$, el cambio $t = \tan{x}$ puede funcionar.
\end{itemize}

\subsubsection{Método 2}% nombre provisional

Sea $L(x) = \frac{ax+b}{cx+d}$ tal que $ad-bc\neq 0$, para integrales del tipo

\[\int F(x, L(x)^{r_1}, L(x)^{r_2}, ..., L(x)^{r_n})\,dx\]

donde $r_1, r_2, ..., r_n \in \mathbb{Q}$, tomando $q$ como el mínimo común denominador de $\{r_i\}_{i=1}^n$, se puede hacer el cambio de variable

\begin{equation}
\begin{split}
\int F(x, L(x)^{r_1}, L(x)^{r_2}, ..., L(x)^{r_n})\,dx & =
\begin{bmatrix}
x = L^{-1}(t^q) = P(t)\\
dx=P'(t)dt
\end{bmatrix}\\ & =
\int F(P(t), t^{qr_1}, t^{qr_2}, ..., t^{qr_n})P'(t)\,dt
\end{split}
\nonumber
\end{equation}

\subsubsection{Integrales Binomias}

Se llama integral binomia a aquellas de forma

\[\int x^\alpha(a+bx^\beta)^\gamma\,dx\]

donde $\alpha, \beta, \gamma \in \mathbb{Q}\setminus\{0\}$ y $a, b\in\mathbb{R}\setminus\{0\}$. Tomando $r = \frac{\alpha+1}{\beta}-1$, se puede hacer el cambio de variable

\[\int x^\alpha(a+bx^\beta)^\gamma\,dx =
\begin{bmatrix}
x = t^\frac{1}{\beta}\\
dx=\frac{1}{\beta}t^{\frac{1}{\beta}-1}dt
\end{bmatrix}
= \frac{1}{\beta}\int t^r(a+bt)^\gamma\,dt\]

Esta integral es del tipo que considera el \textbf{Método 2} si %hay que cambiar lo de "método 2"

\begin{itemize}
    \item $\gamma\in\mathbb{Z}$, siendo de forma $\int F(t, t^r)dt$
    \item $r\in\mathbb{Z}$, siendo de forma $\int F(t, (a+bt)^\gamma)dt$
    \item $r+\gamma\in\mathbb{Z}$, siendo de forma $\int \left(\frac{a+bt}{t}\right)^\gamma t^{r+\gamma}dt$
\end{itemize}

\subsubsection{Raíces de Parábolas}

Para integrales de forma

\[\int F(x, \sqrt{ax^2+bx+c})\, dx\]

Si $\alpha, \beta\in\mathbb{R}$ son raíces de $ax^2+bx+c$, se verifica que

\[\sqrt{ax^2+bx+c}=\sqrt{a(x-\alpha)(x-\beta)} =
(x-\alpha)\sqrt{\frac{a(x-\beta)}{x-\alpha}}\]

por lo que conviene hacer el cambio

\begin{equation}
\int F(x,\sqrt{ax^2+bx+c})\,dx =
\begin{bmatrix}
x = \frac{\alpha t^2-a\beta}{t^2-a} = r(t)\\
dx = r'(t)dt
\end{bmatrix} =
\int F(r(t),(r(t)-\alpha)t)r'(t)\,dt
\nonumber
\end{equation}
\bigbreak
\bigbreak
Si $ax^2+bx+c$ no tiene raíces reales y $c>0$, se puede hacer el cambio

\begin{equation}
\int F(x,\sqrt{ax^2+bx+c})\,dx =
\begin{bmatrix}
x = \frac{b-2t\sqrt{c}}{t^2-a} = r(t)\\
dx = r'(t)dt
\end{bmatrix} =
\int F(r(t),tr(t)+\sqrt{c})r'(t)\,dt
\nonumber
\end{equation}
\bigbreak
donde $tx+\sqrt{c} = \sqrt{ax^2+bx+c}$

\newpage