\section{Cálculo en Varias Variables}

\subsection{Función en Varias Variables}

Sean $\mathbf{E}, \mathbf{F}$ espacios vectoriales arbitrarios, se definen las funciones en varias variables como aquellas de forma

\[f: A\subseteq\mathbf{E}\to B\subseteq\mathbf{F}\]

En adelante usaremos $\mathbf{E}$ y $\mathbf{F}$ para referirnos a espacios vectoriales.

\subsection{Normas}

Dado $\mathbf{E}$ un espacio vectorial sobre $\mathbb{R}$, se dice que $N:\mathbf{E}\to\mathbb{R}$ es una norma de $\mathbf{E}$ si cumple que

\begin{itemize}
    \item $\forall x\in\mathbf{E}:\, N(x)\geq 0$
    \item $\forall x\in\mathbf{E}:\, N(x)=0\Leftrightarrow
    x=0$
    \item $\forall\lambda\in\mathbb{R}\forall x\in\mathbf{E}:\,
    N(\lambda x) = |\lambda | N(x)$
    \item $\forall x,y\in\mathbf{E}:\, N(x+y)\leq N(x)+N(y)$
\end{itemize}

En tal caso se llama a $\mathbf{E}$ un espacio normado. Se suele usar $\|\cdot\|_\infty$ para notar las normas definidas por supremos

\subsubsection{Normas en $\mathbb{R}^n$}

\begin{itemize}
    \item $\|x\|_p=\left(\sum^n_{i=1}|x_i|^p\right)^{\frac{1}{p}}$
    \item $\|x\|_\infty = \max\{|x_i|\}^n_{i=1}$
\end{itemize}

\subsubsection{Normas Equivalentes}

Dos normas $\|\cdot\|_a$ y $\|\cdot\|_b$ de un mismo espacio vectorial $\mathbf{E}$ son equivalentes si se cumple que

\[\exists L_a, L_b\in\mathbb{R}^+\,\forall x\in\mathbf{E}
:\, \|x\|_a \leq L_a\|x\|_b \wedge \|x\|_b \leq L_b\|x\|_a\]

Todas las normas en $\mathbb{R}^n$ son equivalentes.

\subsection{Bolas}

Sean $\mathbf{E}$ un espacio normado, $a\in\mathbf{E}$ y $r\in\mathbb{R}^+$, se definen

\begin{itemize}
    \item \textbf{Bola Abierta}: $B(a,r)=\{x\in\mathbf{E}\,
    |\,\|x-a\|<r\}$
    \item \textbf{Bola Cerrada}: $\bar{B}(a,r)=\{x\in\mathbf{E}\,|\,\|x-a\|<r\}$
\end{itemize}

Para $B_a$ y $B_b$ bolas definidas en un mismo espacio con normas equivalentes, se verifica que

\[\forall\varepsilon\in\mathbb{R}^+
\exists\delta\in\mathbb{R}^+:\,B_a(c,\delta)\subseteq B_b(c,\varepsilon)\]
\[\forall\varepsilon\in\mathbb{R}^+
\exists\delta\in\mathbb{R}^+:\,B_b(c,\delta)\subseteq B_a(c,\varepsilon)\]

\subsection{Conjuntos Acotados}

Un conjunto $A\subseteq\mathbf{E}$ es acotado si verifica que

\[\exists\delta\in\mathbb{R}^+:\,A\subseteq B(0,\delta)\]

\subsection{Conjuntos Abiertos}

Un conjunto $A\subseteq\mathbf{E}$ es abierto si verifica que

\[\forall x\in A\exists\delta\in\mathbb{R}^+:
\,B(x,\delta)\subseteq A\]

\begin{itemize}
    \item La unión finita e infinita de abiertos es un abierto
    \item La intersección finita de abiertos es un abierto
    \item Si un conjunto es abierto para una norma, también lo será para sus equivalentes
\end{itemize}

\subsection{Conjuntos Cerrados}

Un conjunto $A\subseteq\mathbf{E}$ es cerrado si $A^c$ es abierto.

\begin{itemize}
    \item La unión finita de cerrados es un cerrado
    \item La intersección finita e infinita de cerrados es un cerrado
\end{itemize}

\subsection{Interior}

Sea $A\subseteq\mathbf{E}$, se define el interior de $A$ como

\[\mathrm{Int}(A) = \{x\in\mathbf{E}\,|\,
\exists\delta\in\mathbb{R}^+:\,B(x,\delta)\subseteq A\}\]

\begin{itemize}
    \item Si $A$ es abierto si y sólo si $\mathrm{Int}(A)=A$
    \item $\mathrm{Int}(\mathrm{Int}(A)) = \mathrm{Int}(A)$
    \item Si $A\subseteq B$, entonces $\mathrm{Int}(A)
    \subseteq \mathrm{Int}(B)$
    \item La preimagen de un abierto por una función continua también es un abierto
\end{itemize}

\subsection{Adherencia}

Sea $A\subseteq\mathbf{E}$, se define la adherencia de $A$ como

\[\mathrm{Adh}(A)= \bar{A} = \{x\in\mathbf{E}\,|\,
\forall\delta\in\mathbb{R}^+:\,B(x,\delta)\cap A\neq \emptyset\}\]

\begin{itemize}
    \item $\mathrm{Int}(A)\subseteq A\subseteq \bar{A}$
    \item Si $A\subseteq B$, entonces $\bar{A}\subseteq\bar{B}$
    \item $A$ es cerrado si y sólo si $A=\bar{A}$
\end{itemize}

\subsection{Exterior}

Sea $A\subseteq\mathbf{E}$, se define el exterior de $A$ como

\[\mathrm{Ext}(A) = \{x\in\mathbf{E}\,|\,
\exists\delta\in\mathbb{R}^+:\,B(x,\delta)\subseteq A^c\}=
\mathrm{Int}(A^c)\]

\subsection{Frontera}

Sea $A\subseteq\mathbf{E}$, se define la frontera de $A$ como

\[\mathrm{Fr}(A)=\mathrm{Ext}(A)^c\cap\mathrm{Int}(A)^c
=\mathrm{Adh}(A)\setminus\mathrm{Int}(A)\]

\subsection{Sucesiones}

Las sucesiones en $\mathbf{E}$ son funciones del tipo $s:\mathbb{N}\to\mathbf{E}$. También se pueden definir como $\{s_n\}_{n\in\mathbb{N}}\subseteq\mathbf{E}$.

\subsubsection{Convergencia}

Una sucesión $(s_n)$ en $\mathbf{E}$ converge a $l\in\mathbf{E}$ si se verifica que

\[\forall\varepsilon\in\mathbb{R}^+\,
\exists n_o\in\mathbb{N}\, \forall n\geq n_o:\,
s_n\in B(l, \varepsilon)\]

\begin{itemize}
\item Dadas dos normas en $\mathbf{E}$, $\|\cdot\|_1$ y $\|\cdot\|_2$, tales que
\[\exists L\in\mathbb{R}^+\,\forall x\in\mathbf{E}:\,
\|x\|_1\leq L\|x\|_2\]
se cumple que, si $(s_n)$ converge a $l$ para $\|\cdot\|_2$, entonces también se lo hace para $\|\cdot\|_1$.
\end{itemize}

\subsubsection{Sucesiones Acotadas}

Una sucesión $(s_n)$ es acotada si el conjunto $\{s_n\}_{n\in\mathbb{N}}$ es acotado. Toda sucesión convergente es acotada.

\subsubsection{Caracterización de los Cerrados}

$A\subseteq\mathbf{E}$ es cerrado si y sólo si toda sucesión convergente en $A$ tiene su límite en $A$.

\subsubsection{Sucesiones en $\mathbb{R}^n$}

Una sucesión $(\Vec{S_k})$ en $\mathbb{R}^n$ se compone de $n$ sucesiones en $\mathbb{R}$

\[\Vec{S_k}=\begin{pmatrix} s_k^1\\ \cdot\\ \cdot\\
\cdot\\ s_k^n\end{pmatrix}\]

$(\Vec{S_k})$ converge si sus componentes convergen.

\subsection{Puntos de Acumulación}

Dados $(A_n)$ una sucesión en $\mathbf{E}$ y $a\in\mathbf{E}$, se dice que $a$ es un punto de acumulación de $(A_n)$ si verifica que

\[\forall\varepsilon\in\mathbb{R}^+\,
\forall n\in\mathbb{N}\, \exists m\geq n:\,
\|A_m-a\|\leq \varepsilon\]

\begin{itemize}
    \item Si $(A_n)$ converge a $l$, entonces $l$ es el único punto de acumulación de $(A_n)$
    \item $a$ es punto de acumulación de $(A_n)$ si y sólo si existe una subsucesión de $(A_n)$ que converge a $a$
\end{itemize}

\subsubsection{Puntos de Acumulación de un Conjunto}

Sea $A\subseteq\mathbf{E}$, $a$ es un punto de acumulación de $A$ si cumple que

\[\forall\delta\in\mathbb{R}^+:\,A\cap(B(a,\delta)
\setminus\{a\})\neq \emptyset\]

Se nota por $A'$ al conjunto de los putos de acumulación de $A$.

\subsection{Puntos Aislados}

Sea $A\subseteq\mathbf{E}$, $x$ es un punto aislado de $A$ si cumple que

\[\exists\delta\in\mathbb{R}^+:\,A\cap(B(a,\delta)
\setminus\{a\})=\emptyset\]

\subsection{Sucesiones de Cauchy}

Una sucesión $(s_n)$ en $\mathbf{E}$ es de Cauchy si se cumple que

\[\forall\varepsilon\in\mathbb{R}^+\,
\exists n_o\in\mathbb{N}\, \forall m, n\geq n_o:\,
\|s_m-s_n\|\leq \varepsilon\]

\begin{itemize}
    \item Si una sucesión es de Cauchy para una norma, lo es también para sus equivalentes
    \item Toda sucesión convergente es de Cauchy
    \item Si una sucesión de Cauchy tiene un punto de acumulación, entonces esta converge a dicho punto
    \item Toda sucesión de Cauchy es acotada
\end{itemize}

\subsection{Espacios de Banach}

Un espacio vectorial es de Banach si toda sucesión de Cauchy contenida en él es convergente.

\begin{itemize}
    \item $\mathbb{R}^n$ es un espacio de Banach
    \item Un subespacio de un espacio de Banach también es espacio de Banach
\end{itemize}

\subsection{Conjuntos Compactos}

Un conjunto $A\subseteq\mathbf{E}$ es compacto si toda sucesión en $A$ tiene una subsucesión convergente.

\begin{itemize}
    \item Si $A$ es compacto, entones $A$ es cerrado y acotado
    \item Si $A\subseteq\mathbb{R}^n$ es cerrado y acotado, entonces $A$ es compacto
\end{itemize}

\subsection{Continuidad}

Si $\mathbf{E}$ y $\mathbf{F}$ son espacios normados, se dice que una función $f:A\subseteq\mathbf{E}\to\mathbf{F}$ es continua en $\bar{x}=A$ si verifica que

\[\forall\varepsilon\in\mathbb{R}^+\,\exists\delta\in
\mathbb{R}^+\,\forall x\in A:\, \|x-\bar{x}\|\leq \delta \Rightarrow \|f(x)-f(\bar{x})\|\leq \varepsilon\]

\begin{itemize}
    \item Si $f$ es continua para una norma, también los para sus equivalentes
    \item La suma y composición de funciones continuas, así como el producto de una función continua por un escalar, son continuas.
    \item Si $f$ es continua en $\bar{x}$, se tiene que
    \[\forall x\in A\cap B(\bar{x},\delta):\,B(f(\bar{x}),\varepsilon)\]
    lo que equivale a
    \[f(A\cap B(\bar{x}, \delta))\subseteq B(f(\bar{x}),\varepsilon)\]
    \item Sea $K\subseteq A$ un compacto, si $f$ es continua, entonces $f(K)$ es compacto
    \item Si $f:A\to\mathbb{R}$ es continua y $K\subseteq A$ un compacto, se tiene que
    \[\exists a,b\in K\,\forall x\in K:\, f(a)\leq f(x)
    \leq f(b)\]
\end{itemize}

\subsection{Límites}

Sean $f:A\subseteq\mathbf{E}\to\mathbf{F}$, $a \in A'$ y $L\in\mathbf{F}$, $L$ es el límite de de $f$ en $a$ si se verifica que

\[\forall\varepsilon\in\mathbb{R}^+\,\exists\delta\in
\mathbb{R}^+\,\forall x\in A:\, \|x-a\|\leq\delta\Rightarrow
\|f(x)-L\|\leq \varepsilon\]
\bigbreak
Se nota entonces $\lim_{x\to a}f(x)=L$

\begin{itemize}
    \item El álgebra de límites en varias variables es la misma que para funciones reales
    \item $f$ es continua en $a$ si y sólo si para toda sucesión $(x_n)$ en $A$ se cumple que
    \[\lim_{n\to\infty}x_n=a\Rightarrow\lim_{n\to\infty}
    f(x_n)=f(a)\]
\end{itemize}

\subsection{Continuidad Uniforme}

se dice que una función $f:A\subseteq\mathbf{E}\to\mathbf{F}$ es uniformemente continua en $A$ si

\[\forall\varepsilon\in\mathbb{R}^+\,\exists\delta\in
\mathbb{R}^+\,\forall x\in A\,\forall \bar{x}\in A:\,\|x-a\|\leq\delta\Rightarrow
\|f(x)-f(\bar{x})\|\leq \varepsilon\]
\bigbreak
\begin{itemize}
    \item Continuidad uniforme implica continuidad
    \item Si $f$ es continua en un compacto, entonces $f$ es uniformemente continua
\end{itemize}

\subsection{Funciones Lineales Continuas}

Se usa $\mathcal{L}(\mathbf{E},\mathbf{F})$ para denotar el conjunto de todas las funciones lineales y continuas de $\mathbf{E}$ a $\mathbf{F}$.

\begin{itemize}
    \item Si $\mathbf{E}$ es un espacio normado y $\mathbf{F}$ un espacio de Banach, entonces $\mathcal{L}(\mathbf{E},\mathbf{F})$ es un espacio de Banach
    \item Toda función lineal definida sobre un espacio de dimensión finita es continua
    \item Para toda $f\in\mathcal{L}(\mathbf{E},\mathbf{F})$ se verifica que
    \[\exists L\in\mathbb{R}^+\,\forall x\in\exists:\,
    \|f(x)\|\leq L\|x\|\]
\end{itemize}

\subsubsection{Norma}

Se define la norma en $\mathcal{L}(\mathbf{E},\mathbf{F})$ como

\begin{equation}
\begin{split}
\|f\| &= \inf\{L\in\mathbb{R}^+\,|\,\forall x\in
\mathbf{E}:\,\|f(x)\|\leq L\|x\|\}\\
&= \sup_{x\in\mathbf{E}}\left\{\frac{\|f(x)\|}{\|x\|}\right\}
\end{split}
\nonumber
\end{equation}

\subsection{Funciones de Lipchitz}

Una función $f:A\subseteq\mathbf{E}\to\mathbf{F}$ es de Lipchitz si existe $L\in\mathbb{R}^+$ tal que

\[\forall x,y\in A:\, \|f(x)-f(y)\|\leq L\|x-y\|\]

\subsubsection{Funciones Contractantes}

Se dice que $f:A\subseteq\mathbf{E}\to\mathbf{F}$ es contractante si es de Lipchitz con $L<1$.

\subsection{Teorema del Punto Fijo}
\label{T:.PuntoFijo}

Sea $f:A\subseteq\mathbf{E}\to A$ contractante, con $\mathbf{E}$ un espacio de Banach y A cerrado, existe un único $\bar{x}\in A$ tal que $f(\bar{x})=\bar{x}$. A $\bar{x}$ se le llama punto fijo.

\subsection{Espacio de Funciones Acotadas}

Se usa $\mathcal{A}(\mathbf{E},\mathbf{F})$ para denotar el conjunto de todas las funciones acotadas de $\mathbf{E}$ a $\mathbf{F}$.

\begin{itemize}
    \item Si $\mathbf{F}$ es un espacio de Banach, entonces $\mathcal{A}$ es de Banach
    \item Dada $(f_n)$ una sucesión en $\mathcal{A}(A\subseteq\mathbf{E},\mathbf{F})$, si converge a $f\in\mathcal{A}(A\subseteq\mathbf{E},\mathbf{F})$ y para toda $n\in\mathbb{N}$ $f_n$ es continua, entonces $f$ es continua
    \item Para $\mathbf{F}$ un espacio de Banach y $K\subseteq\mathbf{E}$ un compacto, $\mathcal{C}(K,\mathbf{F})$ es subespacio cerrado de $\mathcal{A}(K,\mathbf{F})$, y, dotado de la norma $\|\cdot\|_{\mathcal{A}}$, es un espacio de Banach
\end{itemize}

\subsubsection{Norma}

Se define la norma en $\mathcal{A}(\mathbf{E},\mathbf{F})$ como

\begin{equation}
\begin{split}
\|f\|_{\mathcal{A}} &= \inf\{M\in\mathbb{R}^+\,|\,\forall x\in
\mathbf{E}:\,\|f(x)\|\leq \mathbf{E}\}\\
&= \sup_{x\in\mathbf{E}}\{\|f(x)\|\}
\end{split}
\nonumber
\end{equation}

\subsection{Convergencia en Sucesiones de Funciones}

\subsubsection{Uniforme} Una sucesión de funciones $(f_n)$ converge uniformemente a $f$ si

\[\lim_{n\to\infty}\|f-f_n\|_{\infty}=0\]

\subsubsection{Puntual} Una sucesión de funciones $(f_n)$ converge puntualmente a $f:A\to B$ si

\[\forall x\in A:\,\lim_{n\to\infty}f_n(x)=f(x)\]

\subsection{Polinomio de Berstein}

Dada $f\in\mathcal{C}([0,1],\mathbb{R})$, se llama polinomio de Bernstein de grado $k$ asociado a $f$, a

\[b_k(X) = \sum^k_{i=0}f\left(\frac{i}{k}\right)p_{k,i}(x)\]
\bigbreak
donde $p_{k,i}$ es el polinomio

\[p_{k,i}(x)=\begin{pmatrix}k\\i\end{pmatrix}x^i(1-x)^{k-i}\]
\bigbreak

Para todo $k\in\mathbb{N}$, se verifica

\begin{equation}
\begin{split}
    &\sum^k_{i=0}p_{k,i}(x)=1\\
    &\sum^k_{i=0}\frac{i}{k}p_{k,i}(x)=x\\
    &\sum^k_{i=0}(\frac{i}{k}-x)^2p_{k,i}(x)=\frac{x(x-1)}{k}\\
\end{split}
\nonumber
\end{equation}

\subsection{Teorema de Weierstrass-Stone}
\label{T:Weierstrass-Stone}
Para toda función $f\in\mathcal{C}([a,b],\mathbb{R})$ existe una sucesión de polinomios que converge uniformemente a $f$.

\subsection{Producto Interno}

Se llama producto interno a toda función $b:\mathbf{E}\times\mathbf{E}\to\mathbb{R}$, que para todo $u,v,w\in\mathbf{E}$ y $\lambda\in\mathbb{R}$ verifique

\begin{itemize}
    \item $b(u,w+v)=b(u,v)+b(u,w)$
    \item $b(u,v)=B(v,u)$
    \item $b(\lambda u,v)=\lambda b(u,v)$
    \item $w\neq 0\Rightarrow b(w,w)>0$
\end{itemize}

Se suele notar los productos internos como $b(x,y)=\langle x,y\rangle$

\subsection{Espacios de Hilbert}

Todo espacio de Banach con una norma definida por un producto interno es un espacio de Hilbert.

\begin{itemize}
    \item Si $\mathbf{E}$ es un espacio de Hilbert, la funciones $\langle\cdot,\cdot\rangle:\mathbf{E}\times\mathbf{E}\to\mathbb{R}$ y $\langle\mu,\cdot\rangle:\mathbf{E}\to\mathbb{R}$ con $\mu\in\mathbf{E}$ fijo, son continuas
\end{itemize}

\subsection{Distancia}

Se define la distancia como

\begin{itemize}
    \item Entre dos puntos: $d(x,y)=\|x-y\|$
    \item Entre un punto y un conjunto: $d(x,A)=\inf_{a\in A}\left\{\|x-a\|\right\}$
    \item Entre dos conjuntos: $d(A,B)=\inf\left\{\|b-a\|\,|\,a\in A, b\in B\right\}$
\end{itemize}

\subsection{Proyección}

Dado $a\in\mathbf{E}$, se llama proyección de $a$ en $A\subseteq\mathbf{E}$ a todo $p\in A$ que cumpla

\[d(a,A) = \|a-p\|\]
\bigbreak
Si $\mathbf{E}$ es un espacio de Hilbert y $A$ cerrado y convexo, entonces $p$ es único, además

\begin{itemize}
    \item $p$ es el único punto en $A$ que verifica
    \[\forall x\in A: \langle a-p, x-p\rangle\leq 0\]
    \item Si $A$ es un subespacio vectorial de $\mathbf{E}$, entonces p es el único punto de $A$ que verifica
    \[\forall x\in A: \langle a-p, x\rangle = 0\]
    \item La función que a todo elemento de $\mathbf{E}$ le hace corresponder su proyección en $A$ es lipschitziana
\end{itemize}

\subsection{Representación de Riesz}
\label{Rep.Riesz}
Sea $\mathbf{E}$ un espacio de Hilbert y $l\in\mathcal{L}(\mathbf{E},\mathbb{R)}$, existe $\omega$ tal que

\[\forall x\in\mathbf{E}:\,l(x)=\langle\omega,x\rangle\]

\begin{itemize}
    \item Si $l\neq 0$, el subespacio dado por $\{x\in\mathbf{E}\,|\,l(x)=0\}$ es un hiperplano
\end{itemize}

\subsection{Separación de Hanh-Banach}

Sean $\mathbf{E}$ un espacio de Hilbert y $C\subseteq\mathbf{E}$ un convexo cerrado, se verifica que

\[\exists l\in\mathcal{L}(\mathbf{E},\mathbb{R})\,\exists a\in\mathbb{R^+}\,\forall x\in C:\,l(x)\geq a\]

\subsection{Lema de Farkas}
\label{L:Farkas}
Dadas $\{l_i\}^n_{i=0}\subseteq\mathcal{L}(\mathbf{E},\mathbb{R)}$ con $\mathbf{E}$ un espacio de Hilbert, si

\[\{x\in\mathbf{E}\,|\,\forall i\in\{1,...,n\}:\,
l_i(x)\leq 0\}\subseteq\{x\in\mathbf{E}\,|\,l_0\leq 0\}\]
\bigbreak
Entonces existe $\{\lambda_i\}^n_{i=1}\subseteq\mathbb{R}^+_0$ tales que

\[l_0 = \sum^n_{i=1}\lambda_il_i\]

\newpage