\subsection{Derivada Parcial}

Dados $f:A\subseteq\mathbf{E}\to\mathbf{F}$, $a\in\mathrm{Int}(A)$ y $d\in\mathbf{E}$, si existe

\[Df(a;d)=\lim_{t\to 0}\frac{f(a+td)-f(a)}{t}\]
\bigbreak
se dice que $Df(a;d)$ es la derivada parcial en dirección $d$ evaluada en $a$ de $f$.
\bigbreak
Para $f:\mathbb{R}^n\to\mathbf{F}$, se tiene que

\[Df(a;\hat{x_i})=\frac{\partial f}{\partial x_i}(a)\]
\bigbreak
\begin{itemize}
    \item Si $\|d\|=1$, $Df(a;d)$ se interpreta como la pendiente de $f$ en $a$ con dirección $d$
    \item Para $\lambda\in\mathbb{R}$ se verifica que
    \[Df(a;\lambda d)=\lambda Df(a;d)\]
\end{itemize}

\subsection{Diferencial}

Se dice que $f:A\subseteq\mathbf{E}\to\mathbf{F}$ es diferenciable en $a\in\mathrm{Int}(A)$ si

\[\exists l\in\mathcal{L}(\mathbf{E},\mathbf{F})\,
\exists\delta\in\mathbb{R}^+
\forall h\in B(0,\delta)\subseteq\mathbf{E}:\,
f(a+h)-(f(a)-l(h))=o(h)\]

entonces se llama a $l$ diferencial de $f$ en $a$ y se nota como

\[l=Df(a)\in\mathcal{L}(\mathbf{E},\mathbf{F})\]
\bigbreak
\begin{itemize}
    \item Si $f\in\mathcal{L}(\mathbf{E},\mathbf{F})$, $f$ es diferenciable en todo su dominio, cumpliendo que $Df(a) = f$
    \item Si existe, el diferencial es único
    \item Si $f$ es diferenciable en $a\in A$, entonces es continua en $A$ y se verifica que
    \[\forall x\in A:\,Df(a)(x) = Df(x;a)\]
    \item Si $f$ y $g$ son diferenciables en $a$, se tiene que
    \[D(f+g)(a)=Df(a)+Dg(a)\]
    \item Si $f:A\subseteq\mathbf{E}\to\mathbb{R}$ y $g:A\subseteq\mathbf{E}\to\mathbb{R}$ son diferenciables en $a\in A$, se tiene que
    \[D(fg)(a)=g(a)Df(a)+f(a)Dg(a)\]
    \item Si $f:A\subseteq\mathbf{E}\to\mathbb{R}$ es diferenciable $a\in A$, con $f(a)\neq 0$, entonces
    \[D\left(\frac{1}{f}\right)(a)=\frac{-1}{f(a)^2}Df(a)\]
    \item Si $f = (f_1, f_2, ..., f_n)$ es diferenciable, entonces
    \[Df(a)= (Df_1(a), Df_2(a), ..., Df_n(a))\]
    \item Si $f:A\subseteq\mathbb{R}^n\to\mathbf{F}$ es diferenciable, se verifica que
    \[\forall  h\in\mathbb{R}^n:\, Df(a)(h) =
    \sum^n_{i=1}h_i\frac{\partial f}{\partial x_i}(a)\]
\end{itemize}

\subsubsection{Composición de Funciones}

Dadas $f:\mathbf{G}\to\mathbf{F}$ y $g:\mathbf{E}\to\mathbf{G}$ funciones diferenciables en $a\in\mathbf{G}$, se verifica que

\[D(f\circ g)(a) = Df(g(a))\circ Dg(a)\]
\bigbreak
Para evitar confusiones cabe notar que $D(f\circ g)(a)$ y $Df(g(a))$ no son lo mismo, el primero es el diferencial de $f\circ g$ en $a$, mientras que el segundo es el diferencial de $f$ en $g(a)$.

\subsection{Gradiente}

Sea $f:A\subseteq\mathbb{R}^n\to\mathbb{R}$, con $A$ abierto, diferenciable en $a\in A$, se llama gradiente de $f$ en $a$ al vector

\[\nabla f = \sum^n_{i=1}\frac{\partial f}{\partial x_i}(a)\hat{x_i}\]

\bigbreak
Se puede redefinir el diferencial de $f$ en $a$ evaluado en $h\in\mathbb{R}$ como el producto punto entre el gradiente de $f$ en $a$ y $h$

\[Df(a)(h) = \langle\nabla f(a), h\rangle = \nabla f(a)\cdot h\]

\subsection{Jacobiano}

Sea $f:A\subseteq\mathbb{R}^n\to\mathbb{R}^m$, con $A$ abierto, diferenciable en $a\in A$, se llama jacobiano de $f$ en $a$ a la matriz

\begin{minipage}{0.55\textwidth}
\begin{equation}
\mathrm{J}f(a) = \begin{pmatrix}
\frac{\partial f_1}{\partial x_1}(a) &\cdot &\cdot &\cdot
& \frac{\partial f_1}{\partial x_n}(a)\\
\cdot & & & & \cdot\\
\cdot & & & & \cdot\\
\cdot & & & & \cdot\\
\frac{\partial f_m}{\partial x_1}(a) &\cdot &\cdot &\cdot
& \frac{\partial f_m}{\partial x_n}(a)\\
\end{pmatrix}
\nonumber
\end{equation}
\end{minipage}
\begin{minipage}{0.55\textwidth}
\begin{equation}
\left(\mathrm{J}f(a)\right)_{ij} = 
\frac{\partial f_i}{\partial f_j}(a)
\nonumber
\end{equation}
\end{minipage}

Se puede redefinir el diferencial de $f$ en $a$ evaluado en $h\in\mathbb{R}$ como el producto de matrices entre el jacobiano de $f$ en $a$ y $h$

\[Df(a)(h) = \mathrm{J}f(a)h\]

\subsection{Teorema del Valor Medio}
\label{T:ValorMedio}
$\,$\smallbreak
Dados $a,b\in\mathbf{E}$ se define el conjunto $[a,b]\subseteq\mathbf{E}$ como

\[[a, b]=\{a+t(b-a)\,|\,t\in [0,1]\subseteq\mathbb{R}\}\]
\bigbreak

Sea $f:[a,b]\subseteq\mathbf{E}\to\mathbb{R}$ diferenciable, existe $c\in [a, b]$ tal que

\[f(b)-f(a) = Df(c)(b-a)\]
\bigbreak

\begin{itemize}
    \item Dada $f:A\subseteq\mathbf{E}\to\mathbf{F}$ si $f$ es diferenciable en $[a,b]\subseteq A$ y existe $L\in\mathbb{R}^+$ tal que
    \[\forall x\in [a,b]:\,L\geq \|Df(x)\|\]
    entonces, se cumple que
    \[\|f(b)-f(a)\|\leq L\|b-a\|\]
\end{itemize}

\subsection{Conjuntos Convexos}

Se dice que $C\subseteq\mathbf{E}$ es convexo si

\[\forall a,b \in C:\,[a,b]\subseteq C\]
\bigbreak
\begin{itemize}
    \item Sea $f:C\to\mathbf{F}$ diferenciable en $C$, si existe $L\in\mathbb{R}^+$ tal que, para todo $x\in C$, $L\geq\|Df(x)\|$, entonces
    \[\forall x,y\in C:\,\|f(x)-f(y)\|\leq L\|x-y\|\]
\end{itemize}

\subsection{Conjuntos Conexos}

Se dice que $Q\subseteq\mathbf{E}$ es conexo si no existen $O_1,O_2\subseteq\mathbf{E}$ tales que

\begin{equation}
\begin{split}
    & O_1 \cap C \neq \emptyset\\
    & O_2 \cap C \neq \emptyset\\
    & C \subseteq O_1 \cup O_2\\
    & O_1 \cap C \cap O_2 = \emptyset\\
\end{split}
\nonumber
\end{equation}

\begin{itemize}
    \item Todo convexo es conexo
    \item Si $f:\mathbf{E}\to\mathbf{F}$ es continua y $C\subseteq\mathbf{E}$ un conexo, entonces $f(C)$ es conexo
    \item Sea $f:C\subseteq\mathbf{E}\to\mathbf{F}$ diferenciable con $C$ conexo, si para todo $x\in C$ se verifica que $Df(x)=0$, entonces $f$ es constante en $C$
\end{itemize}

\subsection{Funciones Continuamente Diferenciables}

Se dice que una función $f:A\subseteq\mathbf{E}\to\mathbf{F}$ es continuamente diferenciable si $Df$ es continua.

\subsection{Funciones de Clase $\mathcal{C}^1$}

Se dice que una función $f:A\subseteq\mathbb{R}^n\to\mathbf{F}$ es de clase $\mathcal{C}^1$ si

\[\forall i\in\{1,...,n\}:\,\frac{\partial f}{\partial x_i}\in \mathcal{C}(A,\mathbf{F})\]

\begin{itemize}
    \item $f\in\mathcal{C}^1(A,\mathbf{F})$ si y sólo si $f$ es continuamente diferenciable
\end{itemize}

\newpage