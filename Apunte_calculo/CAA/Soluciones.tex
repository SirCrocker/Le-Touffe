\subsection{Soluciones}

\subsubsection{Cálculo Vectorial}

\sol{1}\\
\bigbreak

En coordenadas esféricas se tiene que

\begin{equation}
\begin{split}
    &\sqrt{x^2+y^2+z^2}=r\\
    &\sqrt{x^2+y^2}= r\sin{\theta}
\end{split}
\nonumber
\end{equation}

De modo que la función $f$ expresada en coordenadas esféricas es

\[f(r,\phi,\theta)=\theta r\]

facilitando así el cálculo del gradiente

    \[\nabla f = \frac{\partial f}{\partial r}\hat{r}
    +\frac{1}{r}\frac{\partial f}{\partial \theta}\hat{\theta}=\theta\hat{r}+\hat{\theta}\]

\bigbreak
\bigbreak

\sol{2}\\
\bigbreak
Considerando $\Omega$ como la esfera de radio $R$ centrada en el origen, por teorema de la divergencia se tiene que

\begin{equation}
\begin{split}
    \oint_\mathcal{S} g\nabla f\cdot d\Vec{S}&=\int_\Omega
    \nabla\cdot\left(g\nabla f\right)\,dV\\
    &= \int_\Omega g\nabla^2 f+\nabla g\cdot\nabla f\,dV\\
    &= \int_\Omega g\nabla^2 f\,dV\\
    &= \int_\Omega\,dV\\
    &=\frac{4\pi R^3}{3}
\end{split}
\nonumber
\end{equation}

\newpage

\sol{3}\\
\bigbreak

\begin{enumerate}[label=\alph*)]
    \item Orientado según la norma positiva, el borde de $A$ es la curva que recorre $y=x^3$ desde $x=0$ a $x=1$ y luego $y=x^2$ desde $x=1$ a $x=0$. Calculada directamente la integral es
    \begin{equation}
    \begin{split}
        \int_{\partial A} y^3\,dx + 2xy^2\,dy &= \int^1_0x^9\,dx+\int^0_1x^6\,dx+\int^1_02y^{\frac{7}{3}}\,dy+\int^0_12y^{\frac{5}{2}}\,dy\\
        &= \frac{1}{10}-\frac{1}{7}+\frac{6}{10}-\frac{4}{7}\\
        &= \frac{7}{10}-\frac{5}{7}\\
        &= -\frac{1}{70}\\
    \end{split}
    \nonumber
    \end{equation}
    Usando el teorema de Green se tiene
    \begin{equation}
    \begin{split}
        \int_{\partial A} y^3\,dx + 2xy^2\,dy &=
        \int_A 2y^2-3y^2\,dxdy\\
        &= \int_A -y^2\,dxdy\\
        &= \int^1_0\int^{x^2}_{x^3}-y^2\,dydx\\
        &= \int^1_0 \frac{x^9}{3}-\frac{x^6}{3} \,dx\\
        &= \frac{1}{30}-\frac{1}{21}\\
        &= -\frac{1}{70}\\
    \end{split}
    \nonumber
    \end{equation}
    \item Orientado según la norma positiva, el borde de $B$ es la curva que recorre $y=x^2$ desde $x=0$ a $x=1$ y luego $y=\sqrt{x}$ desde $x=1$ a $x=0$. Calculada directamente la integral es
    \begin{equation}
    \begin{split}
        \int_{\partial B} (x+y)^2\,dx + (2xy+x^2-y^3)\,dy &= \int^1_0(x+x^2)^2\,dx+\int^0_1(x+\sqrt{x})^2\,dx\\
        &\,\,\,\,\,\,+\int^1_02y^{\frac{3}{2}}+y-y^3\,dy+
        \int^0_1y^3+y^4\,dy\\
        &= \frac{1}{3}+\frac{1}{2}+\frac{1}{5}-\frac{1}{3}-
        \frac{4}{5}-\frac{1}{2}+\frac{4}{5}+\frac{1}{2}-
        \frac{1}{4}-\frac{1}{4}-\frac{1}{5}\\
        &= \frac{1}{2}-\frac{1}{4}-\frac{1}{4}\\
        &=0\\
    \end{split}
    \nonumber
    \end{equation}
    Usando el teorema de Green se tiene
    \begin{equation}
    \begin{split}
        \int_{\partial B} (x+y)^2\,dx + (2xy+x^2-y^3)\,dy &=
        \int_B 2(x+y)-2y-2x\,dxdy\\
        &= \int_B 0\,dxdy\\
        &= 0\\
    \end{split}
    \nonumber
    \end{equation}
\end{enumerate}

\newpage