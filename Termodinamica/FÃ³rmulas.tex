\section{Fórmulas y Ecuaciones}
\textbf{Importante:} Las identidades dependen de las variables que definen a la función, estas son las mostradas como subíndices en la derivada parcial y con respecto a la que se deriva. 
\subsection{Ecuación de los Gases Ideales}

\[pV=Nk_BT\]
\bigbreak
\begin{itemize}
    \item Es válida en el equilibrio
    \item Con $n$ el número de moles y $R$ la constante de los gases, se puede expresar como
    \[pV=nRT\]
\end{itemize}

\subsection{Capacidad Calorífica o térmica}

\[C=\frac{\di Q}{dT}\]

\txtsi{\textbf{En equilibrio se tiene que:}}{(\ref{Cequilibrio})}
% Aparece en el Blundell, p.108 | Mala mía
\[C=\frac{\partial E}{\partial T}+\left(
\frac{\partial E}{\partial V}+p\right)\frac{dV}{dT}\]

\begin{itemize}
    \item Volumen constante:
    \[C_V=\mcte{V}{\frac{\di Q}{d T}}=\devtermo{V}{E}{T} = T\devtermo{V}{S}{T}\]
    \item Presión constante: % Apunte VF página 105, por eso lo cambié
    \[C_p = \mcte{p}{\frac{\di Q}{d T}}=\devtermo{p}{E}{T}+p\mcte{p}{\parfrac{V}{T}} = \devtermo{p}{H}{T} = T\devtermo{p}{S}{T}\]
    \item Relación entre ambas constantes:
    \[C_p = C_V + T\devtermo{V}{p}{T}\cdot \devtermo{p}{V}{T}\]
    \item Para un gas ideal:
    \[C_p = C_V + NK_B \quad\] Con $NK_B = R$ si es que $N = N_A$ (número de Avogrado)
\end{itemize}

% + \left(
% \frac{\partial E}{\partial V}+p\right)\left(
% \frac{\partial V}{\partial T}\right)=C_V+R\]

\subsection{Índice Adiabático}

\[\gamma = \frac{C_p}{C_V} = \frac{C_V+R}{C_V}\]
\bigbreak
\textbf{Gas ideal monoatómico:}

\[\gamma = \frac{5}{3}\]
\bigbreak
\textbf{Gas ideal diatómico:}

\begin{itemize}
    \item Traslacional y rotacional
    \[\gamma = \frac{7}{5}\]
    \item Traslacional, rotacional y vibracional
    \[\gamma = \frac{9}{7}\]
\end{itemize}

\textbf{Relación para gases ideales en proceso adiabático}: para todo $p$, $V$ y $T$ en un proceso adiabático

\[TV^{\gamma -1} = \text{cte}_1 \quad \land \quad pV^\gamma = \text{cte}_2\]

\subsection{Energía Interna}
$\,$\\

\txtsi{\textbf{Primera Ley:}}{(\ref{1ley})}


\[\Delta E = W+Q\]
\[dE = \di W+ \di Q\]

\begin{itemize}
    \item Si el sistema está en equilibrio
    \[dE = -pdV + TdS\]
    \item Si la temperatura final es igual a la inicial $\Delta E=0$
\end{itemize}

\bigbreak

\textbf{Función del volumen y la entropía}: Si $E=E(V,S)$

\begin{equation}
\begin{split}
    T&=\mcte{V}{\parfrac{E}{S}}\\
    p&=-\mcte{S}{\parfrac{E}{V}}\\
\end{split}
\nonumber
\end{equation}

\bigbreak

\textbf{Sistemas cerrados:} Si $N$ es constante y $S$ depende de la temperatura y el volumen, se tiene

\[dE = C_VdT - \lados{[}{p-T\devtermo{V}{p}{T}}dV\]

\textbf{Gas Ideal:}

\[E=\frac{Nk_BT}{\gamma-1}\]


\subsection{Entropía}
\textit{El cambio de entropía no depende del camino seguido, es decir solo depende de los estados inicial y final.}\\

\textbf{Entropía de Boltzmann:}

\[S = k_B\ln{\Omega}\]
\bigbreak

\txtsi{\textbf{Relación entre entropía y temperatura:}}{(\ref{DTemperatura})} En equilibrio se tiene que

\[\left( \frac{\partial S}{\partial E} \right)_{V} \equiv \frac{1}{T} \quad \land \quad T = \devtermo{N,V}{E}{S}\]

\[dS = \frac{1}{T} \di Q\]

\[\Delta S = \int \frac{1}{T}\,\di Q = \int\frac{1}{T}\,dE+
\int\frac{p}{T}\,dV\]

\bigbreak

\textbf{Entropía con un Baño Térmico:}

\[\Delta S_B=-\frac{Q}{T}\]
\[\Sigma = \Delta S+\Delta S_B=\Delta S-\frac{Q}{T}\]
\[\Sigma \geq 0 \text{ }^{(\ref{2ley})}\]
\bigbreak
\textbf{En sistemas cerrados:} Si $dN = 0$ y $S = S(E, V)$ entonces

\[\mcte{E}{\frac{\partial S}{\partial V}} = 
\frac{-1}{T}\mcte{S}{\frac{\partial E}{\partial V}}=\frac{p}{T}\]

\[\devtermo{V}{S}{T}=\frac{C_V}{T}\]

\[dS = \frac{C_V}{T}+\devtermo{V}{p}{T}dV\]

\bigbreak

\textbf{Ecuación de Sackur-Tetrode:} Para un gas ideal monoatómico se cumple \txtsi{que}{(\ref{eq:e(s,v,n)})}

\[S=k_BN\ln{\left(\frac{V}{N}\left(\frac{4\pi m}{3h^2}
\frac{E}{N}\right)^{(3/2)}\right)}+\frac{5}{2}\]
\bigbreak
Donde $m$ es la masa de una partícula de gas y $h$ la constante de Planck.

\[\Delta S = Nk_B\ln{\left(\frac{V_f}{V_i}\left(
\frac{E_f}{E_i}\right)^{(3/2)}\right)}=Nk_B\ln{\left( \frac{V_f}{V_i}\left(\frac{T_f}{T_i}\right)^{(3/2)}\right)}\]

\subsection{Procesos Reversibles}

Los índices $d$ y $r$ refieren a los dos sentidos en los que se puede realizar el proceso.

\begin{equation}
\begin{split}
    &\Sigma = 0\\
    &\Delta E_r = -\Delta E_d\\
    &\Delta S_r = \frac{Q_r}{T}=-\frac{Q_d}{T}=\Delta S_d\\
    &Q_d+Q_r=0\\
    &W_r+W_d=0\\
\end{split}
\nonumber
\end{equation}

En los procesos irreversibles se tiene que

\[Q_d+Q_r=-T\left(\Sigma_d+\Sigma_r\right)<0\]

\subsection{Ciclos}

\[\Delta E = 0\]
\[\Delta S = 0\]
\bigbreak
\txtsi{\textbf{1 baño térmico:}}{\ref{TQciclos}}
\[W=T\Sigma\geq 0\]
\bigbreak

\txtsi{\textbf{2 baños térmicos:}}{\ref{TQciclos}}
\begin{equation}
\begin{split}
    &\Sigma = -\left(\frac{Q_1}{T_1}+\frac{Q_2}{T_2}\right)\\
    &W=-(Q_1+Q_2)\\
    &Q_1 = -Q_2\frac{T_1}{T_2}-\Sigma T_1\\
    &W = -Q_2\left(\frac{T_1}{T_2}-1\right)+T_1\Sigma\\
\end{split}
\nonumber
\end{equation}
\bigbreak

\textbf{Motor:}
\begin{equation}
\begin{split}
    &W<0 \quad \text{(Se extrae trabajo del sistema)}\\
    &Q_1 <0 \quad \text{(El baño frío recibe energía)}\\
    &Q_2>0 \quad \text{(El baño caliente entrega energía)}\\
\end{split}
\nonumber
\end{equation}
\bigbreak

\textbf{Rendimiento de un motor:}
\begin{equation}
\begin{split}
    &\eta = -\frac{W}{Q_2}= 1+\frac{Q_1}{Q_2}  \text{ }^{(\ref{plbr-motor})}\\
    &0\leq\eta\leq 1\\
    &\eta_c=1-\frac{T_1}{T_2}\\
    &\eta=\eta_c-\Sigma\frac{T_1}{Q_2}\\
    &\eta < \eta_c \text{ }^{(\ref{2ley-motor})}\\% Según C2 2017-2 P2 [Barra], el rendimiento de Carnot es inalcanzable
\end{split}
\nonumber
\end{equation}
\bigbreak

\textbf{Refrigerador (``motor inverso'' o Bomba térmica):}
\begin{equation}
\begin{split}
    &W>0 \quad \text{(Se hace trabajo sobre el sistema)}\\
    &Q_1 >0 \quad \text{(El baño frío entrega energía)}\\
    &Q_2<0 \quad \text{(El baño caliente recibe energía)}\\
\end{split}
\nonumber
\end{equation}
\bigbreak

%En caso de refrigerados se denomina eficiencia
\textbf{Eficiencia de un refrigerador:} \txtsi{ }{(\ref{refri-q2-w})}
\begin{equation}
\begin{split}
    &\eta^R = \frac{Q_1}{W} = \lados{)}{\frac{T_2}{T_1} - 1 + \frac{T_2}{T_1}\Sigma}^{-1}\\
    &\eta^R_c=-\frac{Q_1}{Q_1+Q_2}=-\frac{1}{1+Q_2/Q_1}
    =\frac{1}{T_2/T_1-1}\\
    &\eta^R\leq\eta^R_c \quad \land \quad 0 < \eta^R_c < \infty\\
\end{split}
\nonumber
\end{equation}

\subsection{Trabajo y Calor}

\textbf{Diferenciales:} En procesos cuasi-estáticos se cumple que

\begin{equation}
\begin{split}
    &\di W = -pdV\\
    &\di Q = TdS\\
\end{split}
\nonumber
\end{equation}

\txtsi{\textbf{Temperatura final igual a inicial:}}{\ref{DT=0}}

\begin{equation}
\begin{split}
    \Delta T = 0 &\Rightarrow W = T(\Sigma-\Delta S)\\
    &\Rightarrow W\geq-T\Delta S\\
\end{split}
\nonumber
\end{equation}

\subsection{Potenciales termodinámicos y variables naturales}
\textit{Las relaciones de Maxwell correspondientes están en teoría}\txtsi{ }{(\ref{relaciones-maxwell})}


\begin{minipage}{0.5\textwidth}
    \begin{equation}
        \begin{split}
            E &= E(S,V)^{(\ref{eq:e(s,v,n)})}\\
            F &= F(T,V) = E - TS\\
            H &= H(S,p) = E + pV\\
            G &= G(p,T) = E - TS + pV
        \end{split}
        \nonumber
    \end{equation}
\end{minipage}%
\begin{minipage}{0.5\textwidth}
    \begin{equation}
        \begin{split}
            dE &= TdS - pdV\\
            dF &= -pdV - SdT\\
            dH &= TdS + Vdp\\
            dG &= Vdp - SdT
        \end{split}
        \nonumber
    \end{equation}
\end{minipage}%
    
% Se podría agregar las relaciones de Maxwell aquí, tal vez hacer secciones separadas con N como variable si es que cambia algo. No se aún

\newpage