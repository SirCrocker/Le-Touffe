\subsection{Soluciones}

\begin{solucion}{1}
\ics a
En coordenadas cilíndricas, la velocidad de una carga dentro del disco es

\[\Vec{v}=\rho\dot{\phi}\hat{\phi}=\rho\omega\hat{\phi}\]

con lo cual la densidad de corriente es

\[\Vec{K}(\rho) = \sigma\omega\rho\hat{\phi}\]

\ics b
En coordenadas esféricas, la velocidad de una carga de la esfera es

\[\Vec{v}=\dot{r}\hat{r} + r(\dot{\phi}cos(\theta)\hat{\phi}+\dot{\theta}\hat{\theta}=
r\omega\cos(\theta)\hat{\phi}\]

además, la densidad de carga volumétrica es

\[\rho = \frac{3Q}{4\pi R^2}\]

con lo que la densidad de corriente está dada por

\[\Vec{J}(r,\phi,\theta)=\frac{3Q}{4\pi R^2}r\omega\cos(\theta)\hat{\phi}\]

\end{solucion}

\bigbreak

\begin{solucion}{2}
\ics a 
Se tiene que para un cilindro, su resistencia viene dada por su conductividad/resistividad ($g/\eta$), altura ($h$) y superficie ($S$) de la siguiente manera
\[R = \frac{1}{g}\frac{h}{S} = \eta\frac{h}{S}\]

Como fue demostrado en la sección de resultados útiles \ref{ru:resist_cilindro}.\\

Con esto en mente, se puede considerar que un cono truncado es equivalente a una suma infinitesimal de cilindros de altura $dh$, con superficies que varían desde el radio menor del cono truncado al radio mayor. Así

\[dR = \eta \frac{dh}{dS} = \eta \frac{dh}{\pi r^2dr}\]

Donde $R = \int dR$, estableciendo el origen en la cara superior del cono truncado intersectado con su eje \cita{mirando hacia abajo}, la resistencia del cono truncado es

\begin{eqit}
    R &= \int dR\\
    &= \int_0^h \int_{a}^{a+b} \frac{\eta}{\pi}\frac{1}{r^2}dr\,dl\\
    &= \frac{\eta}{\pi}\int_0^h\ dl \int_{a}^{a+b} \frac{1}{r^2}dr\\
    &= \frac{\eta}{\pi}h\lados{[}{-\frac{1}{r}}\Big|_a^{a+b} = \frac{h \eta}{\pi}\frac{b}{a(a+b)}
\end{eqit}
% No sé
% Cómo es E?
% Normal a las caras? con forma de cono?
% Sobre que superficie se integra para la intensidad?
% En todos los discos entre A y B?
% R depende de la altura?

% Puse algo que creo que posee sentido, busqué en internet y encontré algo similar. Para la b) creo que hay usar la misma idea, pero superponiendo cilindros de "resistencia negativa" y dividiendolos en la sección que corresponde. 

\ics b
% Más de lo mismo
% Supongo que cuando dice "entre las caras" hay que asumir que solo en esas cara se distribuye carga

% Yep, que cruza de una a la otra cara solamente

\end{solucion}

\bigbreak

\begin{solucion}{3}
\ics a
Como la densidad de corriente es uniforme, la intensidad es

\[I =\int\Vec{J}\cdot d\Vec{S} = \int J\,dS =J\int dS = \pi a^2J\]

y, por ley de Ohm, para una diferencia de potencial $V_0$ se verifica que

\begin{equation}
\begin{split}
    V_0 &= -\int^0_{l\hat{x}}\Vec{E}\cdot d\Vec{r}\\
    &= -\int^0_{l\hat{x}}\frac{1}{g}\Vec{J}\cdot d\Vec{r}\\
    &= \int^l_0\frac{e^{-x/l}}{g_o}J\,dx\\
    &= \frac{Jl}{g_o}(1-e^{-1})\\
\end{split}
\nonumber
\end{equation}

de lo cual se deduce

\begin{equation}
\begin{split}
    &\Vec{J}=\frac{V_0g_o}{l(1-e^{-1})}\hat{x}\\
    &I=\frac{\pi a^2V_0g_o}{l(1-e^{-1})}\\
    &R=\frac{V_0}{I}=\frac{l(1-e^{-1})}{\pi a^2 g_o}\\
    &\Vec{E}=\frac{V_0e^{-x/l}}{l(1-e^{-1})}\hat{x}\\
\end{split}
\nonumber
\end{equation}

\ics b
Como en $l$ el potencial es nulo, se tiene que

\begin{equation}
\begin{split}
    V(x) &= V(x) - V(l)\\
    &= -\int^{x\hat{x}}_{l\hat{x}}\Vec{E}\cdot d\Vec{r}\\
    &= -\int^x_l\frac{V_0e^{-x'/l}}{l(1-e^{-1})}\,dx'\\
    &= \frac{V_0(e^{-x/l}-e^{-1})}{(1-e^{-1})}\\
    &= \frac{V_0(e^{1-x/l}-1)}{(e-1)}\\
\end{split}
\nonumber
\end{equation}
% no sé hacer esto sin permitividad 
% Creo que la permitividad es \epsilon_0, ya que es un conductor. 

% Este problema está en C2-Montesinos 2017-1 y ocupan epsilon_0, pondré lo que me dio

\ics c 
La carga libre en el sistema es toda la carga que puede fluir por este, se puede obtener usando la ecuación de Maxwell $\nabla\cdot\Vec{E} = \rho/\epsilon_0$, al ser un conductor no dieléctrico, y luego integrando $\rho$ en el volumen del cilindro, así

\begin{eqit}
    \rho &= \epsilon_0\nabla \cdot \Vec{E}\\
    &= \epsilon_0\parfrac{E(x)}{x}\\
    &= \epsilon_0\lados{(}{\frac{V_0}{l(1-e^{-1})}\frac{-1}{l}e^{-x/l}}
\end{eqit}

Con el valor de $\rho$ obtenido se integra en todo el volumen haciendo uso de coordenadas cilíndricas con $x$ en vez de $z$,

\begin{eqit}
    Q &= \int_\V \rho d\V\\
    &= \int_0^a \int_0^{2\pi} \int_0^l\rho(x) r\,dx\,d\phi\,dr\\
    &= \pi a^2 \int_0^l \epsilon_0\frac{V_0}{l(1-e^{-1})}\frac{-1}{l}e^{-x/l}\,dx\\
    &= \epsilon_0 \pi a^2 \frac{V_0}{l(1-e^{-1})} \lados{[}{e^{-x/l}}\Big|_0^l\\
    &= \epsilon_0 \pi a^2 \frac{V_0}{l(1-e^{-1})} (e^{-1} - 1)\\
    &= -\epsilon_0 \pi a^2 \frac{V_0}{l}
\end{eqit}

% Que la carga sea negativa calza teoricamente, no? 

% Sapone dice en su apunte de electro página 69: "Si los portadores de carga son negativos (como en los conductores), [...]" lo que me hace pensar que si calza

Si los conductores están en equilibrio, los campos en su interior son nulos, por lo tanto, debido a las condiciones de borde del desplazamiento eléctrico, las densidades de carga superficial en los extremos son

\begin{eqit}
    &\sigma_o = \epsilon_o E(0) = \frac{\e_oV_0e}{l(e-1)}\\
    &\sigma_l = -\epsilon_o E(l) = -\frac{\e_oV_0}{l(e-1)}\\
\end{eqit}

Luego, las cargas superficiales son

\begin{eqit}
    &Q_o = \int \sigma_o\,dS = \sigma_o\int dS = \frac{\e_oV_0e\pi a^2}{l(e-1)}\\
    &Q_l = \int \sigma_o\,dS = -\frac{\e_oV_0\pi a^2}{l(e-1)}\\
\end{eqit}

La carga libre total es

\[Q_L = Q+Q_o+Q_l = \frac{\e_oV_0\pi a^2}{l}\lados{(}{\frac{e}{e-1}-\frac{1}{e-1}-1}=0\]
% Gracias por escribirlo y agregarlo c:
\ics d
La potencia disipada en el sistema es

\[P = \frac{V_0^2}{R} = \frac{g_oV_o^2\pi a^2}{l(1-e^{-1})}\]

Como $P$ no es nula, existe energía disipada a lo largo del tiempo junto con variación de energía electrostática.

% a dónde va y de dónde viene no hace falta responderlo.

% Si hay una fuente, no se tendría que la energía disipada sería proveída por la fuente, causando que U no cambiaría pero si habría disipación de energía? (Así además se mantendría la corriente estacionaria)

\ics e % ahora resulta que había una fuente | xddd
Si se desconecta la fuente, dejaría de ingresar energía al sistema y eventualmente toda la energía del mismo se disiparía en forma de calor. Esto implica que $\Delta V = 0$, es decir, las cargas se reordenan homogéneamente 

\ics f
El periodo transitorio es el tiempo entre que la fuente se desconecta, y la corriente deja de fluir por el cilindro. Sabiendo que cuando deja de fluir corriente (o hay una diferencia de potencial igual a cero), no hay energía en el sistema a causa de que ya no hay campos eléctricos presentes, se tiene que la variación de energía será el opuesto aditivo de la energía electroestática presente en el sistema. \\

Además, toda la energía electroestática presente en el periodo estacionario será disipada en forma de calor por el Efecto Joule cuando se desconecte la fuente.

\begin{eqit}
    \Delta U &= U_{fin} - U_{estacionario}\\    
    &=0 - \frac{\epsilon_o}{2}\int \norma{\Vec{E}}^2\, d\V\\
    &= -\frac{\e_0}{2}\int \lados{(}{\frac{V_0e^{-x/l}}{l(1-e^{-1})}}^2\,d\V\\
    &= -\frac{\e_0}{2}\frac{V_0^2}{l^2(1-e^{-1})^2}\int\,dS_{cara}\int_0^le^{-2x/l}\,dx\\
    &= -\frac{\e_0}{2}\frac{V_0^2\pi a^2}{l^2(1-e^{-1})^2}\lados{[}{ \frac{l}{2}-\frac{e^{-2}l}{2}}\\
    &= \frac{\e_0V_0^2\pi a^2}{l(1-e^{-1})^2}\lados{[}{\frac{e^{-2} - 1}{4}}
\end{eqit}

Es importante notar que en este caso se hace uso de está definición de la energía electroestática del sistema a causa de que la permivitidad del conductor es $\epsilon_0$. Y como solo hay campos eléctricos presentes en el cilindro con resistencia\indeciso{M}{ (\ref{teo:conductores_perfectos})}, entonces se puede integrar solo en este.

\indeciso{E}{
Por ecuación de continuidad se tiene que

\[\nabla\cdot\Vec{J}=\parfrac{J}{x}=-\parfrac{\rho}{t}\]

Desarrollando la ecuación

\begin{equation}
\begin{split}
  -\parfrac{\rho}{t} &= \nabla\cdot\Vec{J}\\
  &=\parfrac{J}{x}\\
  &=\parfrac{gE}{x}\\
\end{split}
\nonumber
\end{equation}
}

\end{solucion}

\bigbreak

\begin{solucion}{4}
\ics a
Inmediatamente después de conectar el potencial no ha habido movimiento de cargas, $\rho = 0$.\\

Por geometría del sistema (\ref{SimetríaPlanosInf}), se tiene que $\Vec{D}(\Vec{r})=D(z)\hat{z}$ y, por primera ecuación de Maxwell, que $\nabla\cdot\Vec{D}=0$, por lo tanto el desplazamiento eléctrico es

\[\Vec{D}=D\hat{z}\]

con $D$ constante. Luego,

\begin{equation}
\begin{split}
    V_0 &= -\int^0_{(a+b)\hat{z}}\Vec{E}\cdot d\Vec{r}\\
    &= -\int^0_{a+b}E\,dz\\
    &= -\int^0_a\frac{D}{\epsilon}\,dz
    -\int^a_{a+b}\frac{D}{\epsilon_o}\,dz\\
    &= \frac{aD}{\epsilon}+\frac{bD}{\epsilon_o}\\
\end{split}
\nonumber
\end{equation}

se concluye que

\[\Vec{D}=\frac{V_0}{\frac{a}{\epsilon}+\frac{b}{\epsilon_o}}\hat{z}
=\frac{V_0\epsilon_o\epsilon}{a\epsilon_o+b\epsilon}\hat{z}\]

Para $0<z<a$

\begin{equation}
\begin{split}
    &\Vec{E}_a=\frac{V_0\epsilon_o}{a\epsilon_o+b\epsilon}\hat{z}\\
    &\Vec{J}_a=\frac{gV_0\epsilon_o}{a\epsilon_o+b\epsilon}\hat{z}\\
\end{split}
\nonumber
\end{equation}

Para $a<z<a+b$

\begin{equation}
\begin{split}
    &\Vec{E}_b=\frac{V_0\epsilon}{a\epsilon_o+b\epsilon}\hat{z}\\
    &\Vec{J}_b=0\\
\end{split}
\nonumber
\end{equation}

\ics b
Después de un tiempo largo se alcanza el estado estacionario, es decir, $\parfrac{\rho}{t}=0$. Por ley de continuidad la densidad de carga es uniforme.\\

En $0<z<a$ se tiene

\begin{equation}
\begin{split}
    &\Vec{J}_a=J_a\hat{z}\\
    &\Vec{E}_a=\frac{J_a}{g}\hat{z}\\
    &\Vec{D}_a=\epsilon\frac{J_a}{g}\hat{z}\\
\end{split}
\nonumber
\end{equation}

Por condición de borde, para $z=a$, se cumple que

\[J_b-J_a=0\]

como en $a<z<a+b$ no hay conductividad, $J_b$ es nula, lo que implica

\begin{equation}
\begin{split}
    &\Vec{J}_a=0\\
    &\Vec{E}_a=0\\
    &\Vec{D}_a=0\\
\end{split}
\nonumber
\end{equation}

En $a<z<a+b$ se tiene

\begin{equation}
\begin{split}
    &\Vec{E}_b=\frac{D_b}{\epsilon_o}\hat{z}\\
    &\Vec{D}_b=D_b\hat{z}\\
\end{split}
\nonumber
\end{equation}

luego,

\begin{equation}
\begin{split}
    V_0 &= -\int^0_{a+b}\Vec{E}\cdot\hat{z}\,dz\\
    &= -\int^0_a E_a\,dz- \int^a_{a+b}\frac{D_b}{\epsilon_o}\,dz\\
    &= \frac{D_b}{\epsilon_o}b
\end{split}
\nonumber
\end{equation}

se concluye que

\begin{equation}
\begin{split}
    &\Vec{E}_b=\frac{V_0}{b}\hat{z}\\
    &\Vec{D}_b=\frac{V_0\epsilon_o}{b}\hat{z}\\
\end{split}
\nonumber
\end{equation}

\ics c
Por las ecuaciones de borde, se verifica que

\begin{equation}
\begin{split}
    &J_a(t) = \parfrac{\sigma}{t}\\
    &D_b(t) - D_a(t) = D_b(t) - \frac{\epsilon}{g}J_a(t) = \sigma\\
\end{split}
\nonumber
\end{equation}

donde $D_b$ y $J_a$ son constantes en el espacio y variables en el tiempo. Se tienen así dos ecuaciones y tres incógnitas: $D_b$, $J_a$ y $\sigma$. La tercera ecuación se obtiene de la diferencia de potencial.

\begin{equation}
\begin{split}
    V_o &= -\int^0_aE_a\,dz -\int^a_{a+b}E_b\,dz\\
    &= -\int^0_a\frac{J_a}{g}\,dz
    -\int^a_{a+b}\frac{D_b}{\epsilon_o}\,dz\\
    &= \frac{J_a}{g}a + \frac{D_b}{\epsilon_o}b\\
\end{split}
\nonumber
\end{equation}

Despejando $D_b$ se obtiene

\[D_b=\frac{\epsilon_o}{g}\lados{(}{V_0-\frac{J_a}{g}a}\]

reemplazando $D_b$ en la ecuación de borde para el desplazamiento eléctrico y derivando respecto al tiempo, se llega a la EDO

\[-\frac{a\epsilon_o+b\epsilon}{bg}\parfrac{J_a}{t}
=\parfrac{\sigma}{t}=J_a\]

cuya solución es de forma

\[Ae^{-\frac{bg}{a\epsilon_o+b\epsilon}t}\]

de la parte $a)$ se deduce que

\[J_a = \frac{gV_0\epsilon_o}{a\epsilon_o+b\epsilon}
e^{-\frac{bg}{a\epsilon_o+b\epsilon}t}\]

Luego,

\begin{equation}
\begin{split}
    D_b &= \frac{V_0\epsilon_o}{b}-
    \frac{a}{b}\frac{V_0\epsilon_o^2}{a\epsilon_o+b\epsilon}
    e^{-\frac{bg}{a\epsilon_o+b\epsilon}t}\\
    \sigma &= \int^t_0J_a(t')\,dt'\\
    &= -\frac{V_0\epsilon_o}{b}\lados{(}{e^{-\frac{bg}{a\epsilon_o+b\epsilon}t}-1}\\
\end{split}
\nonumber
\end{equation}

Definiendo

\[\omega = \frac{-bg}{a\epsilon_o+b\epsilon}\]

se tiene que

\begin{equation}
\begin{split}
    &\Vec{D}_a = -\frac{V_0\epsilon_o\epsilon}{bg}\omega
    e^{\omega t}\hat{z}\\
    &\Vec{D}_b = \lados{(}{1+\frac{a\epsilon_o\omega}{bg}e^{\omega t}}
    \frac{V_0\epsilon_o}{b}\hat{z}\\
    &\Vec{E}_a = -\frac{V_0\epsilon_o}{bg}\omega
    e^{\omega t}\hat{z}\\
    &\Vec{E}_b = \lados{(}{1+\frac{a\epsilon_o\omega}{bg}e^{\omega t}}
    \frac{V_0}{b}\hat{z}\\
    & \Vec{J}_a = -\frac{V_0\epsilon_o}{b}\omega
    e^{\omega t}\hat{z}\\
    & \Vec{J}_b = 0\\
\end{split}
\nonumber
\end{equation}

\ics d
Haciendo uso del campo eléctrico y el desplazamiento eléctrico encontrado en $c)$, se despeja la polarización de la capa.

\begin{eqit}
    \Vec{P}_a &= \Vec{D}_a - \e_0\Vec{E}_a\\
    &= \lados{(}{-\frac{V_0\epsilon_o\epsilon}{bg}\omega
    e^{\omega t} +\e_0\frac{V_0\epsilon_o}{bg}\omega
    e^{\omega t}}\hat{z}\\
    &= \frac{(\e_0 - \e)V_0\e_0}{bg}\omega e^{\omega t}\hat{z}
\end{eqit}

Con esto las densidades de polarización serán:
\begin{itemize}
    \item \textbf{En la superficie inferior} $(z=0)$
    \[ \sigma_{p(z=0)} = \Vec{P}(z=0)\cdot (-\hat{z}) = \frac{(\e - \e_0)V_0\e_0}{bg}\omega e^{\omega t} \]
    
    \item \textbf{En la superficie superior} $(z=a)$
    \[ \sigma_{p(z=a)} = \Vec{P}(z=a)\cdot \hat{z} = \frac{(\e_0 - \e)V_0\e_0}{bg}\omega e^{\omega t} \]
    
    \item \textbf{En la capa}
    \[ \rho_p = -\nabla \cdot \Vec{P} = 0 \]
    Al ser $\Vec{P}$ independiente de $z$
    
\end{itemize}

Con lo que la carga de polarización es la suma de estas integradas en las superficies correspondientes. Notemos que $\sigma _{p(z=0)} = -\sigma _{p(z=a)}$, y que ambas son constantes, así sea $S$ la superficie de las placas, $Q_p = S\sigma _{p(z=0)} + S\sigma _{p(z=a)} = 0$.\\

Para calcular la carga libre en el sistema, se puede hacer uso de $\oint \Vec{D}\cdot dS = Q_{libre}$, donde la superficie es la de las placas. Se puede usar el desplazamiento eléctrico $D_b$, ya que envuelve a todo el sistema por lo que al calcular la carga libre con este, se obtendrá la total.

\begin{eqit}
    Q_{libre} &= \oint \Vec{D}_b\cdot dS\\
    &= \lados{(}{1+\frac{a\epsilon_o\omega}{bg}e^{\omega t}}
    \frac{V_0\epsilon_o}{b}S\\
\end{eqit}

Obteniéndose la carga libre en el sistema en función del tiempo.

\ics e
Para encontrar la variación se puede calcular la energía almacenada en $t=0$ y en $t=\infty$, y restando ambos encontrarla.

\begin{eqit}
    U_{t=0} &= \frac{1}{2}\int\Vec{E}_{t=0}\cdot\Vec{D}_{t=0}\,d\V\\
    &= \frac{1}{2}\int_S\lados{(}{\int^a_0\lados{(}{\frac{V_0\epsilon_o}{a\epsilon_o+b\epsilon}}^2\epsilon\,dz+\int^{a+b}_a\lados{(}{\frac{V_0\epsilon}{a\epsilon_o+b\epsilon}}^2\epsilon_o\,dz}\,dS\\
    &= \frac{S\epsilon_o\epsilon}{2}\lados{(}{\frac{V_0}{a\epsilon_o+b\epsilon}}^2(a\epsilon_o+b\epsilon)\\
    &= \frac{S\epsilon_o\epsilon}{2}\frac{V_0^2}{a\epsilon_o+b\epsilon}\\
    \\
    U_{t=\infty}&= \frac{1}{2}\int\Vec{E}_{t=\infty}\cdot\Vec{D}_{t=\infty}\,d\V\\
    &= \frac{1}{2}\int_S\lados{(}{\int^a_0\Vec{E}_{a,t=\infty}\cdot\Vec{D}_{a,t=\infty}\,dz+\int^{a+b}_a\Vec{E}_{b,t=\infty}\cdot\Vec{D}_{b,t=\infty}\,dz}\,dS\\
    &= \frac{1}{2}\int_S\int^{a+b}_a\frac{V_0^2\epsilon_o}{b^2}
    \,dzdS\\
    &= \frac{SV_0^2\epsilon_o}{2b}\\
    \\
    \Delta U &= U_{t=\infty} - U_{t=0}\\
    &= \frac{SV_0^2\epsilon_o}{2b}-\frac{S\epsilon_o\epsilon}{2}\frac{V_0^2}{a\epsilon_o+b\epsilon}\\
    &= \frac{SV_0^2\epsilon_o}{2}\lados{(}{\frac{1}{b}-\frac{\epsilon}{a\epsilon_o+b\epsilon}}\\
\end{eqit}

Como es pedida la energía almacenada en el sistema, se integra en el volumen correspondiente a este, es decir en el paralelepípedo de cara $S$ y alto $a+b$.

Durante este periodo, la energía disipada viene dada por
\begin{eqit}
    U_{disipada} &= U = \int_{0}^{\infty}\int_V(\Vec{E}\cdot \Vec{J})d\V dt\\
    &= \int_0^\infty \lados{(}{\int_0^a \Vec{E}_a\cdot \Vec{J}_a\,d\V + \int_a^{a+b}\Vec{E}_b\cdot \Vec{J}_b\,d\V}\,dt\\
    &=\int_0^\infty\lados{(}{\int_0^a \frac{V_0^2\e_0^2}{b^2g}\omega^2\e^{2wt} \,d\V + 0}\,dt\\
    &= \int_0^\infty\lados{(}{\frac{V_0^2\e_0^2}{b^2g}\omega^2 e^{2wt}S(a+b)}\,dt\\
    &= \frac{V_0^2\e_0^2}{b^2g}\omega S(a+b)\int_0^\infty \omega e^{2wt}\,dt\\
    &= \frac{V_0^2\e_0^2}{b^2g}\omega S(a+b) \lados{[}{\frac{e^{2wt}}{2}}\Big|_0^\infty\\
    &= -\frac{V_0^2\e_0^2\omega S(a+b)}{2b^2g} > 0 \quad \quad (\omega < 0)
\end{eqit}

La energía procede de la fuente, que aporta lo que se pierde por efecto Joule y algo más para cargar al sistema. (Si no se mantendría siempre en el estado $t=0$)

\end{solucion}

\bigbreak 

\begin{solucion}{5}
\ics a
Por la simetría del sistema se cumple que, en coordenadas esféricas, $\Vec{E}(\Vec{r})= E(r)\hat{r}$. Por condición de borde, se sabe que las componentes del de $\Vec{E}$ tangenciales a la interfase son iguales. En este caso, para $\theta = \frac{\pi}{2}$, $\hat{r}$ es tangente a la interfaz entre el dieléctrico y el vacío, de modo que el campo eléctrico es descrito por la misma función para todo el espacio entre las esferas.\\

Definiendo $\vec{D}_0$ como el desplazamiento en el vacío y $\vec{D}_1$ en el dieléctrico para $a<r<b$, en el instante inicial ($t=0$) se tiene que

\begin{eqit}
    Q_0 &= \oint\Vec{D}(0,r)\cdot d\vec{S}\\
    &= \int^{2\pi}_0\lados{(}{\int^{\frac{\pi}{2}}_0D_0(0,r)\sin{
    \theta}\,d\theta+\int^\pi_{\frac{\pi}{2}}D_1(0,r)\sin{
    \theta}\,d\theta}r^2\,d\phi\\
    &= 2\pi r^2(D_0(0,r)+D_1(0,r))\\
    &= 2\pi r^2(\epsilon_o+\e)E(0,r)\\
\end{eqit}
\begin{eqit}
    D_0(0,r) &= \frac{Q_0\epsilon_o}{2\pi (\e_o+\e)r^2}\\
    \\
    D_1(0,r) &= \frac{Q_0\epsilon}{2\pi (\e_o+\e)r^2}\\
\end{eqit}

%Aplicando las condiciones de borde en la superficie de la esfera interior, denotando con 2 al interior del conductor, se obtienen las siguientes ecuaciones

%\begin{eqit}
%    &D_0(t,a) - D_2(t,a)= \sigma_a\quad \theta\in[0,\frac{\pi}{2}]\\
%    &D_1(t,a) -D_2(t,a)= \sigma_a\quad \theta\in[\frac{\pi}{2},\pi]\\
%    &J_2(t,a)= \parfrac{\sigma_a}{t}\quad \theta\in[0,\frac{\pi}{2}]\\
%    &J_1(t,a)-J_2(t,a) = -\parfrac{\sigma_a}{t}\quad \theta\in[\frac{\pi}{2},\pi]\\
%\end{eqit}

\end{solucion}

\bigbreak

\begin{solucion}{6}
\ics a
A causa de la simetría del sistema, si se supone que $L \gg b$, se tiene que en coordenadas cilíndricas $\Vec{E}(\Vec{r})=E(\rho)\hat{\rho}$. Por lo que $\Vec{J}$ también solo depende de $\rho$. Por condiciones de borde se tiene que los campos eléctricos serán iguales en los tres medios conductores, al ser tangenciales entre sí.\\

Superponiendo un cilindro de largo $L$ y radio arbitrario $\rho$ fijo, con $a < \rho < b$, se puede usar la Ley de Gauss para calcular el campo eléctrico, donde la carga $Q_*$ será calculada explícitamente después.
(Es importante notar que se puede ocupar la Ley de Gauss a causa de que la permitividad en los tres conductores es $\e_0$)

\begin{eqit}
    \oint \Vec{E}\cdot d\Vec{S} &= \frac{Q_*}{\e_0}\\
    \implies E(\rho)2\pi \rho L &= \frac{Q_*}{\e_0}\\
    \implies E(\rho) &= \frac{Q_*}{2\pi \e_0 L}\frac{1}{\rho}\\
    \implies \vec{E}(\rho) &= \frac{Q_*}{2\pi \e_0 L}\frac{1}{\rho}\hat{\rho}
\end{eqit}

Para despejar $Q_*$ integramos para el potencial

\begin{eqit}
    V_0 &= -\int_b^aE(\rho)\,d\rho\\
    &= -\int_b^a \frac{Q_*}{2\pi \e_0 L}\frac{1}{\rho}\,d\rho\\
    &= -\frac{Q_*}{2\pi \e_0 L}\ln\lados{(}{\frac{a}{b}}\\
\end{eqit}

\[\implies Q_* = \frac{V_0\e_02\pi L}{\ln(b/a)}\]

Con lo que queda que 
\[\vec{E}(\rho) = \frac{V_0}{\ln(b/a)}\frac{1}{\rho}\hat{\rho}\]

Con esto se tiene que ($i \in \{1,2,3\}$),

\[\vec{J}_i(\rho) = g_i\frac{V_0}{\ln(b/a)}\frac{1}{\rho}\hat{\rho}\]

Notemos que $\vec{J}_i$ cumple con el estado de régimen estacionario, al ser $\dvr \vec{J}_i = 0$ en coordenadas cilíndricas.\\

Para calcular la resistencia eléctrica usamos $R = V/I$, donde $I$ es la integral de la densidad de corriente eléctrica sobre la superficie y $V = V_0$. Calculando la corriente

\begin{eqit}
    I &= \int\vec{J}\cdot d\vec{S}\\
    &= \int_0^L\int_0^{2\pi/3}J_1\rho d\phi dz + \int_0^L\int_{2\pi/3}^{4\pi/3}J_2\rho d\phi dz + \int_0^L\int_{4\pi/3}^{2\pi}J_3\rho d\phi dz\\
    &= \frac{2\pi L}{3}\frac{V_0}{\ln(b/a)}\lados{(}{g_1 + g_2 + g_3}
\end{eqit}

Así la resistencia es
\[ R = \frac{3\ln(b/a)}{2\pi L(g_1 + g_2 + g_3)} \]

\ics b
Se tiene que al desconectar la fuente a causa del Efecto Joule la energía disminuirá hasta llegar a cero, al igual que la diferencia de potencial. Por lo que la corriente, al depender de la diferencia de potencial, también disminuirá.\\

Haciendo uso de las condiciones de borde para el desplazamiento eléctrico y la densidad de corriente eléctrica en la superficie $\rho = a$, tenemos que $J = \parfrac{\sigma}{t}$ y $D = \sigma$, al ser $J = D = 0$ en el conductor, al no haber campo eléctrico dentro de este. Así se puede despejar $\sigma$, dando $\sigma(t) = \sigma_0e^{-t(g/\e_0)}$. Tomando en cuento que en $t=0$, $\sigma = \sigma_0 = Q_*/(2\pi aL)$, tenemos la densidad superficial en todo momento.\\

Haciendo uso de la simetría del problema, y de la densidad superficial podemos despejar el campo eléctrico entre $a < \rho < b$, con la Ley de Gauss (la permitividad es $\e_0$). Lo que da

\[ \vec E(\rho) = \frac{\sigma(t)}{\e_0}\frac{a}{\rho}\hat{\rho} \]

Así para calcular la corriente eléctrica, se tiene

\begin{eqit}
    I &= \int\vec{J}\cdot d\vec{S}\\
    &= \int_0^L\int_0^{2\pi/3}J_1\rho d\phi dz + \int_0^L\int_{2\pi/3}^{4\pi/3}J_2\rho d\phi dz + \int_0^L\int_{4\pi/3}^{2\pi}J_3\rho d\phi dz\\
    &= \int_0^L\int_0^{2\pi/3}g_1a\frac{\sigma(t)}{\e_0} d\phi dz + \int_0^L\int_{2\pi/3}^{4\pi/3}g_2a\frac{\sigma(t)}{\e_0} d\phi dz + \int_0^L\int_{4\pi/3}^{2\pi}g_3a\frac{\sigma(t)}{\e_0} d\phi dz\\
    &= \frac{2 \pi L}{3\e_0}\frac{Q_*}{2\pi aL}ae^{-t(g/\e_0)}(g_1+g_2+g_3)\\
    &= \frac{Q_*}{3\e_0}e^{-t(g/\e_0)}(g_1+g_2+g_3)
\end{eqit}

Que calza con la corriente obtenida para el estado estacionario ($t=0$), y con que en infinito se hacer cero la corriente ($t=\infty$). 

\ics c
Se tiene que durante el periodo transitorio toda la energía potencial electroestática será liberada en forma de calor a causa del efecto Joule. Por lo que la cantidad de energía disipada será igual a la energía potencial electroestática existente antes de desconectar la fuente.\\

Esta energía procede del movimiento de cargas aun existente, originalmente esta energía fue proveída por la fuente. 

\end{solucion}

\bigbreak

\begin{solucion}{7}
\ics a
Sea $S$ el área de la superficie de las placas.\\

Se tiene que justo al conectar la batería y dotar de la densidad volumétrica $\rho_0$ no existe carga entre el conductor superior y el inferior. Por lo que la condición de borde para el desplazamiento eléctrico nos \cita{dice} que los campos eléctricos en $x = a$ han de ser iguales (la permitividad es $\e_0$ para ambos conductores.)\\

Usando la Ley de Gauss para el campo eléctrico y notando que, debido a que la corriente irá desde $x=0$ a $x=a+b$ el campo eléctrico apunta en $\hat{x}$, y solo depende de $x$. Sea $Q_0$ la carga ubicada en la placa inferior,

\[ \oint E(x)\,dS = \frac{Q_{\text{lib enc}}}{\e_0}\]
\[
    \implies \vec E (x) = 
        \begin{cases}
            \frac{S\rho_0x + Q_0}{S\e_0} & 0 < x < a\\
            \\
            \frac{aS\rho_0 + Q_0}{S\e_0} & a < x < a+b
        \end{cases}
\]

Con lo que se tiene que
\[ \vec J (x) = 
        \begin{cases}
            g_1\frac{S\rho_0x + Q_0}{S\e_0} & 0 < x < a\\
            \\
            g_2\frac{aS\rho_0 + Q_0}{S\e_0} & a < x < a+b
        \end{cases}\]

Para despejar $Q_0$ usamos que la diferencia de potencial entre $x=a+b$ y $x=0$ es $V_0$,

\begin{eqit}
    V_0 &= V(0) - V(a+b)\\
    &= -\int_{a+b}^0\vec E \cdot \,d\vec l\\
    &= -\int_{a+b}^a\frac{aS\rho_0 + Q_0}{S\e_0}dx - \int_a^0\lados{(}{\frac{\rho_0}{\e_0}x + \frac{Q_0}{S\e_0}}\,dx\\
    &= b\frac{aS\rho_0 + Q_0}{S\e_0} + \frac{\rho_0}{\e_0}\frac{a^2}{2} + \frac{Q_0}{S\e_0}a\\
    \implies Q_0 &= \frac{S}{a+b}\lados{[}{V_0\e_0 - \rho_0ab - \frac{\rho_0a^2}{2}}
\end{eqit}

Finalmente las cargas libres serán
\begin{itemize}
    \item Carga libre total volumétrica
    \[Q_v = \rho a S\]
    
    \item Carga libre en la capa inferior
    \[Q_{inf} = Q_0\]
    
    \item Carga libre en la capa superior
    \[Q_{sup} = -Q_{inf} - Q_v\]
    
    Al ser el sistema neutro, causando que la suma de cargas libres sea cero. También puede sacarse haciendo uso de la condición de borde para el desplazamiento eléctrico.
\end{itemize}

\ics b
En estado estacionario $\vec J$ es constante, ya que por ecuación de continuidad ya no hay cambio en la cantidad de carga almacenada. Así
\[\vec J_1 = J_1\hat x \quad \land \quad \vec J_2 = J_2\hat x\]

Además por condición de borde se tiene que $J_1 = J_2$, lo que implica que $g_1E_1 = g_2E_2$, donde $1$ es el conductor inferior y $2$ el superior. Como $J_1 y J_2$ son constantes entonces al ser $g_1$ y $g_2$ también, $E_1$ y $E_2$ también lo son. Como aún se tiene una diferencia de potencial $V_0$ se puede integrar y encontrar otra relación entre ambos campos eléctricos para despejarlos.

\begin{eqit}
    V_0 &= V(0) - V(a+b)\\
    &= -\int_{a+b}^aE_2\,dx - \int_a^0E_1\,dx\\
    &= E_2b + E_1a\\
\end{eqit}

Reemplazando $E_1 = \frac{g_2}{g_1}E_2$ en la ecuación anterior da que
\[\vec E_2 = \frac{V_0g_1}{bg_1 + ag_2}\hat x\]
Y por lo tanto 
\[\vec E_1 = \frac{V_0g_2}{bg_1 + ag_2}\hat x\]

Las cargas libres se pueden obtener haciendo uso de la Ley de Gauss para el desplazamiento eléctrico, las condiciones de borde para el desplazamiento eléctrico y que la suma total de las cargas ha de dar cero.\\

Notemos que haciendo uso de la forma diferencial de la ley de Gauss para el desplazamiento eléctrico ($\dvr \vec D = \rho_l$) da que $\rho_l = 0$, al ser $\vec D = \e_0\vec E = cte\hat x$, así que habrá carga libre solo en las superficies de las placas.\\

Usando la condición de borde para el desplazamiento tenemos que la densidad de carga superficial en $x=a$ es
\begin{eqit}
    \sigma_{x=a} &= D_2 - D_1 = \frac{\e_0V_0(g_1-g_2)}{bg_1+ag_2}
\end{eqit}

De la misma manera sabiendo que los campos eléctricos fueras de las placas son cero, entonces da que 

\begin{eqit}
    \sigma_{x=0} &= \frac{\e_0V_0g_2}{bg_1 + ag_2}\\
    \sigma_{x=a+b} &= -\frac{\e_0V_0g_1}{bg_1 + ag_2}
\end{eqit}

Donde las cargas libres totales son $S\sigma_{x=0}$, $S\sigma_{x=a}$ y $S\sigma_{x=a+b}$.

\ics c
En el régimen estacionario no hay variación de energía electroestática almacenada en el sistema, a causa de que esta está almacenada en los campos eléctricos y como estos no varían en el régimen estacionario (son independietnes del tiempo), entonces la energía se mantiene constante.

\ics d
Si existe energía disipada a lo largo del tiempo a causa del efecto Joule, esta energía se va a los conductores que la liberan en forma de calor. Y proviene de la fuente, que otorga la misma cantidad de energía disipada para mantener la energía electroestática del sistema constante.

\end{solucion}

\newpage