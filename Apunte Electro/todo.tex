Sección campo eléctrico
- Terminar la matraca de la S.4.3
- Terminar la S.4.4


Sección Conductores
- S.7.6 Tiene soluciones textuales sin valores y copiadas del aux correspondiente. 


Sección Dipolo
- Falta la c de la S.8.10 | Dejé una posible forma de hacerlo con desarrollo como comentario, desconozco si está correcto por eso no lo puse. Apenas me convenza/encuentre una respuesta correcta lo hago!

% - Falta el trabajo en la d de la S.8.12, puse que el trabajo realizado por la fuente es el cambio de energía potencial. Ya que debe mantener el potencial al mismo valor aún después del cambio.

Sección Corriente
- S.9.2 falta la parte b)
- S.9.5 


General
- ¿Revisar problemas?
- Poner imágenes, si se puede, donde ayude a visualizar algo, como las de simetría
- hacer un backup 


Backups en GitHub, el último fue el /12/20 a la



Futuros (posibles) proyectos:

- Ya que haremos varios podríamos podríamos ponerle un nombre oficial a los "apuntes" y el autor | Buena idea
    - Webeando con nombres se me ocurrió: "Le Touffe", significa el mechón en francés xd
    - Otras opciones: * Apuntes Generales
                      * Apuntes OWO
                      * owo apuntes (o resúmenes) owo


Links:

Página web:
    - https://sircrocker.github.io/Le-Touffe/
    - bit.ly/apuntes-fcfm

Econo: https://www.overleaf.com/1631722783zqhnhstfqmqw
Termo: https://www.overleaf.com/4624624988vkyvkznsrrpg
Cálculo: https://es.overleaf.com/8132144621qhjqsdppxpyg







Actualizaciones:
- Nuevo ambiente \begin{eqit} \end{eqit}
- Cambie el nombre de algunas carpetas para que el sistema de errores Overleaf funcionara mejor
- Nuevo comando: \indeciso{}{} sirve para poner texto del que estas indeciso o hacer que al compilar sea ignorado algo.

% Este codigo hace un buho: \widehat{\begin{pmatrix} \odot_{\text{v}}\odot \\ \wr \end{pmatrix}}
