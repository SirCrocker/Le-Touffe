\subsection{Soluciones}
\medbreak
\sol{1}\newline

$a)$ La densidad de carga de ambas varillas es $\lambda = Q/L$. Por principio de superposición, el campo eléctrico de la segunda varilla a lo largo del eje x es

\[\Vec{E}(\Vec{x}) = \int^{-a/2}_{-a/2-L}\frac{\lambda}{4\pi\epsilon_o}\frac{\Vec{x}-\Vec{r}}{\parallel\Vec{x}-\Vec{r}\parallel^3}\,dr\]

\[\,\,\,\,\,\,\,\,\,\,\,\,\,\,= \frac{Q}{4\pi\epsilon_o L} \int^{-a/2}_{-a/2-L}\frac{(x-r)\hat{x}}{(x-r)^3}\,dr\]

\[\,\,\,\,\,\,\,\,\,\,\,\,\,\,\,\,\,\,= \frac{Q}{4\pi\epsilon_o L} \int^{-a/2}_{-a/2-L}(x-r)^{-2}\hat{x}\,dr\]

\[\,\,\,\,\,\,\,\,\,\,\,\,\,\,\,\,\,\,\,\,\,\,\,\,\,\,\,\,\,\,\,=\frac{Q}{4\pi\epsilon_o L}\left( \frac{2}{a+2x} - \frac{2}{a+2x+2L} \right)\hat{x}\]

\[\,\,\,\,\,\,\,\,\,\,\,\,\,\,\,\,\,\,=\frac{Q}{2\pi\epsilon_o(a+2x)(a+2x+2L)}\hat{x}\]

\bigbreak

$b)$ Por la parte anterior se sabe que el campo eléctrico de la segunda varilla sobre los puntos del eje $x$ es de forma $\Vec{E}(\Vec{x}) = E(x)\hat{x}$. Usando ley de Coulomb y el principio de superposición, la fuerza eléctrica que ejerce la segunda varilla sobre la primera es

\[\Vec{F}=\sum_{x\in\Gamma}q_xE(x)\hat{x}\]

Donde $\Gamma = [a/2, a/2+L]$ y $q_x$ es la carga de cada párticula de la varilla. De forma que la magnitud de la fuerza que ejerce una varilla sobre la otra es

\[F=\sum_{x\in\Gamma}q_xE(x)=\int^{a/2+L}_{a/2}\lambda E\,dx\]

\[\,\,\,\,\,\,\,\,\,\,\,\,\,\,\,\,\,\,\,\,\,\,\,\,\,\,\,\,\,\,\,\,\,\,\,\,\,\,=\frac{Q^2}{2\pi\epsilon_o L}\int^{a/2+L}_{a/2}\frac{1}{(a+2x)(a+2x+2L)}dx\]

\[\,\,\,\,\,\,\,\,\,\,\,\,\,\,\,\,\,\,\,\,\,\,\,\,\,\,\,\,\,\,\,\,\,\,\,\,\,\,\,\,\,\,\,\,\,\,\,\,\,\,\,\,\,\,\,\,\,\,\,\,\,\,\,\,\,\,\, = \frac{Q^2}{4\pi\epsilon_o L^2}\left( ln(a+L)-ln(a)-ln(a+2L)+ln(a+L) \right)\]

\[=\frac{Q^2}{4\pi\epsilon_o L^2}ln\left( \frac{(a+L)^2}{a(a+2L)} \right)\,\,\,\]

\bigbreak
\sol{2}\newline

$a)$ Tomando un sistema de coordenadas cilíndricas con origen en el centro del anillo tal que el eje $z$ coincide con el del anillo, el campo eléctrico en $z$ es

\[\Vec{E}(\Vec{z}) = \frac{\lambda}{4\pi\epsilon_o}\int\frac{\Vec{z}-\Vec{r}}{\parallel
\Vec{z}-\Vec{r}\parallel^3}dr\]

\[\,\,\,\,\,\,\,\,\,\,\,\,\,\,\,\,\,\,\,\,\,\,\,\,\,\,\,\,\,\,\,\,\,\,\,\,\,\,\,\,\,\,\,\,\,\,\,\,\,\,\,\,\,\,\,\,\,\,\,\,\,\,\,\,\,\,\,\,\,\,\,\,\,\,\,\,\,\,\,\,\,= \frac{\lambda}{4\pi\epsilon_o}\int^{2\pi}_0\frac{z\hat{z}-R\hat{\rho}}{(z^2+R^2cos^2(\phi)+R^2sin^2(\phi))^{3/2}}R\,d\phi\]

\[\,\,\,\,\,\,\,\,\,\,\,\,\,\,\,\,\,\,\,\,\,\,\,\,\,\,\,\,\,\,= \frac{R\lambda}{4\pi\epsilon_o}\int^{2\pi}_0\frac{z\hat{z}-R\hat{\rho}}{(z^2+R^2)^{3/2}}\,d\phi\]

\[\,\,\,\,\,\,\,\,\,\,\,\,\,\,\,\,\,\,\,\,\,\,\,\,\,\,\,\,\,\,\,\,\,\,\,\,\,\,\,\,\,\,\,\,\,\,\,\,\,\,\,\,\,\,\,\,\,\,\,\,\,\,\,\,\,\,\,\,\,\,\,\,\,\,\,\,\,\,\,\,\,= \frac{R\lambda}{4\pi\epsilon_o}\left(
\int^{2\pi}_0\frac{z\hat{z}}{(z^2+R^2)^{3/2}}\,d\phi-
\int^{2\pi}_0\frac{R(cos(\phi)\hat{x}+sin(\phi)\hat{y})}{(z^2+R^2)^{3/2}}\,d\phi
\right)\]

\[\,\,\,\,\,\,\,\,\,\,\,\,\,\,\,\,\,\,\,\,\,\,\,\,\,\,\,\,\,= \frac{R\lambda}{4\pi\epsilon_o}\left(
\frac{2\pi z\hat{z}}{(z^2+R^2)^{3/2}}-0
\right)\]

\[\,\,\,\,\,\,= \frac{R\lambda z\hat{z}}{2\epsilon_o(z^2+R^2)^{3/2}}\]

\bigbreak

$b)$ Esta parte sigue la misma lógica que la anterior, la integral ahora es de área, con un radio que varía de $R_1$ a $R_2$

\[\Vec{E}(\Vec{z})= \frac{\sigma}{4\pi\epsilon_o}\int^{R_2}_{R_1}\int^{2\pi}_0\frac{z\hat{z}-\rho\hat{\rho}}{(z^2+\rho^2)^{3/2}}\rho\,d\phi d\rho\]

\[= \frac{\sigma}{2\epsilon_o}\int^{R_2}_{R_1}\frac{\rho z\hat{z}}{(z^2+R^2)^{3/2}}\,d\rho\,\,\,\,\,\,\,\,\,\, \]

\[\,\,\,\,\,\,\,\,\,\,\,\,= \frac{\sigma z}{2\epsilon_o}\left( 
\frac{1}{\sqrt{z^2+R_1^2}}-\frac{1}{\sqrt{z^2+R_2^2}}
\right)\hat{z}\]

\bigbreak

$c)$ Si $R_1 \longrightarrow 0$ entonces

\[\Vec{E}(\Vec{z}) = \frac{\sigma z}{2\epsilon_o}\left( 
\frac{1}{z}-\frac{1}{\sqrt{z^2+R_2^2}}
\right)\hat{z}\]

Esto equivale al campo eléctrico del un disco de radio $R_2$ sobre el eje $z$. Si además $R_2 \longrightarrow \infty$ entonces

\[\Vec{E}(\Vec{z}) = \frac{\sigma z}{2\epsilon_o}\left( 
\frac{1}{z}-0
\right) = \frac{\sigma}{2\epsilon_o}\hat{z}\]

Esto equivale al campo eléctrico de un plano infinito

\medbreak

$d)$ Si se tiene que $z^2 << R_1^2$ y $z^2 << R_2^2$, $z$ se puede despreciar de las raíces

\[\Vec{E}(\Vec{z}) \approx \frac{\sigma z}{2\epsilon_o}\left( 
\frac{1}{R_1}-\frac{1}{R_2}
\right)\hat{z}\]

\bigbreak
\sol{3}\newline

Intentar calcular el campo eléctrico integrando directamente resulta en una integral algo fea, por lo que calcularemos primero el potencial. De define el sistema de coordenadas cilíndricas tal que el segmento coincide con el eje $z$ y uno de sus extremos toca el origen

\[V(\Vec{r}) = \frac{\lambda}{4\pi\epsilon_o}\int^L_0\frac{1}{\parallel\Vec{r} - \Vec{z'}\parallel}\,dz'\]

\[ = \frac{\lambda}{4\pi\epsilon_o}\int^L_0\frac{1}{\sqrt{\rho^2 + (z-z')^2}}\,dz'\]

\[=\frac{\lambda}{4\pi\epsilon_o}\left(
asinh\left(\frac{z}{\rho}\right)-
asinh\left(\frac{z-L}{\rho}\right)
\right)\]

\bigbreak
\sol{4}\newline\newline
$a)$ Por principio de superposición el campo eléctrico en todo el espacio es equivalente a la suma de un campo generado por un plano más el de un cilindro con carga opuesta.
\medbreak
\textbf{Plano:}
\medbreak


\newpage