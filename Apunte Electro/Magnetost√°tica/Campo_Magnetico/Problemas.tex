\subsection{Problemas}
--------------------------------------------------------
\\

% AAAAAH, una nueva guía y tarea :/
% Bimage es que en la guía hay una imagen

\np
Una partícula de carga eléctrica $q$ y masa $m$ está inicialmente en reposo en presencia de un campo eléctrico uniforme $\vec{E}$, y un campo magnético uniforme $\vec{B}$ perpendicular a $\vec{E}$.

\begin{enumerate}[label=\alph*)]
	\item Describa el movimiento posterior de la partícula.

	\item Use el resultado anterior para discutir el siguiente problema. Considere un condensador de placas paralelas de superficie $S$, y separación entre las placas $h$, con una diferencia de potencial $V$ entre ellas. Se aplica un campo magnético uniforme $\vec{B}$ en la dirección perpendicular al campo eléctrico, es decir, paralelo a las placas. Una radiación ultravioleta hace que la placa negativa emita electrones, de carga $-e$, con velocidad inicial nula. Determine el valor mínimo de $\vec{B}$ para el cual los electrones no alcanzan la placa positiva.

	\bimage{Magnetostática/Campo_Magnetico/img/P_10_1.png}

\end{enumerate}

\bigbreak

\np
Sea un cascarón esférico de radio $R$ con densidad de carga superficial $\rho$ rotando con una rapidez angular $\omega$.

\begin{enumerate}[label=\alph*)]
	\item Calcule la corriente superficial $\vec{K}$ en la superficie de la esfera debido a su movimiento.

	\item Calcule el potencial vector magnético creado por la esfera en todos los puntos del espacio.

	\item Calcule el campo magnético en todos los puntos del espacio.
\end{enumerate}

\bigbreak

\np
\begin{enumerate}[label=\alph*)]
	\item Determine el campo magnético creado por una espira circular de radio $a$, recorrida por una intensidad de corriente $I$, en los puntos del eje de revolución de la espira (eje $z$).

	\item Utilizando el resultado obtenido en $(a)$, calcule el campo magnético creado por un solenoide cilíndrico de radio $a$ y longitud $h$ en los puntos del eje de revolución de solenoide. El solenoide está construido con un bobinado denso y uniforme de un hilo conductor. El hilo da $N$ vueltas alrededor del cilindro y por él circula una corriente de intensidad $I$.

	\item Calcule el valor que toma el campo magnético obtenido en $(b)$ cuando la longitud del solenoide se hace tender a infinito $(h \to \infty)$, manteniendo constante el cociente $n = N/h$ que corresponde al número de espiras por unidad de longitud.

	\item Obtenga nuevamente el resultado de la parte $(c)$ calculando el campo magnético mediante la Ley de Ampere. Explique detalladamente las simetrías consideradas y el camino escogido para determinar el campo magnetico $\vec{B}$.	

	\item Determine el potencial vector magnético creado por el solenoide cilíndrico en todos los puntos del espacio. Determine el campo magnético a partir de $\vec{A}$, y compare su resultado con el obtenido en $(c)$.

\end{enumerate}

\bimage[0.3\textwidth]{Magnetostática/Campo_Magnetico/img/P_10_3.png}

\pagebreak

\np
Se tienen dos cilindros infinitamente largos y paralelos, cada uno de radio $R$. Por los cilindros circulan densidades de corriente uniformes $+J \hat{z}$ y $-J \hat{z}$, respectivamente. Los ejes de los cilindros distan una distancia $a < 2R$.

\begin{enumerate}[label=\alph*)]
	\item Demuestre que el campo magnético en la zona de intersección es constante. Determine el valor.

	\item Calcule el campo magnético para puntos alejados de los dos cilindros.
\end{enumerate}

\bimage{Magnetostática/Campo_Magnetico/img/P_10_4}

\bigbreak

\np 
Considere un bloque conductor que está limitado entre los planos $x = -a$ y $x = a$, y se extiende hasta infinito en las direcciones de los ejes $\hat{y}$ y $\hat{z}$. Por el bloque conductor circula una corriente estacionaria de densidad volumétrica de corriente no uniforme $J = J_0\frac{\abs{x}}{a}\hat{z}$.

\begin{enumerate}[label=\alph*)]
	\item Calcule el campo magnético creado por el bloque en todos los puntos del espacio.

	\item Grafique el campo magnético en función de $x$.

	\item Calcule la fuerza que ejerce el bloque conductor sobre una espira cuadrada de lado $L$ $(L < a)$ por la cual circula una corriente $I$. Considere que la espira se ubica en el plano $xz$ y que su centro coincide con el origen de coordenadas.
\end{enumerate}

\newpage

\np
Considere un conductor cilíndrico infinito de radio interior $a$ y radio exterior $b$. El conductor lleva una densidad de corriente no uniforme dada por:

\[\vec{J} = \frac{\alpha}{r}\hat{\theta}+\beta\hat{z}\]

Donde $\alpha$ y $\beta$ son constantes y $r$ es la distancia de un punto interior del conductor al eje de éste. Determine el campo magnético en todo el espacio.

\bimage{Magnetostática/Campo_Magnetico/img/P_10_6}

% Guia 10, fts

\np
Determine el momento dipolar magnético correspondiente a una distribución de carga uniforme con carga total $Q$, que gira en torno a un eje con velocidad angular $\omega$, de:

\begin{enumerate}[label=\alph*)]
	\item Un anillo de radio R.
	\item Un disco de radio R.
	\item Una superficie esférica de radio R.
	\item Una esfera maciza de radio R.
\end{enumerate}

\bigbreak

\np
En el plano $z = 0$ se encuentran dos anillos coplanares concéntricos, de radios $a$ y $b$ ($b > a$). 


Por el anillo interior circula una corriente $I_0$ .

\begin{enumerate}
	\item Encuentre la corriente $I_1$ que debe circular por el anillo exterior para que el campo magnético en el centro de los anillos se anule.
	\item Calcule el campo magnético en todos los puntos del eje del sistema.
	\item Encuentre el campo en todos los puntos alejados de los anillos.
	\item Suponga que $b = 2a$ y que nos situamos a una altura $z = 10a$ ¿Cual es la diferencia entre el campo aproximado (aproximación dipolar) y el campo exacto?
\end{enumerate}

\newpage

\np
Dos espiras circulares de distinto tamaño, una de radio $a$ y otra de radio $b$ (con $a < b$), tienen sus centros en el mismo punto del espacio. Sus ejes forman un ángulo $\theta_0$ y por cada una circula una intensidad de corriente dada: $I_1$ para la de radio $a$ e $I_2$ para la de radio $b$. Obtenga el módulo del torque sobre la espira pequeña si el radio de ésta es mucho menor que la de la espira grande $a \ll b$.

\bigbreak

\np
Considere un bloque conductor que está limitado entre los planos $x = -a$ y $x = a$, y se extiende hasta infinito en las direcciones de los ejes $y$ y $z$. Por el bloque conductor circula una corriente estacionaria de densidad volumétrica de corriente $J = J_0\hat z$.

\begin{enumerate}
	\item Calcule el campo magnético creado por el bloque en todos los puntos del espacio.
	
	
	\textbf{Indicación:} Divida el bloque en láminas de espesor infinitesimal $dx'$ por las que circula una densidad superficial de corriente infinitesimal $d\vec K = J_0 dx' \hat z$, y aplique el principio de superposición.
	\item Calcule la fuerza que ejerce el bloque conductor sobre el dipolo de momento dipolar $\vec m = m_0 \hat y$ situado en el origen de coordenadas.
\end{enumerate}

\np
Un loop cuadrado de lado $2a$ se encuentra en el plano $xy$ con su centro en el origen y sus lados paralelos a los ejes $x$ e $y$. Por el loop circula en sentido anti-horario una corriente $I$.

\begin{enumerate}
	\item Encuentre el campo magnético en el eje z.\\

\textbf{Sol:} $B_z(z) = 2 \mu_0 Ia^2 /[\pi(a^2 + z^2 (2a^2 + z^2 )^{1/2} )]$
	\item Muestre que para $z/a \gg 1$ el campo es de un dipolo magnético, encuentre su momento magnético.
	\item Compare el campo en el centro de este loop con el campo magnético al centro de un loop circular de radio $2a$.
\end{enumerate}



\newpage