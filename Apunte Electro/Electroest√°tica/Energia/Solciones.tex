\subsection{Soluciones}

\sol{1}\newline\newline
$a)$ En un cascarón esférico se tiene simetría $\Vec{E}(\Vec{r}) = E(r)\hat{r}$, por lo tanto, usando ley de Gauss el campo eléctrico del cascarón es 0 si $r<R$, dado que al interior la carga es nula, y para $r>R$ está dada por

\[\Vec{E}(\Vec{r})=\frac{Q}{\epsilon_o}\frac{1}{4\pi r^2}\hat{r}\]

Utilizando la expresión para la energía almacenada en el campo eléctrico se tiene que

\[U_e = \frac{\epsilon_o}{2}\int \parallel\Vec{E}\parallel^2\, dV
= \frac{\epsilon_o}{2}\int \frac{Q^2}{16\epsilon_o^2\pi^2 r^4} \, d\V
\]
\[\,\,\,
= \frac{Q^2}{32\pi^2\epsilon_o}\int^{2\pi}_0\int^{\infty}_R\int^\pi_0 \frac{sin(\theta)}{r^2} \, d\theta drd\phi
\]
\[
= \frac{Q^2}{32\pi^2\epsilon_o}\int^{2\pi}_0\int^{\infty}_R \frac{2}{r^2} \, drd\phi
\,\,\,\,\,\,\,\,\,\,\,\,\,\,\,\,\,\,\,\,\,\,\,\,\]
\[
= \frac{Q^2}{32\pi^2\epsilon_o}\int^{2\pi}_0\frac{2}{R}\, d\phi
\,\,\,\,\,\,\,\,\,\,\,\,\,\,\,\,\,\,\,\,\,\,\,\,\,\,\,\,\,\,\,\,\,\,\,\,\,\,\,\,\,\,\,\]
\[
= \frac{Q^2}{32\pi^2\epsilon_o}\frac{4\pi}{R}
\,\,\,\,\,\,\,\,\,\,\,\,\,\,\,\,\,\,\,\,\,\,\,\,\,\,\,\,\,\,\,\,\,\,\,\,\,\,\,\,\,\,\,\,\,\,\,\,\,\,\,\,\,\,\,\,\,\,\,\,\,\]
\[
= \frac{Q^2}{8\pi\epsilon_oR}\,\,\,
\,\,\,\,\,\,\,\,\,\,\,\,\,\,\,\,\,\,\,\,\,\,\,\,\,\,\,\,\,\,\,\,\,\,\,\,\,\,\,\,\,\,\,\,\,\,\,\,\,\,\,\,\,\,\,\,\,\,\,\,\,\,\,\,\,\,\]
\medbreak
*Para $r<R$ la integral es nula
\bigbreak

$b)$ Se pide encontrar la energía por medio del potencial.
\medbreak
El potencial del cascarón está dado por

\[V(R) = -\int^R_\infty\Vec{E}\cdot d\Vec{r}
= -\int^R_\infty\frac{Q}{4\pi\epsilon_o r^2}\,dr
= \frac{Q}{4\pi\epsilon_o R}\]
\medbreak
Con esto, la energía potencial electrostática es

\[U_e = \frac{1}{2}\int\frac{Q}{4\pi R^2}V(R)\,dS
= \frac{Q^2}{32\pi^2\epsilon_o R}\int^{2\pi}_0\int^\pi_0sin(\theta)\,d\theta d\phi
= \frac{Q^2}{8\pi\epsilon_oR}\]


\newpage