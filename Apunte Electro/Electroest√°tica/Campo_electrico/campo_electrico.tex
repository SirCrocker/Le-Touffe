\section{Campo Eléctrico}

Dada una carga puntual $q$ ubicada en $\Vec{r_q}$, se define en campo eléctrico que esta ejerce sobre un punto $\Vec{r}$ como

\[\Vec{E}(\Vec{r}) = \frac{q}{4\pi\epsilon_o}\frac{\Vec{r}-\Vec{r_q}}{\parallel\Vec{r}-\Vec{r_q}\parallel^3}\]

\subsection{Ecuaciones de Maxwell}

Las dos primeras ecuaciones de Maxwell en electrostática son

\begin{equation}
\begin{split}
    &\nabla\cdot\Vec{E}=\frac{\rho}{\epsilon_o}\\
    &\nabla\times\Vec{E}=0
\end{split}
\nonumber
\end{equation}
\bigbreak
La forma integral de la segunda ecuación es
\[\oint_{\Gamma}\Vec{E}\cdot d\Vec{l} = 0\]
\bigbreak
La forma integral de la primera ecuación es la ley de Gauss.

\subsection{Principio de Superposición}

El campo eléctrico es aditivo, es decir, el campo que ejercen $n$ cargas sobre un mismo punto está dado por

\[\Vec{E}(\Vec{r}) = \sum^n_{i=1}\Vec{E_i} = \sum^n_{i=1}\frac{q_i}{4\pi\epsilon_o}\frac{\Vec{r}-\Vec{r_i}}{\parallel\Vec{r}-\Vec{r_i}\parallel^3}\]
\bigbreak
En el caso de una distribución continua se tiene

\[\Vec{E}(\Vec{r}) = \frac{q}{4\pi\epsilon_o}\int\frac{\Vec{r}-\Vec{r_q}}{\parallel\Vec{r}-\Vec{r_q}\parallel^3}\,dq(\Vec{r_q})\]
\bigbreak
donde $dq(\Vec{r_q})$ puede ser $\lambda dr$, $\sigma dS$ o $\rho dV$, con $\lambda$ la densidad de carga lineal, $\sigma$ la densidad superficial y $\rho$ la densidad volumétrica.

\subsection{Ley de Coulomb}

Una carga $q$ que siente un campo eléctrico $\Vec{E}$ se ve sujeta a una fuerza eléctrica, que está dada por

\[\Vec{F} = q\Vec{E}\]

Por acción y reacción, la fuerza que ejerce $q$ sobre la fuente del campo es $\Vec{F} = -q\Vec{E}$.

\subsection{Ley de Gauss}

Se dice que una superficie $\Lambda$ es cerrada si esta encierra un volumen (el cual puede ser infinito). Si la carga total en el espacio encerrado por $\Lambda$ es $Q$, la ley de Gauss garantiza que se satisface la igualdad

\[\oint_{\Lambda} \Vec{E}\cdot d\Vec{S} = \frac{Q}{\epsilon_o}\]

Donde $\Vec{E}$ es el campo eléctrico generado por $Q$, en esta integral en específico se está evaluando en la superficie. Esta es la forma integral de la primera ecuación de Maxwell. El círculo en la integral sólo indica que $\Lambda$ es cerrada, de por sí no altera el cálculo.\\

La ley de Gauss permite determinar el valor del campo eléctrico si este presenta simetría, es decir, cumple que $\Vec{E}(\Vec{r}) = E(n)\hat{n}$, con $\Vec{n} = n\hat{n}$ un vector normal a la superficie. En este caso se verifica que

\[\frac{Q}{\epsilon_o} = \oint_{\Lambda} \Vec{E}\cdot d\Vec{S} = \int_{\Lambda} E(n)\hat{n}\cdot dS\hat{n} = \int_{\Lambda} E(n) \,dS\]

Luego, tomando $\Lambda$ tal que $n$ sea constante, se tiene

\[\int_{\Lambda} E(n)\,dS = E(n)\int_{\Lambda}\,dS\]

De forma que, si el área de $\Lambda$ es $A$, el campo eléctrico en $\Lambda$ es

\[\Vec{E}(\Vec{r}) = \frac{Q}{A\epsilon_o}\hat{n}\]

Para determinar si existe simetría basta ver que, dado un punto $p$ arbitrario en el espacio, la suma de todos vectores que conectan los puntos de la superficie a $p$ resulta perpendicular a la misma.
\bigbreak
La simetría se cumple en esferas, con $\hat{n}=\hat{r}$; cilindros de largo infinito, con $\hat{n}=\hat{\rho}$; y planos infinitos, con $\hat{n}=\hat{z}$. En los 3 casos se asume que el sistema de coordenadas es elegido acorde a la superficie.



\newpage