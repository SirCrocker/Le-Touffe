\chapter{Ecuaciones Diferenciales Ordinarias}

Sea $y: A\subseteq\mathbb{R}\to\mathbb{R}$ una función de clase $\mathcal{C}^n$, se define una EDO como la identidad

\[F(x, y(x),y'(x),...,y^{(n)}(x))=0\]

donde $x\in A$ es una variable arbitraria e $y$ es la incógnita.

\begin{itemize}
    \item \textbf{Orden:} El orden de una EDO es el nivel máximo de derivación de $y$.
    \item\textbf{Grado:} El grado de una EDO es el exponente del término que determina el orden.
\end{itemize}

\section{EDO Lineal}

Se denomina EDO lineal de orden $n$ a aquellas de forma

\[\sum^n_{k=0}a_k(x)y^{(n)}=Q(x)\]

A las funciones $\{a_k\}^n_{k=0}$ se les llama coeficientes y a $Q$ lado derecho. Si el lado derecho es nulo se dice que la EDO es homogénea.

\subsection{Principio de Superposición}

Sea $L(y)$ un \encuote{polinomio diferencial}

\[L(y)=\sum^n_{k=0}a_ky^{(k)}\]

el principio de superposición indica que si para $y_1,y_2\in \mathcal{C}^n$

\[L(y_1)=P_1(x)\]
\[L(y_2)=P_2(x)\]

entonces se verifica que

\[L(y_1+y_2) = L(y_1) + L(y_2) = P_1(x)+P_2(x)\]

$L$ es una transformación lineal. Con esto, el kernel de $L$ 

\[\mathcal{H} = \mathrm{ker}(L) = \{y_h\in\mathcal{C}^n\,|\,L(y_h)=0\}\]

define el subespacio vectorial de las soluciones de la EDO homogénea

\[\sum^n_{k=0}a_ky^{(k)}=0\]

\subsection{EDO no Lineal}

\textbf{\textit{Una EDO no lineal es una EDO que no es lineal}}


\newpage