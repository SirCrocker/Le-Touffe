\section{Demostraciones}

\subsection{Gradiente en Coordenadas Ortogonales}

\label{Dem:Gradiente}

Dado que $\{\hat{u},\hat{v},\hat{w}\}$ es base ortonormal de $\R^3$, se cumple que

\[\nabla f = \nabla f\cdot\hat{u}\,\hat{u}+
\nabla f\cdot\hat{v}\,\hat{v}+\nabla f\cdot\hat{w}\,\hat{w}\]

Con $f$ un campo escalar arbitrario. Por definición de los factores escalares, se tiene que

\[\parfrac{\Vec{r}}{u}=h_u\hat{u}\]

Con lo cual

\[\nabla f\cdot\hat{u}=\frac{1}{h_u}
\lados{(}{\nabla f\cdot\parfrac{\Vec{r}}{u}}=\frac{1}{h_u}
\frac{f}{u}\]
\\
Aplicando esto a $\hat{v}$ y $\hat{w}$, se concluye que

\[\nabla f = \frac{1}{h_u}\parfrac{f}{u}\hat{u}+
\frac{1}{h_v}\parfrac{f}{v}\hat{v}+\frac{1}{h_w}\parfrac{f}{w}\hat{w}\]


\subsection{Rotor en Coordenadas Ortogonales}
\label{Dem:Rotor}
Por linealidad del rotor, para $F$ un campo vectorial arbitrario, se cumple que

\[\nabla\times F=\nabla\times(F_u\hat{u})+
\nabla\times(F_v\hat{v})+\nabla\times(F_w\hat{w})\]


Se verifica además que 

\[\nabla u = \frac{1}{h_u}\hat{u}\]
\\

lo que implica $\hat{u}=h_u\nabla u$, con lo cual

\begin{equation}
\begin{split}
    \nabla\times(F_u\hat{u})&=\nabla\times(F_uh_u\nabla u)\\
    &=\nabla(F_uh_u)\times\nabla u+F_uh_u\nabla\times
    \lados{(}{\nabla u}\\
    &=\nabla(F_uh_u)\times\nabla u\\
    &=\lados{(}{\frac{1}{h_u}\frac{\partial}{\partial u}
    (F_uh_u)\hat{u}+\frac{1}{h_v}\frac{\partial}{\partial v}(F_uh_u)\hat{v}+\frac{1}{h_w}\frac{\partial}{\partial w}(F_uh_u)\hat{w}}\times\frac{1}{h_u}\hat{u}\\
    &=\frac{1}{h_uh_v}\frac{\partial}{\partial v}(F_uh_u)(\hat{v}\times\hat{u})+\frac{1}{h_uh_w}\frac{\partial}{\partial w}(F_uh_u)(\hat{w}\times\hat{u})\\
    &=\frac{-1}{h_uh_v}\frac{\partial}{\partial v}(F_uh_u)\hat{w}+\frac{1}{h_uh_w}\frac{\partial}{\partial w}(F_uh_u)\hat{v}\\
\end{split}
\nonumber
\end{equation}
\\
Aplicando el mismo proceso a $\hat{w}$ y $\hat{v}$, se obtiene el rotor en coordenadas ortogonales.

\newpage