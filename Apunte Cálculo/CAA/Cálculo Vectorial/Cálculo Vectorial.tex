\chapter{Cálculo Avanzado y Aplicaciones}

\section{Cálculo Vectorial}

\subsection{Campo Escalar}

Sea $\Omega\subseteq\mathbb{R}^3$ un abierto no vacío, se llama campo escalar sobre $\Omega$ a toda función $f:\Omega\to\mathbb{R}$.

\subsection{Campo Vectorial}

Sea $\Omega\subseteq\mathbb{R}^3$ un abierto no vacío, se llama
campo vectorial sobre $\Omega$ a toda función $f:\Omega\to\mathbb{R}^3$.

\subsection{Divergencia}

Sean $\Omega\subseteq\mathbb{R}^n$ abierto y $f:\Omega\to\mathbb{R}^n$ diferenciable, se define la divergencia de $f$ como

\[\nabla\cdot f=\sum^n_{i=1} \frac{\partial f_i}{\partial x_i}\]

El operador divergencia es una transformación lineal.

\subsection{Rotor}

Sea $f\in\mathcal{C}^1(\mathbb{R}^3,\mathbb{R}^3)$, se define el rotor de $f$ como

\[\nabla\times f = \left(\frac{\partial f_z}{\partial y}-
\frac{\partial f_y}{\partial z}\right)\hat{x}+
\left(\frac{\partial f_x}{\partial z}-
\frac{\partial f_z}{\partial x}\right)\hat{y}+
\left(\frac{\partial f_y}{\partial x}-
\frac{\partial f_x}{\partial y}\right)\hat{z}\]

\subsection{Laplaciano}

Sea $f\in\mathcal{C}^2(\mathbb{R}^n, \mathbb{R})$, se define el laplaciano de $f$ como

\[\nabla^2 f = \nabla\cdot(\nabla f) =
\sum^n_{i=1}\frac{\partial^2f}{\partial x_i^2}\]
\bigbreak
\bigbreak

Para $F\in\mathcal{C}^2(\mathbb{R}^n, \mathbb{R})$ se define como

\[\nabla^2 F = \sum^n_{i=0}\nabla^2F_i\hat{x_i}\]
\bigbreak
\bigbreak

El operador laplaciano es una transformación lineal, también se suele notar como $\Delta$.\\

Si un campo $\varphi$ verifica $\nabla^2\varphi=0$, se dice que $\varphi$ es armónico.

\subsection{Sistemas Ortogonales}

Un sistema de coordenadas en $\mathbb{R}$ está dado por tres variables que describen a un vector

\[\Vec{r} = \Vec{r}(u,v,w)\]

se definen los factores escalares del sistema como

\[h_u=\left\|\frac{\partial\Vec{r}}{\partial u}\right
\|\,\,\,\,\,\,\,\,
h_v=\left\|\frac{\partial\Vec{r}}{\partial v}\right\|\,\,\,\,\,\,\,\,
h_w=\left\|\frac{\partial\Vec{r}}{\partial w}\right\|\]

Con lo cual, los vectores unitarios del sistema son

\[\hat{u}=\frac{1}{h_u}\frac{\partial\Vec{r}}{\partial u}
\,\,\,\,\,\,\,\,\,
\hat{v}=\frac{1}{h_v}\frac{\partial\Vec{r}}{\partial v}\,\,\,\,\,\,\,\,\,
\hat{w}=\frac{1}{h_w}\frac{\partial\Vec{r}}{\partial w}\]

\bigbreak
Se dice que un sistema es ortogonal si $\hat{u}$, $\hat{v}$ y $\hat{w}$ son ortogonales.

\subsubsection{Operadores Diferenciales}

\begin{itemize}
    \item Gradiente:
    \[\nabla f =
    \frac{1}{h_u}\frac{\partial f}{\partial u}\hat{u}+
    \frac{1}{h_v}\frac{\partial f}{\partial v}\hat{v}+
    \frac{1}{h_w}\frac{\partial f}{\partial w}\hat{w}\]
    \item Divergencia:
    \[\nabla\cdot\Vec{F}=\frac{1}{h_uh_vh_w}\left(
    \frac{\partial}{\partial u}(F_uh_vh_w) +
    \frac{\partial}{\partial v}(h_uF_vh_w) +
    \frac{\partial}{\partial w}(h_uh_vF_w)\right)\]
    \item Rotor:
    \[\left(\nabla\times\Vec{F}\right)_u = \frac{1}{h_vh_w}
    \left(\frac{\partial}{\partial v}(F_wh_w)-
    \frac{\partial}{\partial w}(F_vh_v)\right)\]
    \[\left(\nabla\times\Vec{F}\right)_v = \frac{1}{h_uh_w}
    \left(\frac{\partial}{\partial w}(F_uh_u)-
    \frac{\partial}{\partial u}(F_wh_w)\right)\]
    \[\left(\nabla\times\Vec{F}\right)_w = \frac{1}{h_vh_u}
    \left(\frac{\partial}{\partial u}(F_vh_v)-
    \frac{\partial}{\partial v}(F_uh_u)\right)\]
\end{itemize}

Demostraciones: \ref{Dem:Gradiente}; \ref{Dem:Rotor}

\subsubsection{Uso de los Factores Escalares}

Una característica importante de los factores escalares es que verifican

\[\frac{\partial\Vec{r}}{\partial u} = h_u\hat{u}\]

de modo que, cuando se trabaja con $\frac{\partial\Vec{r}}{\partial u}$ reemplazarlo por $h_u\hat{u}$ suele facilitar los cálculos puesto que $h_u$ es un escalar y $\hat{u}$ es un vector unitario además de ortogonal a $\hat{v}$ y $\hat{w}$.
\medbreak
Un ejemplo de esto es el cálculo de los diferenciales de linea, superficie y volumen para el sistema de coordenadas
\medbreak
\begin{itemize}
    \item Linea:
    \begin{equation}
    \begin{split}
        d\Vec{r} &= \frac{\partial\Vec{r}}{\partial t}dt\\
        &= \frac{\partial\Vec{r}}{\partial u}
        \frac{\partial u}{\partial t}dt +
        \frac{\partial\Vec{r}}{\partial v}
        \frac{\partial v}{\partial t}dt +
        \frac{\partial\Vec{r}}{\partial w}
        \frac{\partial w}{\partial t}dt\\
        &= \frac{\partial\Vec{r}}{\partial u}du+
        \frac{\partial\Vec{r}}{\partial v}dv+
        \frac{\partial\Vec{r}}{\partial w}dw\\
        &= h_u\,du\hat{u}+h_v\,dv\hat{v}+h_w\,dw\hat{w}\\
    \end{split}
    \nonumber
    \end{equation}
    \item Superficie:
    \begin{equation}
    \begin{split}
        d\Vec{S} &= \left(\frac{\partial\Vec{r}}{\partial v}
        \times\frac{\partial\Vec{r}}{\partial w}\right)dvdw+
        \left(\frac{\partial\Vec{r}}{\partial w}
        \times\frac{\partial\Vec{r}}{\partial u}\right)dwdu+
        \left(\frac{\partial\Vec{r}}{\partial u}
        \times\frac{\partial\Vec{r}}{\partial v}\right)dudv\\
        &= (h_v\hat{v}\times h_w\hat{w})dvdw+
        (h_w\hat{w}\times h_u\hat{u})dwdu+
        (h_u\hat{u}\times h_v\hat{v})dudv\\
        &= h_vh_w(\hat{v}\times\hat{w})dvdw+
        h_wh_u(\hat{w}\times\hat{u})dwdu+
        h_uh_v(\hat{u}\times\hat{v})dudv\\
        &= h_vh_w\,dvdw\hat{u}+
        h_wh_u\,dwdu\hat{v}+
        h_uh_v\,dudv\hat{w}\\
    \end{split}
    \nonumber
    \end{equation}
    \item Volumen: \[dV = det(J\Vec{r})\,dudvdw = h_uh_vh_w\, dudvdw\]
\end{itemize}

\subsection{Curvas}

se dice que $\Gamma\subseteq\mathbb{R}^n$ es una curva en $\mathbb{R}^n$ si existe una función continua $\Vec{r}:D\subseteq\mathbb{R}\to\mathbb{R}^n$ tal que

\[\Gamma = \Vec{r}(D)\]

A la función $\Vec{r}$ se le llama parametrización de $\Gamma$. A $\Gamma$ se le apoda:

\begin{itemize}
    \item \textbf{Suave}: si $\Vec{r}$ es de clase $\mathcal{C}^1$
    \item \textbf{Simple}: si $\Vec{r}$ es de clase $\mathcal{C}^1$ e inyectiva
    \item \textbf{Cerrada}: si es suave y admite una parametrización $\Vec{r}:[a,b]\to\mathbb{R}^n$ tal que $\Vec{r}(a)=\Vec{r}(b)$
    \item \textbf{Regular}: si es suave y se verifica que
    \[\forall t\in D:\, \left\|\frac{d\Vec{r}}{dt}(t)\right\|>0\]
\end{itemize}

\subsubsection{Parametrizaciones Equivalentes}

Dos parametrizaciones $\Vec{r_1}:D\to\mathbb{R}^n$ y $\Vec{r_2}:E\to\mathbb{R}^n$ de una misma curva son equivalentes si existe una biyección $\theta\in\mathcal{C}^1(D,E)$ tal que

\[\forall t\in D:\, \Vec{r_1}(t) = \Vec{r_2}(\theta(t))\]

A $\theta$ se le denomina reparametrización. Si $\theta$ es creciente se dice que $\Vec{r_1}$ y $\Vec{r_2}$ recorren la curva en el mismo sentido, si es decreciente la recorren en sentidos opuestos.
\medbreak
Para toda curva $\Gamma$ simple y regular se cumple que

\begin{itemize}
    \item Si $\Gamma$ es cerrada, todas sus parametrizaciones inyectivas son equivalentes
    \item Si $\Gamma$ no es cerrada, todas sus parametrizaciones regulares son inyectivas y equivalentes
\end{itemize}

\subsection{Integral de Linea}

\subsubsection{Campos Escalares}

\[\int_\Gamma f\, dr= \int f(\Vec{r}(t))\left\|
\frac{d\Vec{r}}{dt}\right\|\,dt\]

\subsubsection{Campos Vectoriales}

A estas integrales se les llama integral de trabajo

\[\int_\Gamma \Vec{F}\cdot d\Vec{r}=
\int \Vec{F}(\Vec{r}(t))\frac{d\Vec{r}}{dt}\cdot dt\]

Si $\Gamma$ es cerrada, se suele notar como

\[\oint_\Gamma \Vec{F}\cdot d\Vec{r}\]

\subsection{Superficies Parametrizables}

Se dice que $S\subseteq\mathbb{R}^3$ es una superficie parametrizable si existe $\varphi:D\subseteq\mathbb{R}^2\to\mathbb{R}^3$ tal que

\[S = \varphi(D)\]

A la función $\varphi$ se le llama parametrización de $S$. A $S$ se le apoda:

\begin{itemize}
    \item \textbf{Suave}: si $\varphi$ es de clase $\mathcal{C}^1$
    \item \textbf{Simple}: si $\varphi$ es inyectiva
    \item \textbf{Cerrada}: si encierra a un volumen, es decir, existe $A\subseteq\mathbb{R}^3$ tal que\newline $S=\mathrm{Fr}(A)$
\end{itemize}

\subsubsection{Vectores Tangentes}

Para un punto $(u_o, v_o)\in D$ tal que las derivadas parciales de $\varphi$ no se anulan, las funciones $u=\varphi(\cdot, v_o)$ y $v=\varphi(u_o, \cdot)$ descriven curvas en $S$. Se definen los vectores tangentes a $S$ en $\varphi(u_o, v_o)$ como

\[\hat{t_u} = \frac{u'(u_o)}{\|u'(u_o)\|} = \frac{1}{\|u'(u_o)\|}
\frac{\partial\varphi(u_o, v_o)}{\partial u}
\,\,\,\,\,\,\,\,
\hat{t_v} = \frac{v'(v_o)}{\|v'(v_o)\|} = \frac{1}{\|v'(v_o)\|}
\frac{\partial\varphi(u_o, v_o)}{\partial v}\]

Si $\hat{t_u}$ y $\hat{t_v}$ son linealmente independientes, se dice que $\varphi$ es regular.

\subsubsection{Campo de Normales}

Si $\varphi$ es regular, se define un campo de normales de $S$ como

\begin{equation}
\begin{split}
    \hat{n}: &\,\,\,\,\,\,S\,\,\,
    \longrightarrow \,\, \mathbb{R}^3\\
    &(u, v) \longmapsto \hat{n}(u,v)
    =\frac{\hat{t_u}\times\hat{t_v}}{\|\hat{t_u}
    \times\hat{t_v}\|}\\
\end{split}
\nonumber
\end{equation}

los vectores $\hat{n}$ son perpendiculares a S.

\subsubsection{Superficies Orientables}

Se dice que $s$ es orientable si toda curva cerrada contenida en ella se puede recorrer en dos sentidos, esto equivale a decir que existen dos campos normales asociados a $S$ con signos opuestos. En una superficie orientable se pueden distinguir dos caras. Si además $\hat{n}$ es continua, entonces $S$ es regular orientable.

\subsection{Integral de Superficie}

\subsubsection{Campos Escalares}

\[\int_S f\,dS = \iint f(\varphi(u, v))\left\|
\frac{\partial\varphi}{\partial u}\times
\frac{\partial\varphi}{\partial v}\right\|\,dudv\]

\subsubsection{Campos Vectoriales}

A estas integrales se les llama integral de flujo

\begin{equation}
\begin{split}
\int_S \Vec{F}\cdot d\Vec{S} & = \int_S \Vec{F}\cdot\hat{n} \,dS\\
& =\iint \Vec{F}(\varphi(u, v))\cdot\left(
\frac{\partial\varphi}{\partial u}\times
\frac{\partial\varphi}{\partial v}\right)\,dudv\\
\end{split}
\nonumber
\end{equation}

\subsection{Teorema de la Divergencia}

Sean $\Omega\subseteq\mathbb{R}^3$ tal que $\partial\Omega=\mathrm{Fr}(\Omega)$ es una superficie cerrada regular orientable y $\Vec{F}\in\mathcal{C}^1(\Omega\cup\partial\Omega,\mathbb{R}^3)$, se verifica que

\[\int_{\partial \Omega}\Vec{F}\cdot d\Vec{S}=
\int_\Omega\nabla\cdot\Vec{F}\,dV\]

con $\Vec{S}$ orientado al exterior de $\Omega$.

\subsection{Teorema de Stokes o del Rotor}

Sean $S\subseteq\mathbb{R}^3$ una superficie regular orientable cuyo borde $\partial S$ es una curva cerrada, simple y regular, y $\Vec{F}\in\mathcal{C}^1(S\cup\partial S,\mathbb{R}^3)$, se verifica que

\[\oint_{\partial S}\Vec{F}\cdot d\Vec{r}=
\int_S\nabla\times\Vec{F}\cdot d\Vec{S}\]

\subsection{Teorema de Green}

Sean $S\subseteq\mathbb{R}^2$ acotada tal que $\partial S = \mathrm{Fr}(S)$ es una curva cerrada, simple y regular, y $f,g$ campos escalares de clase $\mathcal{C}^1$, se verifica que

\[\oint_{\partial S}f\,dx+g\,dy = \int_S
\left(\frac{\partial g}{\partial x}-
\frac{\partial f}{\partial y}\right)\,dxdy\]

\subsection{Campos Conservativos}

Sean $\Vec{F}$ un campo vectorial y $P$ un campo escalar, se dice que $\Vec{F}$ es conservativo y $P$ el potencial de $\Vec{F}$, si se verifica que

\[\Vec{F} = \nabla P\]
\bigbreak
Alternativamente, se dice que $\Vec{F}$ es conservativo si $\Vec{F} = -\nabla U$. Esta definición se usa sobre todo en física, y es equivalente a decir $-U = P$.

\bigbreak
Se cumple además que, para $\Gamma$ una curva parametrizada por $\Vec{r}:[a,b]\to\mathbb{R}^3$

\begin{equation}
\begin{split}
    \int_\Gamma\Vec{F}\cdot d\Vec{r} &=
    \int^b_a\Vec{F}(\Vec{r})\cdot\frac{d\Vec{r}}{dt}\,dt\\
    &=\int^b_a\nabla P(\Vec{r})\cdot\frac{d\Vec{r}}{dt}\,dt\\
    &=\int^b_a\frac{d}{dt} P(\Vec{r})\,dt\\
    &=P(\Vec{r}(b))-P(\Vec{r}(a))
\end{split}
\nonumber
\end{equation}
\bigbreak
Para $\Vec{F}$ un campo vectorial continuo sobre un abierto convexo $\Omega\subseteq\mathbb{R}^3$, son equivalentes:

\begin{itemize}
    \item $\Vec{F}$ es conservativo en $\Omega$
    \item Para toda curva $\Gamma\subseteq\Omega$ cerrada y regular por pedazos se tiene
    \[\oint_\Gamma\Vec{F}\cdot d\Vec{r}=0\]
    \item Para todo par de curvas regulares $\Gamma_1, \Gamma_2\subseteq\Omega$ con iguales puntos inicial y final, se tiene
    \[\int_{\Gamma_1}\Vec{F}\cdot d\Vec{r} =
    \int_{\Gamma_2}\Vec{F}\cdot d\Vec{r}\]
\end{itemize}

Antes de realizar una integral de trabajo, siempre es conveniente verificar si el campo vectorial es conservativo, o si es de la forma $\Vec{F}=\nabla g + \Vec{G}$.

\newpage