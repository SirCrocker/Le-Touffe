\subsection{Soluciones}

\sol{1}\\
\bigbreak

En coordenadas esféricas se tiene que

\begin{equation}
\begin{split}
    &\sqrt{x^2+y^2+z^2}=r\\
    &\sqrt{x^2+y^2}= r\sin{\theta}
\end{split}
\nonumber
\end{equation}

De modo que la función $f$ expresada en coordenadas esféricas es

\[f(r,\phi,\theta)=\theta r\]

facilitando así el cálculo del gradiente

    \[\nabla f = \frac{\partial f}{\partial r}\hat{r}
    +\frac{1}{r}\frac{\partial f}{\partial \theta}\hat{\theta}=\theta\hat{r}+\hat{\theta}\]

\bigbreak
\bigbreak

\sol{2}\\
\bigbreak
Considerando $\Omega$ como la esfera de radio $R$ centrada en el origen, por teorema de la divergencia se tiene que

\begin{equation}
\begin{split}
    \oint_\mathcal{S} g\nabla f\cdot d\Vec{S}&=\int_\Omega
    \nabla\cdot\left(g\nabla f\right)\,dV\\
    &= \int_\Omega g\nabla^2 f+\nabla g\cdot\nabla f\,dV\\
    &= \int_\Omega g\nabla^2 f\,dV\\
    &= \int_\Omega\,dV\\
    &=\frac{4\pi R^3}{3}
\end{split}
\nonumber
\end{equation}

\newpage

\sol{3}\\
\bigbreak

\begin{enumerate}[label=\alph*)]
    \item Orientado según la norma positiva, el borde de $A$ es la curva que recorre $y=x^3$ desde $x=0$ a $x=1$ y luego $y=x^2$ desde $x=1$ a $x=0$. Calculada directamente la integral es
    \begin{equation}
    \begin{split}
        \oint_{\partial A} y^3\,dx + 2xy^2\,dy &= \int^1_0x^9\,dx+\int^0_1x^6\,dx+\int^1_02y^{\frac{7}{3}}\,dy+\int^0_12y^{\frac{5}{2}}\,dy\\
        &= \frac{1}{10}-\frac{1}{7}+\frac{6}{10}-\frac{4}{7}\\
        &= \frac{7}{10}-\frac{5}{7}\\
        &= -\frac{1}{70}\\
    \end{split}
    \nonumber
    \end{equation}
    Usando el teorema de Green se tiene
    \begin{equation}
    \begin{split}
        \oint_{\partial A} y^3\,dx + 2xy^2\,dy &=
        \int_A 2y^2-3y^2\,dxdy\\
        &= \int_A -y^2\,dxdy\\
        &= \int^1_0\int^{x^2}_{x^3}-y^2\,dydx\\
        &= \int^1_0 \frac{x^9}{3}-\frac{x^6}{3} \,dx\\
        &= \frac{1}{30}-\frac{1}{21}\\
        &= -\frac{1}{70}\\
    \end{split}
    \nonumber
    \end{equation}
    \item Orientado según la norma positiva, el borde de $B$ es la curva que recorre $y=x^2$ desde $x=0$ a $x=1$ y luego $y=\sqrt{x}$ desde $x=1$ a $x=0$. Calculada directamente la integral es
    \begin{equation}
    \begin{split}
        \oint_{\partial B} (x+y)^2\,dx + (2xy+x^2-y^3)\,dy &= \int^1_0(x+x^2)^2\,dx+\int^0_1(x+\sqrt{x})^2\,dx\\
        &\,\,\,\,\,\,+\int^1_02y^{\frac{3}{2}}+y-y^3\,dy+
        \int^0_1y^3+y^4\,dy\\
        &= \frac{1}{3}+\frac{1}{2}+\frac{1}{5}-\frac{1}{3}-
        \frac{4}{5}-\frac{1}{2}+\frac{4}{5}+\frac{1}{2}-
        \frac{1}{4}-\frac{1}{4}-\frac{1}{5}\\
        &= \frac{1}{2}-\frac{1}{4}-\frac{1}{4}\\
        &=0\\
    \end{split}
    \nonumber
    \end{equation}
    Usando el teorema de Green se tiene
    \begin{equation}
    \begin{split}
        \oint_{\partial B} (x+y)^2\,dx + (2xy+x^2-y^3)\,dy &=
        \int_B 2(x+y)-2y-2x\,dxdy\\
        &= \int_B 0\,dxdy\\
        &= 0\\
    \end{split}
    \nonumber
    \end{equation}
    \item $C$ corresponde al disco unitario en el plano $xy$, integrando directamente por medio de un cambio de variables a coordenadas polares se tiene
    \begin{equation}
    \begin{split}
        \oint_{\partial C}(x^3-y^3)dx+(x^3+y^3)dy&=
        \oint_{\partial C}((x^3-y^3)\hat{x}+(x^3+y^3)\hat{y})
        \cdot d\Vec{r}\\
        &=\oint_{\partial C}((\cos^3{\phi}-\sin^3{\phi})\hat{x}
        +(\cos^3{\phi}+\sin^3{\phi})\hat{y})
        \cdot \hat{\phi}d\phi\\
        &=\int^{2\pi}_0-\sin{\phi}\cos^3{\phi}+\sin^4{\phi}
        +\cos^4{\phi}+\cos{\phi}\sin^3{\phi}\,d\phi\\
        &=\int^{2\pi}_0\sin^4{\phi}+\cos^4{\phi}\,d\phi\\
        &=\int^{2\pi}_01-2\sin^2{\phi}\cos^2{\phi}\,d\phi\\
        &=\int^{2\pi}_01-\frac{1}{2}\sin^2{2\phi}\,d\phi\\
        &=\frac{3\pi}{2}\\
    \end{split}
    \nonumber
    \end{equation}
    Usando teorema de Green se tiene
    \begin{equation}
    \begin{split}
        \oint_{\partial C}(x^3-y^3)dx+(x^3+y^3)dy&=
        \int_C 3x^2+3y^2\,dxdy\\
        &=\int_C3(\rho^2\cos^2{\phi}+\rho^2\sin^2{\phi})\rho\,d\phi d\rho\\
        &=\int^1_0\int^{2\pi}_03\rho^3\,d\phi d\rho\\
        &=\int^1_06\pi\rho^3\,d\rho\\
        &=\frac{3\pi}{2}\\
    \end{split}
    \nonumber
    \end{equation}
\end{enumerate}
\newpage
\sol{4}\\
\bigbreak

\begin{enumerate}[label=\alph*)]
    \item La primitiva de la componente en $x$ de $\Vec{F}$ respecto a $x$ es
    \[\int F_x\,dx=\int -z\,dx=-zx+C\]
    Si se deriva respecto a $z$ se obtiene $-x \neq F_z$, por lo tanto no existe $u$ tal que $\nabla u = \Vec{F}$.\\
    Otra forma es calcular el rotor de $\Vec{F}$
    \[\nabla\times\Vec{F}=-2\hat{y}\]
    como el rotor no es nulo, $\Vec{F}$ no es conservativo.
    \item La divergencia de $\Vec{F}$ es
    \[\nabla\cdot\Vec{F}=1\]
    como no es nula no existe $\Vec{G}$ tal que $\nabla\times\Vec{G}=\Vec{F}$.
\end{enumerate}
\bigbreak

\sol{5}\\
\bigbreak

\begin{enumerate}[label=\alph*)]
    \item Dado que, en coordenadas cilíndricas, $\Vec{F}$ sólo posee componente en $\rho$ y esta no depende de $\phi$ ni de $z$, es fácil ver que su rotor es 0.
    \item \[\nabla\cdot\Vec{F}=\frac{1}{\rho}
    \lados{(}{\parfrac{F\rho}{\rho}}=\parfrac{F}{\rho}+
    \frac{F}{\rho}\]
    \item
    \begin{equation}
    \begin{split}
        \nabla\cdot\Vec{F}=0&\Leftrightarrow \parfrac{F}{\rho}=-\frac{F}{\rho}\\
        &\Leftrightarrow \int\frac{1}{F}\parfrac{F}{\rho}
        \,d\rho = -\int\frac{1}{\rho}\,d\rho\\
        &\Leftrightarrow \int\frac{1}{F}
        \,dF = -\int\frac{1}{\rho}\,d\rho\\
        &\Leftrightarrow \ln{F}=-\ln{\rho}+\ln{K}\\
        &\Leftrightarrow F=\frac{K}{\rho}\\
        &\Leftrightarrow \Vec{F}=
        \frac{K}{\rho}\hat{\rho}\\
    \end{split}
    \nonumber
    \end{equation}
\end{enumerate}

\newpage
\sol{6}\\

\begin{enumerate}[label=\alph*)]
    \item La primitiva de la componente en $x$ de $\Vec{F}$ con respecto a $x$ es
    \[f_x=\int F_x\,dx=\int y^2\cos{x}+z^3\,dx=y^2\sin{x}+xz^3+C\]
    derivándola respecto a $z$ e $y$ se obtiene
    \begin{equation}
    \begin{split}
        \parfrac{f_x}{y} &= 2y\sin{x}\\
        \parfrac{f_x}{z} &= 3xz^2\\
    \end{split}
    \nonumber
    \end{equation}
    Con esto, se define el campo escalar
    \[f=y^2\sin{x}+xz^3-4y+z^2\]
    que verifica $\nabla f=\Vec{F}$, es decir, $f$ es un potencial de $\Vec{F}$ y en consecuencia $\Vec{F}$ es conservativo.
    \item Al igual que en $a)$, calculamos la primitiva de una de las componentes de $\Vec{F}$ y luego sus derivadas parciales
    \begin{equation}
    \begin{split}
        f_x&=\int F_x\,dx=\int \frac{y}{z^2+4}\,dx=\frac{xy}{z^2+4}+C\\
        \parfrac{f_x}{y} &= \frac{x}{z^2+4}=F_y\\
        \parfrac{f_x}{z} &= \frac{-2xyz}{(z^2+4)^2}=F_z\\
    \end{split}
    \nonumber
    \end{equation}
    se concluye que $\Vec{F}$ es un campo conservativo con potencial $f_x$.
\end{enumerate}
\bigbreak
\sol{12}\\

\textbf{Directo:}\\

Usando coordenadas cilíndricas se puede expresar $\Vec{F}$ como

\[\Vec{F}=\cos{\phi}\hat{x}+\sin{\phi}\hat{y}+z\hat{z}\]

luego,

\begin{equation}
\begin{split}
    &\int_{\Lambda_1}\Vec{F}\cdot d\Vec{r}=\int^{2\pi}_\pi (\cos{\phi}\hat{x}+\sin{\phi}\hat{y})\cdot a\hat{\phi}d\phi=\int^{2\pi}_\pi -a\sin{\phi}\cos{\phi}+a\sin{\phi}\cos{\phi}d\phi=0\\
    &\int_{\Lambda_2}\Vec{F}\cdot d\Vec{r}=\int^1_0 (\cos{\phi}\hat{x}+\sin{\phi}\hat{y}+z\hat{z})\cdot a\hat{z}\,dz=\int^1_0a^2z\,dz=\frac{1}{2}a^2\\
    &\int_{\Lambda_3}\Vec{F}\cdot d\Vec{r}=\int^{2\pi}_\pi (\cos{\phi}\hat{x}+\sin{\phi}\hat{y}+a\hat{z})\cdot a\hat{\phi}d\phi=\int^{2\pi}_\pi -a\sin{\phi}\cos{\phi}+a\sin{\phi}\cos{\phi}d\phi=0\\
    &\int_\Lambda \Vec{F}\cdot d\Vec{r}=\sum^3_{i=1}\int_{\Lambda_i}\Vec{F}\cdot d\Vec{r}=\frac{1}{2}a^2\\
\end{split}
\nonumber
\end{equation}
\bigbreak
\textbf{Potencial:}\\

La primitiva de $F_x$ respecto a $x$ es

\[\int F_x\,dx=\int\frac{x}{\sqrt{x^2+y^2}}\,dx=
\sqrt{x^2+y^2}+C\]

es fácil ver que si se deriva respecto a $y$ se obtiene $F_y$, de modo que se puede construir un potencial

\[g = \sqrt{x^2+y^2}+\frac{1}{2}z^2\]

que verifica $\nabla g=\Vec{F}$. Con esto, la integral es

\[\int_\Lambda\Vec{F}\cdot d\Vec{r}=
\int_\Lambda\nabla g\cdot d\Vec{r}=g(-a,0,a)-g(-a,0,0)=\frac{1}{2}a^2\]

\bigbreak
\sol{13}\\

\textbf{Directo:}\\

La superficie encerrada por $\Gamma$ está parametrizada por

\[\Vec{\varphi}= r\cos{\phi}\,\hat{x}
+r\sin{\phi}\,\hat{y}\]

con $\phi\in [0,2\pi]$ y $r\in [0,1]$. Sus derivadas son

\begin{equation}
\begin{split}
    \parfrac{\varphi}{r}&=\cos{\phi}\,\hat{x}
+\sin{\phi}\,\hat{y}\\
    \parfrac{\varphi}{\phi}&=-r\sin{\phi}\,\hat{x}
+r\cos{\phi}\,\hat{y}\\
\end{split}
\nonumber
\end{equation}

luego, el área encerrada es


\begin{equation}
\begin{split}
    \int dS &= \int^1_0\int^{2\pi}_0\left\|\parfrac{\varphi}{r}\times\parfrac{\varphi}{\phi}\right\|\,d\phi dr\\
    &= \int^1_0\int^{2\pi}_03r\cos^4\phi\sin^2\phi+
    3r\cos^2\phi\sin^4\phi\,d\phi dr\\
    &= \int^1_0\int^{2\pi}_03r\cos^2\phi\sin^2\phi\,d\phi dr\\
    &= \int^1_0\int^{2\pi}_0\frac{3}{4}r\sin^2(2\phi)\,d\phi dr\\
    &= \int^1_0\frac{3}{4}\pi r dr\\
    &= \frac{3}{8}\pi\\
\end{split}
\nonumber
\end{equation}

\bigbreak

\textbf{Green:}\\

Considerando el campo $\Vec{F}=x\hat{y}$, por teorema de Green se tiene que

\[\oint_\Gamma \Vec{F}\cdot d\Vec{r}=\int_\Gamma 0\,dx+x\,dy=\iint\,dxdy=\int\,dS\]
\bigbreak
Además, $\Gamma$ está parametrizada por

\[\Vec{r}=\cos^3\phi\,\hat{x}+\sin^3\phi\,\hat{y}\]

con derivada

\[\frac{d\Vec{r}}{d\phi}=-3\sin\phi\cos^2\phi\,\hat{x}+3\cos\phi\sin^2\phi\,\hat{y}\]

con lo que se obtiene

\begin{equation}
\begin{split}
    \int\,dS&=\oint_\Gamma \Vec{F}\cdot d\Vec{r}\\
    &=\int^{2\pi}_03\cos^4\phi\sin^2\phi\,d\phi\\
    &=\frac{1}{2}\int^{2\pi}_0\cos^4\phi\,d\phi\\
    &=\frac{3}{8}\int^{2\pi}_0\cos^2\phi\,d\phi\\
    &=\frac{3}{8}\pi\\
\end{split}
\nonumber
\end{equation}

\newpage