\section{Problemas y Ejercicios}


\np

Calcule el gradiente de

\[f(x, y, z)=\arcsin\left(\frac{\sqrt{x^2+y^2}}{\sqrt{x^2+y^2+z^2}}\right)
\sqrt{x^2+y^2+z^2}\]
\bigbreak
\np

Considere $\mathcal{S}$ una superficie esférica centrada en el origen de radio $R$ y $f$, $g$ campos escalares tales que sus gradientes son ortogonales y $\nabla^2 f = \frac{1}{g}$. Calcule

\[\oint_\mathcal{S}g\nabla f\cdot d\Vec{S}\]
\bigbreak
\np

Considere los conjuntos

\begin{equation}
\begin{split}
    & A = \{(x,y)\in [0,0]\times[1,1]\,|\,x^3\leq y\leq x^2\}\\
    & B = \{(x,y)\in [0,0]\times[1,1]\,|\,x^2\leq y\leq \sqrt{x}\}\\
    & C = \{(x,y)\in \mathbb{R}^2\,|\, 0 \leq x^2+y^2\leq 1\}\\
\end{split}
\nonumber
\end{equation}

Usando $\partial$ para denotar el borde de una superficie, calcule

\begin{equation}
\begin{split}
    \mathrm{a})& \int_{\partial A} y^3\,dx + 2xy^2\,dy\\
    \mathrm{b})& \int_{\partial B} (x+y)^2\,dx + (2yx + x^2 - y^3)\,dy\\
    \mathrm{c})& \int_{\partial C} (x^3-y^3)\,dx + (x^3+y^3)\,dy
\end{split}
\nonumber
\end{equation}
\bigbreak
\np

Dada $\Vec{F}= -z\hat{x}+y\hat{y}+x\hat{z}$

\begin{enumerate}[label=\alph*)]
    \item ¿Existe $u$ tal que $\nabla u = \Vec{F}$?
    \item ¿Existe $\Vec{G}$ tal que $\nabla\times
    \Vec{G}=\Vec{F}$?
\end{enumerate}
\newpage
\np

Se dice que un campo vectorial $\Vec{F}:\mathbb{R}^3\setminus\{eje Z\}\to\mathbb{R}^3$ tiene simetría cilíndrica si verifica que

\[\Vec{F}(\Vec{r}) = F(\rho)\hat{\rho}\]

con $F\in\mathcal{C}^1(\mathbb{R}^+,\mathbb{R})$

\begin{enumerate}[label=\alph*)]
    \item Calcule $\nabla\times\Vec{F}$
    \item Calcule $\nabla\cdot\Vec{F}$
    \item Demuestre que la divergencia de $\Vec{F}$ es nula si y sólo si
    \[\Vec{F}= \frac{k}{\rho}\hat{\rho}\]
    con $k$ una constante real
\end{enumerate}
\bigbreak
\np

Verifique si los siguientes campos son conservativos

\begin{enumerate}[label=\alph*)]
    \item $\Vec{F} = (y^2\cos{x}+z^3)\hat{x} + (2y\sin{x}-4)\hat{y} + (3xz^2+2z)\hat{z}$
    \item $\Vec{F} = \frac{y}{z^2+4}\hat{x} + \frac{x}{z^2+4}\hat{y} - \frac{2xyz}{(z^2+4)^2}\hat{z}$
\end{enumerate}% estaría bien agregar más
\bigbreak
\np

Pruebe que 
\[\nabla\times(\nabla\times\Vec{F}) = \nabla(\nabla\cdot\Vec{F})-\nabla^2\Vec{F}\]
\bigbreak
\np

\begin{enumerate}[label=\alph*)]
    \item Demuestre que
    \[\hat{z}=\cos{\theta}\hat{r}-\sin{\theta}\hat{\theta}\]
    \item Considere el campo escalar
    \[f = \frac{\alpha}{r^2}\hat{z}\cdot\hat{r}\]
    Calcule el gradiente de $f$
    \item Considere el campo vectorial
    \[\Vec{F}=\frac{\alpha}{r^2}\hat{z}\times\hat{r}\]
    Calcule el rotor de $\Vec{F}$
    \item Dadas $a$ y $K$ constantes, considere el campo escalar
    \[U(r) = -K\frac{e^{-ar}}{r}\]
    Calcule el gradiente de $U$
    \item Pruebe que $\nabla^2 U= a^2U$ en $\mathbb{R}^3\setminus\{0\}$
\end{enumerate}
\bigbreak
\np

\begin{enumerate}[label=\alph*)]
    \item Encuentre el área de la superficie $S_a = \{\Vec{r}\in\mathbb{R}^3\,|\,x^2+y^2\leq 2\wedge z=xy\}$
    \item Encuentre el área de la superficie $S_b = \{\Vec{r}\in\mathbb{R}^3\,|\,0<x<y\wedge 00<y<1\wedge z=\cosh{x}\}$
    \item Sean $D$ el disco unitario en el plano $xy$ y $\phi(x,y)=(x-y)\hat{x}+(x+y)\hat{y}+xy\hat{z}$, encuentre el área de $\phi(D)$
    \item El cilindro $x^2+y^2=x$ en $\mathbb{R}^3$ divide la superficie de la esfera unitaria en dos secciones, $S_1$ está dentro del cilindro y $S_2$ está fuera. Encuentre calcule el cociente entre el área de $S_2$ y la de $S_1$
\end{enumerate}
\bigbreak
\np

Sea $\Gamma\subseteq\mathbb{R}^3$ un curva parametrizada por

\[\Vec{r}(t) = \cos{(t)}\hat{\rho}+\sin{(2t)}\hat{z}\]

Con $t\in [-\frac{\pi}{2},\frac{\pi}{2}]$. Considere la superficie $S$ que se obtiene al rotar $\Gamma$ en torno al eje $z$

\begin{enumerate}[label=\alph*)]
    \item Encuentre una parametrización regular para $S$ en función de $(t, \theta)$
    \item Calcule el volumen de la región encerrada por $S$
\end{enumerate}
\bigbreak
\np

Considere el campo vectorial en coordenadas esféricas

\[\Vec{F} = r\hat{\theta}\]

Calcule el flujo de $\Vec{F}$ a través de la semiesfera descrita por

\[x^2+(y-1)^2+z^2=1\]

con $x\geq 0$
\bigbreak
\np

Considere las curvas

\begin{equation}
\begin{split}
    &\Lambda_1:\,\Vec{r}_1(t)=a\cos{t}\,\hat{x}+a\sin{t}\,\hat{y}\quad t\in [\pi,2\pi]\\
    &\Lambda_2:\,\Vec{r}_2(t)=a\hat{x}+at\hat{z}\quad t\in [0,1]\\
    &\Lambda_3:\,\Vec{r}_3(t)=a\cos{t}\,\hat{x}+a\sin{t}\,\hat{y}+a\hat{z}\quad t\in [0,\pi]\\
    &\Lambda=\Lambda_1\cup\Lambda_2\cup\Lambda_3\\
\end{split}
\nonumber
\end{equation}

y el campo

\[\Vec{F}=\frac{x}{\sqrt{x^2+y^2}}\hat{x}+\frac{y}{\sqrt{x^2+y^2}}+z\hat{z}\]

Calcule

\[\int_\Lambda\Vec{F}\cdot d\Vec{r}\]
\bigbreak

\np

Considere la curva $\Gamma$ dada por la ecuación $x^{2/3}+y^{2/3}=1$. Calcule el área encerrada por $\Gamma$ en el plano $xy$.

\newpage