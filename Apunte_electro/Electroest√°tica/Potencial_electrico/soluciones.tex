\subsection{Soluciones}


\sol{1}
\begin{enumerate}[label=\alph*)]
    \item \textit{Respuesta:} Si es posible \newline Primero definamos $V(r = \infty)=0$, que es posible al ser una distribución finita de cargas. \\ Ahora notemos que el sistema está compuesto de dos cargas, \textit{q} y \textit{-q} tal que a su distancia media el potencial es cero. Ahora el potencial en un punto cualquiera vendrá dado por \[V(\Vec{r})=\sum_{i=1}^2\frac{q_i}{4\pi\epsilon_o}\frac{1}{\abs{\Vec{r}-\Vec{r_i}}}\] en donde reemplazando los valores de las cargas nos queda (con $r_+$ la distancia a la carga \textit{q} y $r_{-}$ la distancia a la carga \textit{-q}) \[V(\Vec{r})=\frac{q}{4\pi\epsilon_o}\left(\frac{1}{\norma{\Vec{r} - \Vec{r_+}}} - \frac{1}{\norma{\Vec{r} - \Vec{r_{-}}}} \right)\]
    Lo que implica que el trabajo hecho para mover una carga $q_0$ desde un punto a otro será $W_{ext} = q_0 \Delta V = q_0(V(\Vec{r})-V(\infty)) = q_0V(\Vec{r})$, con esto podemos darnos cuenta que si el desplazamiento es a través de la recta perpendicular a la distancia media entre ambas cargas, el trabajo será siempre cero, ya que la distancia en norma hacia ambas cargas será igual, dando cero el potencial en cualquiera de los puntos. 
    
    \item \textit{Respuestas:} Cero y No \newline Sabemos que la diferencia de potencial (\textit{d.d.p}) viene dado por \[V(\Vec{r_b}) - V(\Vec{r_a}) = - \int_{\Vec{r_a}}^{\Vec{r_b}}\Vec{E}\cdot\Vec{dl}\]
    Por lo que si el campo eléctrico es cero en todo el camino, entonces también lo es en $\Vec{r_a}$ y $\Vec{r_b}$ y queda $V(\Vec{r_b}) - V(\Vec{r_a}) = 0$ \newline Tenemos que el potencial en $\Vec{r_a}$ es cero, ya que el campo es cero en ese punto, pero desconocemos si el campo es cero en todo el espacio, por lo que si existe un sector en donde el campo es distinto de cero y se escoge una trayectoria que pase por ese sector se tendrá entonces que la diferencia de potencial \textbf{no} será cero en cualquier punto.
    
    \item \textit{Respuesta:} No \newline Las relaciones entre campo y potencial nos permite despejar uno en función del otro solo si conocemos su expresión general (ya sea global o local), por lo que conociendo un valor específico no se puede obtener el valor del campo. Aunque si se conociera el valor del potencial en un punto, el punto en donde se hace cero y la densidad de carga del espacio, si se podría haciendo uso de la ecuación de Poisson.

\end{enumerate}
\bigbreak
\sol{2}
\begin{enumerate}[label=\alph*)]
    \item \textit{Respuesta:} \newline Para $r < a$, $V(\rho) = -\frac{\rho_0}{4\epsilon_0}\rho^2$ y $\vec{E}(\rho) = \frac{\rho_0}{2\epsilon_0}\rho\hat{\rho}$
    \newline Para $r > a$, $V(\rho) = -\frac{\rho_0a^2}{4\epsilon_0} -\frac{\rho_0a^2}{2\epsilon_0}\ln\left(\frac{\rho}{a}\right)$ y $\vec{E}(\rho) = \frac{\rho_0}{2\epsilon_0}\frac{a^2}{\rho}\hat{\rho}$ 
    
%    \textbf{\textit{Revisar}}
    \begin{enumerate}[label=\arabic*)]
        \item Siguiendo la idea de la Ley de Gauss, tenemos que el flujo corresponde a la cantidad de carga encerrada, pudiéndose despejar el campo en casos de simetría. En este caso notemos que esto es posible, y que en $r=0$ la carga encerrada correspondería a cero, al ser el eje del cilindro. Así tendremos que el campo será cero en ese punto.
        \item Si es posible, si existiera una densidad de carga superficial tal que aumentara bruscamente el campo este sería discontinuo en $r=a$
        \item Como fue explicado al inicio de la sección se tiene que el potencial no puede ser discontinuo. 
        \item Haciendo uso de la lógica en (1), al no haber carga en $r=0$ podemos establecer el punto de referencia ahí. Además es necesario al no poder establecerlo en infinito a causa de la existencia de una distribución infinita de cargas.
    \end{enumerate}
    \bigbreak
    \textit{Solución} \newline
    Tenemos que las ecuaciones de Poisson y Laplace respectivamente son $\nabla^2V = \frac{-\rho}{\epsilon_0}$ y $\nabla^2V = 0$, con $\rho_0$ la densidad de carga del objeto. Además conocemos que $\Vec{\nabla}V = -\Vec{E}$. Donde $\nabla$ representa el gradiente y $\nabla^2$ el laplaciano de $V$.
    
    En este caso, como las coordenadas están en cilíndricas tendremos que usar la divergencia y el gradiente en cilíndricas. Notemos que a causa de que es un cilindro infinito el campo eléctrico solo dependerá de la coordenada $\rho$ por argumentos de simetría, al igual que el potencial, facilitando la aplicación de las ecuaciones para el potencial y el campo eléctrico. 
    
    Analicemos por caso:
    
    I) \boxed{r < a} Utilizamos Poisson
    
    \[\nabla^2V = \frac{-\rho}{\epsilon_0} \implies \frac{1}{\rho}\frac{\partial}{\partial \rho}\left( \rho \frac{\partial V}{\partial \rho}\right) = -\frac{\rho_0}{\epsilon_0}\]
    \[\implies \rho\frac{\partial V}{\partial \rho} = -\frac{\rho_0}{\epsilon_0}\frac{\rho^2}{2} + A \quad (\text{A} \in \mathbb{R} )\]
    \[\implies V(\rho) = -\frac{\rho_0}{4\epsilon_0}\rho^2 + A\ln(\rho) + B \quad (\text{A, B} \in \mathbb{R})\]
    Ocupamos las condiciones de borde entregadas por el problema y despejamos $A$ y $B$,
    \[V(0) = A\ln(0^*) + B = 0 \implies A = B = 0\]
    \[\implies V(\rho) = -\frac{\rho_0}{4\epsilon_0}\rho^2\]
    Ahora haciendo uso del valor de $V(\rho)$ encontrado y de la relación $\vec{E} = -\vec{\nabla}V$, despejamos el campo eléctrico
    \begin{equation}
        \begin{split}
            &\vec{E} = - \vec{\nabla}V \\
            \implies &\vec{E} = - \frac{\partial}{\partial \rho}\left(\frac{-\rho_0 \rho^2}{4\epsilon_0}  \right)\hat{\rho} \\
            \implies &\vec{E} = \frac{\rho_0 \rho^2}{4\epsilon_0}2\rho\hat{\rho}\\
            \implies &\vec{E} = \frac{\rho_0}{2\epsilon_0}\rho\hat{\rho}
        \end{split}
        \nonumber
    \end{equation}
    Obteniéndose $V$ y $\vec{E}$ para $r < a$\\
    
    II) \boxed{r > a} \newline
    Hacemos uso de la ecuación de Maxwell ($\nabla \cdot \vec{E} = \frac{\rho_0}{\epsilon_0}$) donde en nuestro caso $\rho_0 = 0$ al no haber densidad de carga en el espacio fuera del cilindro. Así
    \begin{equation}
        \begin{split}
            &\nabla\cdot\vec{E} = 0\\
            \implies &\frac{1}{\rho}\left( \frac{\partial (E\cdot\rho)}{\partial \rho}\right) = 0\\
            \implies &E\rho = A \quad\quad;A \in \mathbb{R}\\
            \implies &\vec{E} = \frac{A}{\rho}\hat{\rho} \quad\quad;A \in \mathbb{R}
        \end{split}
        \nonumber
    \end{equation}
    Haciendo uso de la continuidad del campo eléctrico en $r = a$ y el valor para $\vec{E}$ encontrado para $r < a$, despejamos el valor de la constante $A$
    \begin{equation}
        \begin{split}
            &\vec{E}_{(r<a)}(a) = \frac{\rho_0}{2\epsilon_0}a\hat{\rho}\\ 
            &\vec{E}_{(r>a)}(a) = \frac{A}{a}\hat{\rho}\\
            \implies &A = \frac{\rho_0a^2}{2\epsilon_0}\\
            \implies &\vec{E}_{(r>a)} = \frac{\rho_0a^2}{2\epsilon_0}\frac{1}{\rho}\hat{\rho}
        \end{split}
        \nonumber
    \end{equation}
    Para encontrar el potencial ahora usamos la relación $\vec{E} = -\vec{\nabla}V$
    \begin{equation}
        \begin{split}
            &\vec{E} = - \vec{\nabla}V\\
            \implies &\frac{\rho_0a^2}{2\epsilon_0}\frac{1}{\rho} = - \frac{\partial (V)}{\partial \rho}\\
            \implies &-\frac{\rho_0a^2}{2\epsilon_0}\ln{\rho} - B = V(\rho) \quad\quad; B \in \mathbb{R}\\
        \end{split}
        \nonumber
    \end{equation}
    Y aplicando la continuidad del potencial en $r =a$, despejamos $B$
    \begin{equation}
        \begin{split}
            &V_{(r<a)}(a) = -\frac{\rho_0}{4\epsilon_0}a^2\\
            &V_{(r>a)}(a) = -\frac{\rho_0a^2}{2\epsilon_0}\ln{a} - B\\
            \implies &B + \frac{\rho_0a^2}{2\epsilon_0}\ln{a} = \frac{\rho_0}{4\epsilon_0}\rho^2\\
            \implies &B = \frac{\rho_0}{4\epsilon_0}a^2 - \frac{\rho_0a^2}{2\epsilon_0}\ln{a}\\
            \implies &V_{(r>a)} = -\frac{\rho_0a^2}{2\epsilon_0}\ln{\rho} -\frac{\rho_0}{4\epsilon_0}a^2 - \frac{\rho_0a^2}{2\epsilon_0}\ln{a}\\
            \implies &V_{(r>a)}(\rho) = -\frac{\rho_0}{4\epsilon_0}a^2 - -\frac{\rho_0a^2}{2\epsilon_0}\ln{\frac{\rho}{a}}
        \end{split}
        \nonumber
    \end{equation}
    
    
    \item A causa de que ya se han hecho ejercicios relacionados con la Ley de Gauss en problemas anteriores solo se dejarán indicaciones de como calcular para este caso. 
    
    Para obtener el valor del campo eléctrico y el potencial haciendo uso del teorema de Gauss ($\int\vec{E}\cdot\vec{dS} = Q_{enc}/\epsilon_0$) hay que sobreponer un cilindro con largo $\gg$ ancho sobre el cilindro ya existente, en el caso $r<a$ existirá una carga encerrada dependiente de $r$ y para $r>a$ existirá una carga constante. 
    
    Luego de haber despejado el campo se utiliza \[V(\vec{r_b}) - V(\vec{r_b}) = -\int_{\vec{r_b}}^{\vec{r_a}}\vec{E}\cdot\vec{dl} \] donde se parte calculando el potencial al interior del cilindro estableciendo $V(r=0) = 0$, y luego ocupando la continuidad en $r=a$ se calcula, de manera similar, el potencial en todo el espacio para $r>a$. Llegando a los resultados puestos en a)
\end{enumerate}
\bigbreak
\bigbreak
\sol{3}\newline % P1 \ Riquelme 2017-1 o P2 | Cordero 2017-2

$a)$ El espacio se divide en 3 subespacios dados por: $r<a$ (1), $a<r<b$ (2) y $b<r$ (3). Usando la ecuación de Poisson (calculada en \textbf{\ref{PoissonEsferas}}) se tiene que el potencial en cada subespacio es

\begin{itemize}
    \item $V_1 = B_1-\frac{A_1}{r}-\frac{\rho_0 r^2}{6\epsilon_o}$
    \item $V_2 = -\frac{k}{6}r^2$
    \item $V_3 = B_3-\frac{A_3}{r}$
\end{itemize}

Donde la densidad de carga en (3) es 0. Sigue encontrar las condiciones de borde para determinar las constantes.
\medbreak
Por ley de Gauss el campo eléctrico en (1) está dado por

\[\Vec{E_1} = \frac{\rho_0}{\epsilon_o}\frac{4\pi r^3}{3}
\frac{1}{4\pi r^2}\hat{r} = \frac{\rho_0r}{3\epsilon_o}\hat{r}\]

luego

\[\frac{\rho_0r}{3\epsilon_o}\hat{r} =
\Vec{E_1} = -\nabla V_1 = \left(
\frac{\rho_0r}{3\epsilon_o}-\frac{A_1}{r^2}
\right)\hat{r}\]
\bigbreak
lo que implica que $A_1 = 0$. Como el potencial es una función continua, se debe cumplir que $V_1(a) = V_2(a)$

\[B_1-\frac{\rho_0 a^2}{6\epsilon_o} = -\frac{k}{6}a^2
\Leftrightarrow
B_1= \frac{\rho_0-k\epsilon_o}{6\epsilon_o}a^2
\]

El campo eléctrico en (2) es

\[\Vec{E_2}=-\nabla V_2=\frac{k}{3}\Vec{r}\]

Para $V_3$, como el potencial en el infinito es nulo, $B_3$ debe ser 0, además, por continuidad se cumple que $V_3(b) = V_2(b)$

\[-\frac{A_3}{b} = -\frac{k}{6}b^2
\Leftrightarrow
A_3 = \frac{k}{6}b^3
\]

El campo eléctrico en (3) es

\[\Vec{E_3}=-\nabla V_3=-\frac{A_3}{r^2}\hat{r}\]

Se tiene así que

\begin{itemize}
    \item $V_1 = \frac{\rho_0-k\epsilon_o}{6\epsilon_o}a^2
    -\frac{\rho_0 r^2}{6\epsilon_o}$
    \item $V_2 = -\frac{k}{6}r^2$
    \item $V_3 = -\frac{kb^3}{6r}$
    \item $\Vec{E_1}=\frac{\rho_0 r}{3\epsilon_o}\hat{r}$
    \item $\Vec{E_2}=\frac{k}{3}\Vec{r}$
    \item $\Vec{E_3}=-\frac{kb^3}{6r^2}\hat{r}$
\end{itemize}
\bigbreak
$b)$ Por ley de Gauss $E_2$ se puede escribir como

\[E_2 = \frac{Q_a+Q_k(r)}{\epsilon_o}\frac{1}{4\pi r^2}\]

Donde $Q_a$ es la carga encerrada por la cáscara de radio $a$ incluyendo la superficie

\[Q_a = \frac{4\pi a^3}{3}\rho_0+4\pi a^2\sigma_1\]
\medbreak
y $Q_k(r)$ es la carga de la segunda capa que hay entre la cáscara de radio $a$ y otra de radio $r$. Como $Q_k(a)=0$, evaluando $E_2$ cuando $r\rightarrow a$ se tiene

\[\frac{k}{3}a = \frac{a}{3\epsilon_o}\rho_0 +
\frac{\sigma_1}{\epsilon_o} \implies \sigma_1 = \epsilon_o\frac{k}{3}a - \frac{a}{3}\rho_0\]

evaluando $E_2$ cuando $r\rightarrow b$ se tiene

\[\frac{k}{3}b = \frac{Q_b}{\epsilon_o}\frac{1}{4\pi b^2}\]

Donde $Q_b$ es la carga encerrada por la cáscara de radio $b$ excluyendo la superficie. Siguiendo la misma lógica, al evaluar $E_3$ cuando $r\rightarrow b$ se obtiene

\[-\frac{k}{6}b = \frac{Q_b}{\epsilon_o}\frac{1}{4\pi b^2}
+ \frac{\sigma_2}{\epsilon_o}\]

De las dos últimas ecuaciones se desprende que

\[\sigma_2 = -\epsilon_o\frac{k}{6}b-\frac{k}{3}b = -\frac{k}{2}b\]

El potencial $V_2$ calculado con la ecuación de Poisson es

\[V_2 = B_2-\frac{A_2}{r}-\frac{\rho_2 r^2}{6\epsilon_o}\]

Con las condiciones de borde $V_1(a) = V_2(a)$ y $V_2(b) = V_3(b)$

\[-\frac{k}{6}a^2 = B_2-\frac{A_2}{a}-\frac{\rho_2 a^2}{6\epsilon_o}\]
\[-\frac{k}{6}b^2 = B_2-\frac{A_2}{b}-\frac{\rho_2 b^2}{6\epsilon_o}\]

Se obtiene que

\[A_2=\frac{ab}{6}(b+a)\left(\frac{\rho_2}{\epsilon_o}-k\right)\]
\[B_2=\frac{b^2+ab+a^2}{6}
\left(\frac{\rho_2}{\epsilon_o}-k\right)\]

Luego

\[\Vec{E_2} = -\nabla V_2 =
-\frac{A_2}{r^2}\hat{r}+\frac{\rho_2 r}{3\epsilon_o}\hat{r}\]

Igualando esto al valor de $\Vec{E_2}$ que ya se tenía se puede obtener el valor de $\rho_2$

\[\frac{kr}{3} =
-\frac{A_2}{r^2}+\frac{\rho_2 r}{3\epsilon_o}
= \frac{ab}{6r^2}(b+a)\left(\frac{\rho_2}{\epsilon_o}-k\right)
+\frac{\rho_2 r}{3\epsilon_o}\]

\[\frac{k}{6r^2}(ab(b+a)-2r^2) = \frac{1}{6r^2}\frac{\rho_2}{\epsilon_o}(ab(b+a)-2r^2)\]

\[k\epsilon_o = \rho_2\]

%Con esto

%\[Q_k(b) = \frac{4\pi(b^3-a^3)}{3}k\epsilon_o\]
%\[Q_b = \frac{4\pi(b^3-a^3)}{3}k\epsilon_o + Q_a\]
%\[\frac{k}{3}b = \frac{(b^3-a^3)k}{b^2}+
%\frac{a^3}{3\epsilon_ob^2}\rho_0+
%\frac{a^2\sigma_1}{\epsilon_0b^2}\]
%\bigbreak

\bigbreak
c) Para encontrar el trabajo necesario para mover una carga $q$ desde $r =a$ hasta $r=b$, se puede usar la identidad $W_{ext} = -q \Delta V$, con $V(r) = -kr^2/6$. Por lo tanto
\[W_{ext} = -q(V(b) - V(a)) = -q\left( -\frac{k}{6}b^2 + \frac{k}{6}a^2 \right) = q\frac{k(a^2 - b^2)}{6}\]

\bigbreak


\sol{4}\newline\newline
\textit{Respuesta:} \newline\newline
a) $W = 0$
\newline
\newline
b) El trabajo representa la energía usada para mover la carga $q_i$ desde infinito hasta su posición final, esta energía es almacenada en el sistema como energía potencial. (más adelante se verá que la energía electrostática está realmente almacenada en los campos eléctricos)

Por lo que este trabajo representa que al mover las cargas a la configuración establecida no se obtiene energía potencial adicional en el sistema.
\newline
\newline
\textit{Solución}
\newline
\newline
Haciendo uso de la ecuación $W = \frac{1}{2}\sum_{i=1}^4q_iV(\Vec{r_i})$ se obtendrá el trabajo total. Notemos que $V(\Vec{r_i})$ corresponde al potencial causado por todas las cargas excepto la $i$, así, para la carga 1 ($q_1 = -2q, \hspace{4px} q_2 = 4q, \hspace{4px} q_3 = 2q \text{  y  } q_4 = q$) se tiene
\[V_{(2,3,4)}(\Vec{r_1}) = \frac{1}{4\pi\epsilon_0}\left[\frac{4q}{a} + \frac{2q}{a\sqrt{2}} + \frac{q}{a} \right] \implies \frac{1}{2}q_1V_{(2,3,4)}(\Vec{r_1})= \frac{-q}{\kte}\left[ \frac{5q}{a} + \frac{2q}{a\sqrt{2}}\right]\]
\newline Para la carga $q_2$ ($q_2 = 4q$) se tiene
\[V_{(1,3,4)}(\Vec{r_2}) = \frac{1}{4\pi\epsilon_0}\left[\frac{-2q}{a} + \frac{2q}{a} + \frac{q}{a\sqrt{2}} \right] \implies \frac{1}{2}q_2V_{(1,3,4)}(\Vec{r_2})= \frac{2q}{\kte} \frac{q}{a\sqrt{2}}\]
\newline Para la carga $q_3$ ($q_3 = 2q$) se tiene
\[V_{(1,2,4)}(\Vec{r_3}) = \frac{1}{4\pi\epsilon_0}\left[\frac{4q}{a} + \frac{-2q}{a\sqrt{2}} + \frac{q}{a} \right] \implies \frac{1}{2}q_3V_{(1,2,4)}(\Vec{r_3})= \frac{q}{\kte}\left[ \frac{5q}{a} - \frac{2q}{a\sqrt{2}}\right]\]
\newline Para la carga $q_4$ ($q_4 = q$) se tiene
\[V_{(1,2,3)}(\Vec{r_4}) = \frac{1}{4\pi\epsilon_0}\left[\frac{-2q}{a} + \frac{2q}{a} + \frac{4q}{a\sqrt{2}} \right] \implies \frac{1}{2}q_4V_{(1,2,3)}(\Vec{r_4})= \frac{1}{2}\frac{q}{\kte} \frac{4q}{a\sqrt{2}}\]

Sumando todos los términos encontrados, nos queda que 
\[W = \frac{4q^2}{\kte a \sqrt{2}} - \frac{4q^2}{\kte a \sqrt{2}} = 0\]


\newpage