\section{Energía Potencial Electrostática}

\subsection{Trabajo}

Dadas una carga $Q$ que genera un campo eléctrico $\Vec{E}$ y otra carga $q$ ubicada en la posición $\Vec{r_a}$ respecto a un origen arbitrario, el trabajo que debe realizar una fuerza externa para desplazar $q$ a un punto $\Vec{r_b}$ está dado por

%Cambio: dr -> dl
\[W_{ext} = \int^{\Vec{r_b}}_{\Vec{r_a}}\Vec{F_{ext}}\cdot d\Vec{l}
= -q\int^{\Vec{r_b}}_{\Vec{r_a}}\Vec{E}\cdot d\Vec{l}
= q(V(\Vec{r_b})-V(\Vec{r_a}))
\]

Para un sistema compuesto de $n$ cargas puntuales $q_i$ $(i \in [1,...,n])$, el trabajo necesario para armar está configuración (trayendo cada carga desde el infinito) viene dado por

\[W_{ext} = \sum_{i=1}^n\sum_{j>i}^n\frac{q_jq_i}{4\pi\epsilon_o r_{ij}}
= \frac{1}{2}\sum_{i=1}^n\sum_{j\neq i}^n\frac{q_jq_i}{4\pi\epsilon_o r_{ij}}
= \frac{1}{2}\sum_{i=1}^nq_iV(\Vec{r_i})\]

Donde $r_{ij} = \parallel\Vec{r_i}-\Vec{r_j}\parallel$ y $V(\Vec{r_i})$ es el potencial dado por todas las cargas puntuales existentes en el sistema exceptuando $i$.
\medbreak
Hay que notar que este no es el trabajo que realiza el sistema, sino un trabaja externo que se debe realizar para desplazar las cargas bajo efecto del campo eléctrico.
\medbreak
El trabajo efectuado por el agente externo para armar la configuración de cargas queda almacenado como energía potencial electrostática (almacenada en el sistema en su conjunto, no como una suma de cargas independientes).

\subsection{Energía Potencial}

La energía potencial $U$ asociada a una fuerza conservativa $\Vec{F}$ se define como

%Cambio: dr -> dl
\[U (\Vec{r_1})= -\int^{\Vec{r_1}}_{\Vec{r_o}}\Vec{F}\cdot d\Vec{l}\]

De forma que la energía potencial electrostática de una carga $q$ es

\[U_e = qV \implies \Delta U_e = q \Delta V\]

Para un sistema de $n$ cargas es

\[U_e = \frac{1}{2}\sum_{i=1}^nq_iV(\Vec{r_i})\]

Para una distribución de cargas continuas es

\[U_e = \frac{1}{2}\int V(\Vec{r})\, dq(\Vec{r})\]

El principio de superposición no aplica para $U_e$, esto puede entenderse como que en un sistema no sólo hay energía asociada a las cargas, sino que también a la interacción entre las mismas.

\subsubsection{Energía de un Campo Eléctrico}

La energía potencial electrostática también se puede entender como energía almacenada en un campo eléctrico

\[U_e = \frac{\epsilon_o}{2}\left(
\oint V\Vec{E}\cdot d\Vec{S}
+\int \parallel\Vec{E}\parallel^2\, dV
\right)\]

Integrando en todo el espacio

\[U_e = \frac{\epsilon_o}{2}\int \parallel\Vec{E}\parallel^2\, dV\]

$dV$ es diferencial de volumen.

\subsection{Teorema trabajo-energía cinética y su relación con la energía potencial}

El teorema trabajo-energía cinética nos dice que trabajo realizado por la fuerza neta, aplicada a una partícula de carga $q$, es igual al cambio de energía cinética
\[W = \Delta K\]

Y debido a que la fuerza externa es conservativa, se tiene que $\Delta K = \Delta U_e$, con lo que se puede concluir que
\[W_{ext} = \Delta U_e = q \Delta V\]
En el caso de la fuerza eléctrica, haciendo uso de la relación $\Vec{F}_{ext} = -\Vec{F}_{el\Acute{e}ctrica}$ se puede obtener que $ W_{elec} = -\Delta U_e =-q\Delta V$

\newpage