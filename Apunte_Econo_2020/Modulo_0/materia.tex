\section{Introducción a la economía}
\econ{https://www.core-econ.org/the-economy/book/es/text/01.html}{1}
\\


La economía es una ciencia social que permite estudiar las interacciones entre diversos agentes que toman decisiones y su entorno. En donde existen dos supuestos fundamentales

\begin{enumerate}[label=\arabic*)]
    \item Las asignaciones de los recursos afectan el bienestar de las personas
    \item Las personas reaccionan a incentivos buscando mejorar su bienestar
\end{enumerate}


El sistema económico capitalista se caracteriza por la presencia de propiedad privada, mercados y empresas. En este es muy importante la especialización del trabajo, ya que permite aumentar la productividad de este, generando más bienes y servicios, los cuales pueden ser intercambiados en los mercados, beneficiando a ambas partes de la transacción gracias al intercambio.

\subsection{Mercado}
Son instituciones económicas que permiten el intercambio de bienes y servicios entre diferentes agentes económico, a partir del cual se benefician mutuamente.

\subsection{Trabajo}
Componente fundamental de la economía para la producción de bienes y servicios.
\subsubsection{Especialización del trabajo}
Aumenta la productividad por que las personas se vuelven más hábiles realizando una cantidad más acotada de tareas específicas.

\subsection{Producto Interno Bruto (PIB)}
Es una medida utilizada para comparar los niveles de vida de cada país. El PIB es el valor total de todo lo que se produce en un periodo de tiempo determinado, por ejemplo, en un año. Por ende, el PIB per cápita corresponde al promedio anual de ingresos. Por este motivo, también se utiliza Ingreso Interno Bruto como término equivalente a PIB. \textit{[Fuente CORE]}

\begin{itemize}
    \item \textbf{PIB nominal}: es la suma de los valores estimados multiplicado por la cantidad de bienes generados que poseen ese valor. La suma se realiza cuando está todo medido en una unidad común.
    \item \textbf{PIB real}: se ajusta por la inflación de los precios en el tiempo, sirve para evaluar si la economía está creciendo o contrayéndose.
\end{itemize}

\subsubsection{Tasa de crecimiento del PIB}
Permite saber el cambio porcentual del pib entre un año y otro, este se calcula como
\[\text{tasa de crecimiento} = \frac{\text{cambio en el PIB}}{\text{nivel original del PIB}}\cdot100\]


\subsection{Paridad de poder de compra (PPC o PPP)}
La paridad de poder de compra (PPC) o \textit{Purchasing Power Parity (PPP)} en inglés es una corrección estadística que permite comparar el poder adquisitivo, es decir, lo que las personas pueden comprar en países que tienen diferentes monedas.

\subsection{Índice de precios del consumidor (IPC)}
Es un indicador económico que mide mes a mes la variación de los precios de una canasta de bienes y servicios representativa del consumo de los hogares urbanos del conjunto de las capitales regionales y sus zonas conurbadas dentro de las fronteras del país. \textit{[Fuente \href{www.ine.cl}{ine.cl}]}

Para obtener el precio en un periodo de tiempo $t$ hay que realizar el siguiente cálculo
\[\text{Precio}_t = \text{Precio}_{t-1}\cdot\frac{\text{IPC}_t}{\text{IPC}_{t-1}}\]
\newpage