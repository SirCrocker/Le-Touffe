\section{Fallas de Mercado y Externalidades}

% \subsection{Eficiencia y Externalidades}

\subsection{Equilibrio de mercado}

En un mercado competitivo:

\begin{itemize}
    \item Las firmas tienen costos similares, son numerosas y toman precios.
    \item Así la oferta total del mercado es la suma de de las ofertas de las firmas individuales.
    \item Así se obtiene la oferta total del mercado, y junto a la demanda de mercado se obtiene el punto de equilibrio.
\end{itemize}

Al tener la oferta total del mercado (la suma) se puede obtener el equilibrio de este mercado cuando la curva de oferta agregada se intersecta con la curva de demanda agregada.\\

Un mercado perfectamente competitivo genera un asignación de recursos \textbf{Pareto-eficiente}, es decir, no pueden existir mejoras sino es a costa de sacrificar el bienestar de otros.\\

Una condición para ello es que las transacciones no tengan \textbf{efectos externos} al mercado.


\subsection{Efectos externos o externalidades}
Corresponden a los beneficios o daños (o costos) que una decisión económica puede generar sobre otros actores o personas y no están contempladas en el contrato.\\

Cuando existen externalidades los costos y precios del mercado no consiguen capturar de manera adecuada los efectos de las decisiones económicas.\\

Esto da lugar a tres distinciones de los costos marginales
\begin{enumerate}[label=\roman*.-]
    \item Costo Marginal Privado ($C_p'(Q)$) [El analizado hasta ahora]
    \item Costo Marginal Externo ($C_e'(Q)$)
    \item Costo Marginal Social ($C_s'(Q)$)
\end{enumerate}

En presencia de externalidades, el costo social de una decisión difiere al costo privado recibido por el agente dando lugar a \textbf{fallas de mercado}.\\

\subsection{Costo Marginal P/E/S}

Para obtener el óptimo social hay que considerar el costo marginal social, este se define como el costo de producir una unidad adicional de un bien, teniendo en cuenta el costo para el productor como el costo incurrido por otros que son afectados por la producción de los bienes.\\

Donde

\[C_s'(Q) = C_p'(Q) + C_e'(Q)\]

Si el costo marginal social es mayor al costo marginal privado, entonces se dice que hay externalidades negativas (efecto externo negativo).\\

\textbf{CME}: Los costos impuestos por la firma, respecto a la toma de decisiones a la sociedad.

\subsection{Óptimo social}

Se debe asumir el costo externo, por lo que se quiere derivar el $Q^{\text{opt}}$ que maximice el excedente social. El óptimo social se alcanza en la cantidad donde se intersecta el costo marginal social y el precio de mercado.

\[E_s(Q) = PQ - C_s(Q) \implies C_s'(Q^{\text{opt}}) = P\]

\subsection{Formas de resolver las fallas de mercado}

\subsubsection{Negociación}
Consiste en una negociación entre ambas partes para llegar a un acuerdo de beneficio mutuo. (e.g. reducción de producción para aumentar el beneficio social)\\


\textbf{Teorema de Coase}
El economista Ronald Coase estudió este tipo de negociaciones para resolver problemas de externalidades. Según plantea, a partir de una negociación privada, se podría lograr una resultado Pareto-óptimo, pero bajo las siguientes condiciones necesarias:
\begin{itemize}
    \item Derechos de propiedad bien establecidos desde ambas partes, que permitan distinguir de manera clara quien ejerce una externalidad a quién.
    \item Costo de transacción bajos en la negociación, que no dificulten llegar a un acuerdo.
\end{itemize}

Este tipo de solución surge como alternativa privada, en comparación a las intervenciones estatales.

\subsubsection{Impuestos Pigouvianos}
Es un mecanismo de intervención estatal para resolver problemas de externalidades.\\

Este impuesto obliga a las firmas a internalizar los costos por los daños hechos a la sociedad.\\

Se le cobra a la firma el daño que están haciendo para pagárselo al Estado.

\subsubsection{Subsidios Pigouvianos}
También pueden existir externalidades positivas. Esto ocurre cuando una transacción genera beneficios externos, pero que no se traducen en una compensación para quién genera la externalidad y, en consecuencia, se produce una cantidad ineficientemente pequeña por no tener propiedad sobre dichos beneficios.

\subsubsection{Compensación/Indemnización}
El Estado obliga a la empresa a compensar a las personas que se vieron dañadas por esta.

\subsubsection{Cuotas}
No se permite producir más allá del óptimo social. Es una regulación de producción - tope en la cantidad.

\subsection{Fuentes de ineficiencias de mercado}
Las fallas del mercado se deben principalmente debido a contratos incompletos o los derechos de propiedad no funcionan adecuadamente. Esto significa que los contratos o derechos de propiedad no siempre pueden contemplar todas las situaciones que pueden ocurrir porque los agentes disponen información asimétrica.\\

Por lo tanto, en realidad es imposible utilizar contratos o derechos de propiedad para que todos los costos/beneficios sociales se incluyan en el proceso de toma de decisiones.\\


Tipos

\begin{enumerate}[label=\arabic*.]
    \item \textbf{Bienes públicos}, trae el problema de free-rider en un buen público puro.
    \item \textbf{Asimetría de información}
    

    Cuando una parte de la transacción sabe algo relevante que la otra no. Existen dos tipos
    
    \begin{enumerate}[label=\roman*.-]
        \item \textbf{Atributos ocultos}: Conduce un problema de selección adversa (características ocultas).
        \item \textbf{Acción oculta}: Conduce a un problema de riesgo moral o agente principal (acciones ocultas).
    \end{enumerate}

    \item \textbf{Precio mayor a costo marginal}: esto se debe a competencia limitada o reducción de costos a medios o largo plazo. El mercado falla porque la asignación no es Pareto-Eficiente.
    
\end{enumerate}


\newpage