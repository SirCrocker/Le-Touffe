\documentclass{article}
\usepackage[utf8]{inputenc}
\usepackage{amssymb}       % Librerías matemáticas
\usepackage{amsthm}        % Definición de teoremas
\usepackage{array}         % Nuevas características a las tablas
\usepackage{bigstrut}      % Líneas horizontales en tablas
\usepackage{bm}            % Caracteres en negrita en ecuaciones
\usepackage{booktabs}      % Permite manejar elementos visuales en tablas
\usepackage{caption}       % Leyendas
\usepackage{changepage}    % Condicionales para administrar páginas
\usepackage{chngcntr}      % Añade números a las leyendas
\usepackage{color}         % Colores
\usepackage{datetime}      % Fechas
\usepackage{enumitem}      % Listas con letras
\usepackage{floatpag}      % Maneja números de páginas
\usepackage{floatrow}      % Permite administrar posiciones en los caption
\usepackage{framed}        % Permite creación de recuadros
\usepackage{gensymb}       % Simbología común
\usepackage{graphicx}      % Propiedades extra para los gráficos
\usepackage{lipsum}        % Permite crear párrafos de prueba
\usepackage{listings}      % Permite añadir código fuente
\usepackage{longtable}     % Permite utilizar tablas en varias hojas
\usepackage{mathtools}     % Permite utilizar notaciones matemáticas
\usepackage{multicol}      % Múltiples columnas
\usepackage{needspace}     % Maneja los espacios en página
\usepackage{pdflscape}     % Modo página horizontal de página
\usepackage{pdfpages}      % Permite administrar páginas en pdf
\usepackage{physics}       % Paquete de matemáticas
% \usepackage{ragged2e}    % Redefine centering
\usepackage{rotating}      % Permite rotación de objetos
\usepackage{selinput}      % Compatibilidad con acentos
\usepackage{setspace}      % Cambia el espacio entre líneas
\usepackage{soul}          % Permite subrayar texto
\usepackage{subfig}        % Permite agrupar imágenes
\usepackage{textcomp}      % Simbología común
\usepackage{url}           % Permite añadir enlaces
\usepackage{wrapfig}       % Posición de imágenes
\usepackage{xspace}        % Adminsitra espacios en párrafos y líneas
\usepackage{amsmath}
%\usepackage{tikz}
\usepackage{mathdots}
\usepackage{yhmath}
\usepackage{cancel}
\usepackage{siunitx}
\usepackage{multirow}
\usepackage{tabularx}
\usepackage{xstring}
\usepackage[spanish,british]{babel}
\usepackage{hyperref}
\hypersetup{
    colorlinks=true,
    linkcolor=blue,
    filecolor=cyan,      
    urlcolor=magenta
}
\usepackage{titlesec}
%\usetikzlibrary{fadings}
%\usetikzlibrary{patterns}
%\usetikzlibrary{shadows.blur}
%\usetikzlibrary{shapes}


% ---- CONFIG ----

\titleformat{\section}
  {\Large\bfseries}{Módulo \thesection:}{10px}{}
 
\setcounter{section}{-1}

\newcommand{\econ}[2]{\textit{Capítulo correspondiente en CORE ECON: \href{#1}{Capítulo #2}}}

\newcommand{\fuente}[1]{\textit{[Fuente: #1]}}

\newcounter{problemas}[section]
\newcommand{\newpbm}{\stepcounter{problemas}\begin{flushleft}\boxed{\textbf{P.\thesection.\arabic{problemas}}} \hyperlink{S.\thesection.\arabic{problemas}}{Sol} \end{flushleft}
\hfill\\}  % Permite crear problema con ajuste de número y sección automático

\newcommand{\np}{\newpbm} % Alias por si prefieres

\newenvironment{solucion}[1]{
    \hfill\\
    \hypertarget{S.\thesection.#1}
    \text{}\boxed{\textbf{S.\thesection.#1}}
    \hfill\\
    }{}

% Cita o comillas " "
% Modo de uso: \cita{ texto }
\newcommand{\cita}[1]{``#1''}

% Superíndice text
% Modo de uso: \txtsi{ texto }{ objeto que irá en el índice }
\newcommand{\txtsi}[2]{$\text{#1}^{#2}$}

% Paréntesis grandes en comando 
% Modo de uso: \lados{ tipo de parentésis }{ Ecuación }
% Requisitos: \usepackage{xstring}
\newcommand{\lados}[2]{%
    \IfEqCase{#1}{%
        {(}{\left( #2 \right)}%
        {)}{\left( #2 \right)}%
        {()}{\left( #2 \right)}%
        {[}{\left[ #2 \right]}%
        {]}{\left[ #2 \right]}%
        {[]}{\left[ #2 \right]}%
        {\{}{\left\{ #2 \right\}}%
        {\}}{\left\{ #2 \right\}}%
        {\lfloor}{\left\lfloor #2 \right\rfloor}%
        {\rfloor}{\left\lfloor #2 \right\rfloor}%
    }[\PackageError{lados}{Opción no definida: #1}{}]%
}%

% ---- END CONFIG ----

\title{Economía IN2201}
\author{\textit{Le Touffe}}
\date{\monthname\text{ 2020}}


\begin{document}

\maketitle
\begin{abstract}
    \begin{center}
        
    
	Información pertinente al ramo de economía del 
	
	semestre primavera 2020. 
	
	Se recomienda vehemente leer el CORE en los módulos donde es posible, esto es un simple 'apunte'.
	\end{center}
\end{abstract}
\selectlanguage{spanish}
\newpage

\tableofcontents
\newpage

\section{Escasez}

\econ{https://www.core-econ.org/the-economy/book/es/text/03.html}{3}
\newline
\newline
Sea $Y = f(h)$ la función de producción, con $h$ el bien gastado en producir (puede ser horas, número de trabajadores, tierra utilizada, tiempo libre, etc). Se tendrá que:

\begin{itemize}
    \item \textbf{Producto medio}: $P_{Me} = \frac{f(h)}{h}$
    \newline 
    \newline Es la cantidad promedio de $Y$ conseguido con una cantidad $h$.
    \item \textbf{Producto marginal}: $P_{Mg} = \frac{df(h)}{dh}$
    \newline
    \newline Es el incremento en $Y$ que se obtiene por usar (o gastar) una cantidad mayor o menor de $h$. \textit{(Literalmente la derivada)}
    \newline También se puede ver como el retorno obtenido al usar distintas cantidades de $h$. \textit{(Cuanto Y cada h te da, que varía al aumentar o disminuir la cantidad de h)}
\end{itemize}

Donde la función de producción determina el conjunto factible de alternativas a elegir.

\subsection{Costo de oportunidad} 
\hyperlink{ejemplo-Costo de oportunidad}{Ejemplo}

Es el costo de la alternativa a la que renunciamos cuando tomamos una determinada decisión, incluyendo los beneficios que podríamos haber obtenido de haber escogido la opción alternativa. 

\subsection{Preferencias}

Para escoger una opción se depende de los beneficios frente a las otras opciones, que está intrínsecamente ligado a las preferencias de la persona.
\newline

Las preferencias de una persona se modelan a través de la función de \hyperlink{utilidad}{utilidad} $U$, que indica la utilidad percibida al consumir cierta cantidad dada del bien.

\subsubsection{Curvas de indiferencia}
Cuando las opciones indiferentes entre sí se marcan, se da origen a una curva de indiferencia. (En esta curva la utilidad siempre posee el mismo valor). 
\\

\textbf{Propiedades de las curvas de indiferencia:}

\begin{itemize}
    \item Las curvas de indiferencia tienen pendiente negativa que refleja las disyuntivas que implican una cierta renuncia: \textit{si hay dos combinaciones ante las que nos mostramos indiferentes, necesariamente eso implica que la que tenga más de un bien tendrá menos del otro bien.}
    \item Unas curvas de indiferencia más altas se corresponden a niveles de utilidad más altos: \textit{a medida que nos movemos hacia arriba y a la derecha en el diagrama, más lejos del origen, nos movemos a combinaciones que tienen más de ambos bienes.}
    \item Las curvas de indiferencia son suaves por lo general: \textit{cambios pequeños en la cantidad de bienes no causan grandes saltos en la utilidad.}
    \item Las curvas de indiferencia no se cruzan.
    \item A medida que te mueves hacia la derecha a lo largo de una curva de indiferencia, esta se vuelve más plana.
\end{itemize}

El último punto hace referencia a la \textit{utilidad decreciente de los bienes}, que establece que mientras más de un bien se tiene (o consume), menor es la utilidad adicional que se le otorga.

\subsection{Tasa marginal de sustitución (TMS)}
Es la cantidad de bien del eje vertical que se está dispuesto a dar por 'un' bien del eje horizontal, su valor viene dado por la tangente en un punto de la curva de indiferencia.
%Es cuanto se está dispuesto a dar por cada bien del eje vertical por algo del eje horizontal, 

En el caso de que la función de producción venga dada por $y = f(x)$ con $U = U(x, y)$ \[TMS = \abs{\frac{\partial U}{\partial x} / \frac{\partial U}{\partial y}} = -\frac{\partial U}{\partial x} / \frac{\partial U}{\partial y}\]

En palabras la fórmula anterior se puede escribir como
\[\text{TMS} = \abs{\frac{\text{utilidad marginal de \textit{x}}}{\text{utilidad marginal de \textit{y}}}}\]

\subsection{Optimización de la decisión}
Siempre se busca maximizar la utilidad sujeta al conjunto factible, este óptimo se encuentra en la \hyperlink{frontera-factible}{frontera del conjunto factible} y/o en la curva de indiferencia de mayor utilidad ('arriba a la derecha').

En el óptimo se cumple que la curva de indiferencia es tangente a la frontera factible y \[TMS = -P_{Mg}\] (Esto es equivalente a que las tangentes de la frontera factible y el producto marginal sean iguales)

\subsubsection{Tasa marginal de transformación (TMT)}
Cantidad de algún bien que debe sacrificarse para adquirir una unidad adicional de otro bien. En cualquier punto, es la pendiente de la frontera factible. Es equivalente a $-P_{Mg} \implies TMT = TMS$ también es válido para obtener óptimos.

\subsection{Efecto neto}
Es la suma del efecto ingreso y el efecto neto.

\subsubsection{Efecto ingreso}
Son los cambios en el consumo debido a una expansión en el conjunto factible, pero manteniendo el mismo trade-off entre cantidad consumida y tiempo libre.

Se deriva de la perdida (o ganancia) del poder adquisitivo, es ajeno a la variación relativa de los precios.
\\

\textit{Se desplaza la frontera factible verticalmente}, puede causar que la cantidad de bien dispuesto a invertir disminuya.\\


\subsubsection{Efecto sustitución}
Es la variación del consumo de bien cuyo precio varía. Se deriva exclusivamente de la variación de los precios, y es por lo tanto, ajeno al cambio de poder adquisitivo.
\\

\textit{Se da únicamente por cambios en el precio o el costo de oportunidad, dado el nuevo nivel de utilidad}, puede causar que la cantidad de bien dispuesto a invertir aumente.

\subsubsection{Como calcular los efectos}
\href{https://www.core-econ.org/the-economy/book/es/text/leibniz-03-07-01.html}{Según CORE.\\}


En un problema se obtienen valores de \textit{x} e \textit{y} en el óptimo, sean estos $x_0$ e $y_0$.

Cuando hay un cambio en el problema/situación planteada para calcular el efecto ingreso hay que, obtener la nueva utilidad óptima y los valores de las variables correspondientes ($x_1$ e $y_1$), e intentar conseguir un ingreso que causaría que la utilidad óptima fuera igual a la del problema modificado ($x_2$ e $y_2$). Luego la diferencia de ''x'' ($x_2 - x_0$) es el efecto ingreso. Si es que la diferencia no es igual a la diferencia total ($x_2 - x_0 \neq x_1 - x_0$), entonces lo faltante es el efecto sustitución. 

\subsection{Problemas de optimización}
En estos problemas se intenta optimizar una función objetivo, para encontrar su mínimo o máximo. Esta está sujeta a restricciones, que ponen límites a las variables presentes en el problema. 
\\

Un ejemplo es maximizar la función utilidad $U(x_1,x_2)$ donde la restricción es que $g(x_1,x_2) \leq w$, con $w$ un valor. Aquí las variables son $x_1$ y $x_2$.
\newpage
\section{Escasez}

\econ{https://www.core-econ.org/the-economy/book/es/text/03.html}{3}
\newline
\newline
Sea $Y = f(h)$ la función de producción, con $h$ el bien gastado en producir (puede ser horas, número de trabajadores, tierra utilizada, tiempo libre, etc). Se tendrá que:

\begin{itemize}
    \item \textbf{Producto medio}: $P_{Me} = \frac{f(h)}{h}$
    \newline 
    \newline Es la cantidad promedio de $Y$ conseguido con una cantidad $h$.
    \item \textbf{Producto marginal}: $P_{Mg} = \frac{df(h)}{dh}$
    \newline
    \newline Es el incremento en $Y$ que se obtiene por usar (o gastar) una cantidad mayor o menor de $h$. \textit{(Literalmente la derivada)}
    \newline También se puede ver como el retorno obtenido al usar distintas cantidades de $h$. \textit{(Cuanto Y cada h te da, que varía al aumentar o disminuir la cantidad de h)}
\end{itemize}

Donde la función de producción determina el conjunto factible de alternativas a elegir.

\subsection{Costo de oportunidad} 
\hyperlink{ejemplo-Costo de oportunidad}{Ejemplo}

Es el costo de la alternativa a la que renunciamos cuando tomamos una determinada decisión, incluyendo los beneficios que podríamos haber obtenido de haber escogido la opción alternativa. 

\subsection{Preferencias}

Para escoger una opción se depende de los beneficios frente a las otras opciones, que está intrínsecamente ligado a las preferencias de la persona.
\newline

Las preferencias de una persona se modelan a través de la función de \hyperlink{utilidad}{utilidad} $U$, que indica la utilidad percibida al consumir cierta cantidad dada del bien.

\subsubsection{Curvas de indiferencia}
Cuando las opciones indiferentes entre sí se marcan, se da origen a una curva de indiferencia. (En esta curva la utilidad siempre posee el mismo valor). 
\\

\textbf{Propiedades de las curvas de indiferencia:}

\begin{itemize}
    \item Las curvas de indiferencia tienen pendiente negativa que refleja las disyuntivas que implican una cierta renuncia: \textit{si hay dos combinaciones ante las que nos mostramos indiferentes, necesariamente eso implica que la que tenga más de un bien tendrá menos del otro bien.}
    \item Unas curvas de indiferencia más altas se corresponden a niveles de utilidad más altos: \textit{a medida que nos movemos hacia arriba y a la derecha en el diagrama, más lejos del origen, nos movemos a combinaciones que tienen más de ambos bienes.}
    \item Las curvas de indiferencia son suaves por lo general: \textit{cambios pequeños en la cantidad de bienes no causan grandes saltos en la utilidad.}
    \item Las curvas de indiferencia no se cruzan.
    \item A medida que te mueves hacia la derecha a lo largo de una curva de indiferencia, esta se vuelve más plana.
\end{itemize}

El último punto hace referencia a la \textit{utilidad decreciente de los bienes}, que establece que mientras más de un bien se tiene (o consume), menor es la utilidad adicional que se le otorga.

\subsection{Tasa marginal de sustitución (TMS)}
Es la cantidad de bien del eje vertical que se está dispuesto a dar por 'un' bien del eje horizontal, su valor viene dado por la tangente en un punto de la curva de indiferencia.
%Es cuanto se está dispuesto a dar por cada bien del eje vertical por algo del eje horizontal, 

En el caso de que la función de producción venga dada por $y = f(x)$ con $U = U(x, y)$ \[TMS = \abs{\frac{\partial U}{\partial x} / \frac{\partial U}{\partial y}} = -\frac{\partial U}{\partial x} / \frac{\partial U}{\partial y}\]

En palabras la fórmula anterior se puede escribir como
\[\text{TMS} = \abs{\frac{\text{utilidad marginal de \textit{x}}}{\text{utilidad marginal de \textit{y}}}}\]

\subsection{Optimización de la decisión}
Siempre se busca maximizar la utilidad sujeta al conjunto factible, este óptimo se encuentra en la \hyperlink{frontera-factible}{frontera del conjunto factible} y/o en la curva de indiferencia de mayor utilidad ('arriba a la derecha').

En el óptimo se cumple que la curva de indiferencia es tangente a la frontera factible y \[TMS = -P_{Mg}\] (Esto es equivalente a que las tangentes de la frontera factible y el producto marginal sean iguales)

\subsubsection{Tasa marginal de transformación (TMT)}
Cantidad de algún bien que debe sacrificarse para adquirir una unidad adicional de otro bien. En cualquier punto, es la pendiente de la frontera factible. Es equivalente a $-P_{Mg} \implies TMT = TMS$ también es válido para obtener óptimos.

\subsection{Efecto neto}
Es la suma del efecto ingreso y el efecto neto.

\subsubsection{Efecto ingreso}
Son los cambios en el consumo debido a una expansión en el conjunto factible, pero manteniendo el mismo trade-off entre cantidad consumida y tiempo libre.

Se deriva de la perdida (o ganancia) del poder adquisitivo, es ajeno a la variación relativa de los precios.
\\

\textit{Se desplaza la frontera factible verticalmente}, puede causar que la cantidad de bien dispuesto a invertir disminuya.\\


\subsubsection{Efecto sustitución}
Es la variación del consumo de bien cuyo precio varía. Se deriva exclusivamente de la variación de los precios, y es por lo tanto, ajeno al cambio de poder adquisitivo.
\\

\textit{Se da únicamente por cambios en el precio o el costo de oportunidad, dado el nuevo nivel de utilidad}, puede causar que la cantidad de bien dispuesto a invertir aumente.

\subsubsection{Como calcular los efectos}
\href{https://www.core-econ.org/the-economy/book/es/text/leibniz-03-07-01.html}{Según CORE.\\}


En un problema se obtienen valores de \textit{x} e \textit{y} en el óptimo, sean estos $x_0$ e $y_0$.

Cuando hay un cambio en el problema/situación planteada para calcular el efecto ingreso hay que, obtener la nueva utilidad óptima y los valores de las variables correspondientes ($x_1$ e $y_1$), e intentar conseguir un ingreso que causaría que la utilidad óptima fuera igual a la del problema modificado ($x_2$ e $y_2$). Luego la diferencia de ''x'' ($x_2 - x_0$) es el efecto ingreso. Si es que la diferencia no es igual a la diferencia total ($x_2 - x_0 \neq x_1 - x_0$), entonces lo faltante es el efecto sustitución. 

\subsection{Problemas de optimización}
En estos problemas se intenta optimizar una función objetivo, para encontrar su mínimo o máximo. Esta está sujeta a restricciones, que ponen límites a las variables presentes en el problema. 
\\

Un ejemplo es maximizar la función utilidad $U(x_1,x_2)$ donde la restricción es que $g(x_1,x_2) \leq w$, con $w$ un valor. Aquí las variables son $x_1$ y $x_2$.
\newpage
\section{Escasez}

\econ{https://www.core-econ.org/the-economy/book/es/text/03.html}{3}
\newline
\newline
Sea $Y = f(h)$ la función de producción, con $h$ el bien gastado en producir (puede ser horas, número de trabajadores, tierra utilizada, tiempo libre, etc). Se tendrá que:

\begin{itemize}
    \item \textbf{Producto medio}: $P_{Me} = \frac{f(h)}{h}$
    \newline 
    \newline Es la cantidad promedio de $Y$ conseguido con una cantidad $h$.
    \item \textbf{Producto marginal}: $P_{Mg} = \frac{df(h)}{dh}$
    \newline
    \newline Es el incremento en $Y$ que se obtiene por usar (o gastar) una cantidad mayor o menor de $h$. \textit{(Literalmente la derivada)}
    \newline También se puede ver como el retorno obtenido al usar distintas cantidades de $h$. \textit{(Cuanto Y cada h te da, que varía al aumentar o disminuir la cantidad de h)}
\end{itemize}

Donde la función de producción determina el conjunto factible de alternativas a elegir.

\subsection{Costo de oportunidad} 
\hyperlink{ejemplo-Costo de oportunidad}{Ejemplo}

Es el costo de la alternativa a la que renunciamos cuando tomamos una determinada decisión, incluyendo los beneficios que podríamos haber obtenido de haber escogido la opción alternativa. 

\subsection{Preferencias}

Para escoger una opción se depende de los beneficios frente a las otras opciones, que está intrínsecamente ligado a las preferencias de la persona.
\newline

Las preferencias de una persona se modelan a través de la función de \hyperlink{utilidad}{utilidad} $U$, que indica la utilidad percibida al consumir cierta cantidad dada del bien.

\subsubsection{Curvas de indiferencia}
Cuando las opciones indiferentes entre sí se marcan, se da origen a una curva de indiferencia. (En esta curva la utilidad siempre posee el mismo valor). 
\\

\textbf{Propiedades de las curvas de indiferencia:}

\begin{itemize}
    \item Las curvas de indiferencia tienen pendiente negativa que refleja las disyuntivas que implican una cierta renuncia: \textit{si hay dos combinaciones ante las que nos mostramos indiferentes, necesariamente eso implica que la que tenga más de un bien tendrá menos del otro bien.}
    \item Unas curvas de indiferencia más altas se corresponden a niveles de utilidad más altos: \textit{a medida que nos movemos hacia arriba y a la derecha en el diagrama, más lejos del origen, nos movemos a combinaciones que tienen más de ambos bienes.}
    \item Las curvas de indiferencia son suaves por lo general: \textit{cambios pequeños en la cantidad de bienes no causan grandes saltos en la utilidad.}
    \item Las curvas de indiferencia no se cruzan.
    \item A medida que te mueves hacia la derecha a lo largo de una curva de indiferencia, esta se vuelve más plana.
\end{itemize}

El último punto hace referencia a la \textit{utilidad decreciente de los bienes}, que establece que mientras más de un bien se tiene (o consume), menor es la utilidad adicional que se le otorga.

\subsection{Tasa marginal de sustitución (TMS)}
Es la cantidad de bien del eje vertical que se está dispuesto a dar por 'un' bien del eje horizontal, su valor viene dado por la tangente en un punto de la curva de indiferencia.
%Es cuanto se está dispuesto a dar por cada bien del eje vertical por algo del eje horizontal, 

En el caso de que la función de producción venga dada por $y = f(x)$ con $U = U(x, y)$ \[TMS = \abs{\frac{\partial U}{\partial x} / \frac{\partial U}{\partial y}} = -\frac{\partial U}{\partial x} / \frac{\partial U}{\partial y}\]

En palabras la fórmula anterior se puede escribir como
\[\text{TMS} = \abs{\frac{\text{utilidad marginal de \textit{x}}}{\text{utilidad marginal de \textit{y}}}}\]

\subsection{Optimización de la decisión}
Siempre se busca maximizar la utilidad sujeta al conjunto factible, este óptimo se encuentra en la \hyperlink{frontera-factible}{frontera del conjunto factible} y/o en la curva de indiferencia de mayor utilidad ('arriba a la derecha').

En el óptimo se cumple que la curva de indiferencia es tangente a la frontera factible y \[TMS = -P_{Mg}\] (Esto es equivalente a que las tangentes de la frontera factible y el producto marginal sean iguales)

\subsubsection{Tasa marginal de transformación (TMT)}
Cantidad de algún bien que debe sacrificarse para adquirir una unidad adicional de otro bien. En cualquier punto, es la pendiente de la frontera factible. Es equivalente a $-P_{Mg} \implies TMT = TMS$ también es válido para obtener óptimos.

\subsection{Efecto neto}
Es la suma del efecto ingreso y el efecto neto.

\subsubsection{Efecto ingreso}
Son los cambios en el consumo debido a una expansión en el conjunto factible, pero manteniendo el mismo trade-off entre cantidad consumida y tiempo libre.

Se deriva de la perdida (o ganancia) del poder adquisitivo, es ajeno a la variación relativa de los precios.
\\

\textit{Se desplaza la frontera factible verticalmente}, puede causar que la cantidad de bien dispuesto a invertir disminuya.\\


\subsubsection{Efecto sustitución}
Es la variación del consumo de bien cuyo precio varía. Se deriva exclusivamente de la variación de los precios, y es por lo tanto, ajeno al cambio de poder adquisitivo.
\\

\textit{Se da únicamente por cambios en el precio o el costo de oportunidad, dado el nuevo nivel de utilidad}, puede causar que la cantidad de bien dispuesto a invertir aumente.

\subsubsection{Como calcular los efectos}
\href{https://www.core-econ.org/the-economy/book/es/text/leibniz-03-07-01.html}{Según CORE.\\}


En un problema se obtienen valores de \textit{x} e \textit{y} en el óptimo, sean estos $x_0$ e $y_0$.

Cuando hay un cambio en el problema/situación planteada para calcular el efecto ingreso hay que, obtener la nueva utilidad óptima y los valores de las variables correspondientes ($x_1$ e $y_1$), e intentar conseguir un ingreso que causaría que la utilidad óptima fuera igual a la del problema modificado ($x_2$ e $y_2$). Luego la diferencia de ''x'' ($x_2 - x_0$) es el efecto ingreso. Si es que la diferencia no es igual a la diferencia total ($x_2 - x_0 \neq x_1 - x_0$), entonces lo faltante es el efecto sustitución. 

\subsection{Problemas de optimización}
En estos problemas se intenta optimizar una función objetivo, para encontrar su mínimo o máximo. Esta está sujeta a restricciones, que ponen límites a las variables presentes en el problema. 
\\

Un ejemplo es maximizar la función utilidad $U(x_1,x_2)$ donde la restricción es que $g(x_1,x_2) \leq w$, con $w$ un valor. Aquí las variables son $x_1$ y $x_2$.
\newpage
\section{Escasez}

\econ{https://www.core-econ.org/the-economy/book/es/text/03.html}{3}
\newline
\newline
Sea $Y = f(h)$ la función de producción, con $h$ el bien gastado en producir (puede ser horas, número de trabajadores, tierra utilizada, tiempo libre, etc). Se tendrá que:

\begin{itemize}
    \item \textbf{Producto medio}: $P_{Me} = \frac{f(h)}{h}$
    \newline 
    \newline Es la cantidad promedio de $Y$ conseguido con una cantidad $h$.
    \item \textbf{Producto marginal}: $P_{Mg} = \frac{df(h)}{dh}$
    \newline
    \newline Es el incremento en $Y$ que se obtiene por usar (o gastar) una cantidad mayor o menor de $h$. \textit{(Literalmente la derivada)}
    \newline También se puede ver como el retorno obtenido al usar distintas cantidades de $h$. \textit{(Cuanto Y cada h te da, que varía al aumentar o disminuir la cantidad de h)}
\end{itemize}

Donde la función de producción determina el conjunto factible de alternativas a elegir.

\subsection{Costo de oportunidad} 
\hyperlink{ejemplo-Costo de oportunidad}{Ejemplo}

Es el costo de la alternativa a la que renunciamos cuando tomamos una determinada decisión, incluyendo los beneficios que podríamos haber obtenido de haber escogido la opción alternativa. 

\subsection{Preferencias}

Para escoger una opción se depende de los beneficios frente a las otras opciones, que está intrínsecamente ligado a las preferencias de la persona.
\newline

Las preferencias de una persona se modelan a través de la función de \hyperlink{utilidad}{utilidad} $U$, que indica la utilidad percibida al consumir cierta cantidad dada del bien.

\subsubsection{Curvas de indiferencia}
Cuando las opciones indiferentes entre sí se marcan, se da origen a una curva de indiferencia. (En esta curva la utilidad siempre posee el mismo valor). 
\\

\textbf{Propiedades de las curvas de indiferencia:}

\begin{itemize}
    \item Las curvas de indiferencia tienen pendiente negativa que refleja las disyuntivas que implican una cierta renuncia: \textit{si hay dos combinaciones ante las que nos mostramos indiferentes, necesariamente eso implica que la que tenga más de un bien tendrá menos del otro bien.}
    \item Unas curvas de indiferencia más altas se corresponden a niveles de utilidad más altos: \textit{a medida que nos movemos hacia arriba y a la derecha en el diagrama, más lejos del origen, nos movemos a combinaciones que tienen más de ambos bienes.}
    \item Las curvas de indiferencia son suaves por lo general: \textit{cambios pequeños en la cantidad de bienes no causan grandes saltos en la utilidad.}
    \item Las curvas de indiferencia no se cruzan.
    \item A medida que te mueves hacia la derecha a lo largo de una curva de indiferencia, esta se vuelve más plana.
\end{itemize}

El último punto hace referencia a la \textit{utilidad decreciente de los bienes}, que establece que mientras más de un bien se tiene (o consume), menor es la utilidad adicional que se le otorga.

\subsection{Tasa marginal de sustitución (TMS)}
Es la cantidad de bien del eje vertical que se está dispuesto a dar por 'un' bien del eje horizontal, su valor viene dado por la tangente en un punto de la curva de indiferencia.
%Es cuanto se está dispuesto a dar por cada bien del eje vertical por algo del eje horizontal, 

En el caso de que la función de producción venga dada por $y = f(x)$ con $U = U(x, y)$ \[TMS = \abs{\frac{\partial U}{\partial x} / \frac{\partial U}{\partial y}} = -\frac{\partial U}{\partial x} / \frac{\partial U}{\partial y}\]

En palabras la fórmula anterior se puede escribir como
\[\text{TMS} = \abs{\frac{\text{utilidad marginal de \textit{x}}}{\text{utilidad marginal de \textit{y}}}}\]

\subsection{Optimización de la decisión}
Siempre se busca maximizar la utilidad sujeta al conjunto factible, este óptimo se encuentra en la \hyperlink{frontera-factible}{frontera del conjunto factible} y/o en la curva de indiferencia de mayor utilidad ('arriba a la derecha').

En el óptimo se cumple que la curva de indiferencia es tangente a la frontera factible y \[TMS = -P_{Mg}\] (Esto es equivalente a que las tangentes de la frontera factible y el producto marginal sean iguales)

\subsubsection{Tasa marginal de transformación (TMT)}
Cantidad de algún bien que debe sacrificarse para adquirir una unidad adicional de otro bien. En cualquier punto, es la pendiente de la frontera factible. Es equivalente a $-P_{Mg} \implies TMT = TMS$ también es válido para obtener óptimos.

\subsection{Efecto neto}
Es la suma del efecto ingreso y el efecto neto.

\subsubsection{Efecto ingreso}
Son los cambios en el consumo debido a una expansión en el conjunto factible, pero manteniendo el mismo trade-off entre cantidad consumida y tiempo libre.

Se deriva de la perdida (o ganancia) del poder adquisitivo, es ajeno a la variación relativa de los precios.
\\

\textit{Se desplaza la frontera factible verticalmente}, puede causar que la cantidad de bien dispuesto a invertir disminuya.\\


\subsubsection{Efecto sustitución}
Es la variación del consumo de bien cuyo precio varía. Se deriva exclusivamente de la variación de los precios, y es por lo tanto, ajeno al cambio de poder adquisitivo.
\\

\textit{Se da únicamente por cambios en el precio o el costo de oportunidad, dado el nuevo nivel de utilidad}, puede causar que la cantidad de bien dispuesto a invertir aumente.

\subsubsection{Como calcular los efectos}
\href{https://www.core-econ.org/the-economy/book/es/text/leibniz-03-07-01.html}{Según CORE.\\}


En un problema se obtienen valores de \textit{x} e \textit{y} en el óptimo, sean estos $x_0$ e $y_0$.

Cuando hay un cambio en el problema/situación planteada para calcular el efecto ingreso hay que, obtener la nueva utilidad óptima y los valores de las variables correspondientes ($x_1$ e $y_1$), e intentar conseguir un ingreso que causaría que la utilidad óptima fuera igual a la del problema modificado ($x_2$ e $y_2$). Luego la diferencia de ''x'' ($x_2 - x_0$) es el efecto ingreso. Si es que la diferencia no es igual a la diferencia total ($x_2 - x_0 \neq x_1 - x_0$), entonces lo faltante es el efecto sustitución. 

\subsection{Problemas de optimización}
En estos problemas se intenta optimizar una función objetivo, para encontrar su mínimo o máximo. Esta está sujeta a restricciones, que ponen límites a las variables presentes en el problema. 
\\

Un ejemplo es maximizar la función utilidad $U(x_1,x_2)$ donde la restricción es que $g(x_1,x_2) \leq w$, con $w$ un valor. Aquí las variables son $x_1$ y $x_2$.
\newpage
\section{Escasez}

\econ{https://www.core-econ.org/the-economy/book/es/text/03.html}{3}
\newline
\newline
Sea $Y = f(h)$ la función de producción, con $h$ el bien gastado en producir (puede ser horas, número de trabajadores, tierra utilizada, tiempo libre, etc). Se tendrá que:

\begin{itemize}
    \item \textbf{Producto medio}: $P_{Me} = \frac{f(h)}{h}$
    \newline 
    \newline Es la cantidad promedio de $Y$ conseguido con una cantidad $h$.
    \item \textbf{Producto marginal}: $P_{Mg} = \frac{df(h)}{dh}$
    \newline
    \newline Es el incremento en $Y$ que se obtiene por usar (o gastar) una cantidad mayor o menor de $h$. \textit{(Literalmente la derivada)}
    \newline También se puede ver como el retorno obtenido al usar distintas cantidades de $h$. \textit{(Cuanto Y cada h te da, que varía al aumentar o disminuir la cantidad de h)}
\end{itemize}

Donde la función de producción determina el conjunto factible de alternativas a elegir.

\subsection{Costo de oportunidad} 
\hyperlink{ejemplo-Costo de oportunidad}{Ejemplo}

Es el costo de la alternativa a la que renunciamos cuando tomamos una determinada decisión, incluyendo los beneficios que podríamos haber obtenido de haber escogido la opción alternativa. 

\subsection{Preferencias}

Para escoger una opción se depende de los beneficios frente a las otras opciones, que está intrínsecamente ligado a las preferencias de la persona.
\newline

Las preferencias de una persona se modelan a través de la función de \hyperlink{utilidad}{utilidad} $U$, que indica la utilidad percibida al consumir cierta cantidad dada del bien.

\subsubsection{Curvas de indiferencia}
Cuando las opciones indiferentes entre sí se marcan, se da origen a una curva de indiferencia. (En esta curva la utilidad siempre posee el mismo valor). 
\\

\textbf{Propiedades de las curvas de indiferencia:}

\begin{itemize}
    \item Las curvas de indiferencia tienen pendiente negativa que refleja las disyuntivas que implican una cierta renuncia: \textit{si hay dos combinaciones ante las que nos mostramos indiferentes, necesariamente eso implica que la que tenga más de un bien tendrá menos del otro bien.}
    \item Unas curvas de indiferencia más altas se corresponden a niveles de utilidad más altos: \textit{a medida que nos movemos hacia arriba y a la derecha en el diagrama, más lejos del origen, nos movemos a combinaciones que tienen más de ambos bienes.}
    \item Las curvas de indiferencia son suaves por lo general: \textit{cambios pequeños en la cantidad de bienes no causan grandes saltos en la utilidad.}
    \item Las curvas de indiferencia no se cruzan.
    \item A medida que te mueves hacia la derecha a lo largo de una curva de indiferencia, esta se vuelve más plana.
\end{itemize}

El último punto hace referencia a la \textit{utilidad decreciente de los bienes}, que establece que mientras más de un bien se tiene (o consume), menor es la utilidad adicional que se le otorga.

\subsection{Tasa marginal de sustitución (TMS)}
Es la cantidad de bien del eje vertical que se está dispuesto a dar por 'un' bien del eje horizontal, su valor viene dado por la tangente en un punto de la curva de indiferencia.
%Es cuanto se está dispuesto a dar por cada bien del eje vertical por algo del eje horizontal, 

En el caso de que la función de producción venga dada por $y = f(x)$ con $U = U(x, y)$ \[TMS = \abs{\frac{\partial U}{\partial x} / \frac{\partial U}{\partial y}} = -\frac{\partial U}{\partial x} / \frac{\partial U}{\partial y}\]

En palabras la fórmula anterior se puede escribir como
\[\text{TMS} = \abs{\frac{\text{utilidad marginal de \textit{x}}}{\text{utilidad marginal de \textit{y}}}}\]

\subsection{Optimización de la decisión}
Siempre se busca maximizar la utilidad sujeta al conjunto factible, este óptimo se encuentra en la \hyperlink{frontera-factible}{frontera del conjunto factible} y/o en la curva de indiferencia de mayor utilidad ('arriba a la derecha').

En el óptimo se cumple que la curva de indiferencia es tangente a la frontera factible y \[TMS = -P_{Mg}\] (Esto es equivalente a que las tangentes de la frontera factible y el producto marginal sean iguales)

\subsubsection{Tasa marginal de transformación (TMT)}
Cantidad de algún bien que debe sacrificarse para adquirir una unidad adicional de otro bien. En cualquier punto, es la pendiente de la frontera factible. Es equivalente a $-P_{Mg} \implies TMT = TMS$ también es válido para obtener óptimos.

\subsection{Efecto neto}
Es la suma del efecto ingreso y el efecto neto.

\subsubsection{Efecto ingreso}
Son los cambios en el consumo debido a una expansión en el conjunto factible, pero manteniendo el mismo trade-off entre cantidad consumida y tiempo libre.

Se deriva de la perdida (o ganancia) del poder adquisitivo, es ajeno a la variación relativa de los precios.
\\

\textit{Se desplaza la frontera factible verticalmente}, puede causar que la cantidad de bien dispuesto a invertir disminuya.\\


\subsubsection{Efecto sustitución}
Es la variación del consumo de bien cuyo precio varía. Se deriva exclusivamente de la variación de los precios, y es por lo tanto, ajeno al cambio de poder adquisitivo.
\\

\textit{Se da únicamente por cambios en el precio o el costo de oportunidad, dado el nuevo nivel de utilidad}, puede causar que la cantidad de bien dispuesto a invertir aumente.

\subsubsection{Como calcular los efectos}
\href{https://www.core-econ.org/the-economy/book/es/text/leibniz-03-07-01.html}{Según CORE.\\}


En un problema se obtienen valores de \textit{x} e \textit{y} en el óptimo, sean estos $x_0$ e $y_0$.

Cuando hay un cambio en el problema/situación planteada para calcular el efecto ingreso hay que, obtener la nueva utilidad óptima y los valores de las variables correspondientes ($x_1$ e $y_1$), e intentar conseguir un ingreso que causaría que la utilidad óptima fuera igual a la del problema modificado ($x_2$ e $y_2$). Luego la diferencia de ''x'' ($x_2 - x_0$) es el efecto ingreso. Si es que la diferencia no es igual a la diferencia total ($x_2 - x_0 \neq x_1 - x_0$), entonces lo faltante es el efecto sustitución. 

\subsection{Problemas de optimización}
En estos problemas se intenta optimizar una función objetivo, para encontrar su mínimo o máximo. Esta está sujeta a restricciones, que ponen límites a las variables presentes en el problema. 
\\

Un ejemplo es maximizar la función utilidad $U(x_1,x_2)$ donde la restricción es que $g(x_1,x_2) \leq w$, con $w$ un valor. Aquí las variables son $x_1$ y $x_2$.
\newpage
\section{Escasez}

\econ{https://www.core-econ.org/the-economy/book/es/text/03.html}{3}
\newline
\newline
Sea $Y = f(h)$ la función de producción, con $h$ el bien gastado en producir (puede ser horas, número de trabajadores, tierra utilizada, tiempo libre, etc). Se tendrá que:

\begin{itemize}
    \item \textbf{Producto medio}: $P_{Me} = \frac{f(h)}{h}$
    \newline 
    \newline Es la cantidad promedio de $Y$ conseguido con una cantidad $h$.
    \item \textbf{Producto marginal}: $P_{Mg} = \frac{df(h)}{dh}$
    \newline
    \newline Es el incremento en $Y$ que se obtiene por usar (o gastar) una cantidad mayor o menor de $h$. \textit{(Literalmente la derivada)}
    \newline También se puede ver como el retorno obtenido al usar distintas cantidades de $h$. \textit{(Cuanto Y cada h te da, que varía al aumentar o disminuir la cantidad de h)}
\end{itemize}

Donde la función de producción determina el conjunto factible de alternativas a elegir.

\subsection{Costo de oportunidad} 
\hyperlink{ejemplo-Costo de oportunidad}{Ejemplo}

Es el costo de la alternativa a la que renunciamos cuando tomamos una determinada decisión, incluyendo los beneficios que podríamos haber obtenido de haber escogido la opción alternativa. 

\subsection{Preferencias}

Para escoger una opción se depende de los beneficios frente a las otras opciones, que está intrínsecamente ligado a las preferencias de la persona.
\newline

Las preferencias de una persona se modelan a través de la función de \hyperlink{utilidad}{utilidad} $U$, que indica la utilidad percibida al consumir cierta cantidad dada del bien.

\subsubsection{Curvas de indiferencia}
Cuando las opciones indiferentes entre sí se marcan, se da origen a una curva de indiferencia. (En esta curva la utilidad siempre posee el mismo valor). 
\\

\textbf{Propiedades de las curvas de indiferencia:}

\begin{itemize}
    \item Las curvas de indiferencia tienen pendiente negativa que refleja las disyuntivas que implican una cierta renuncia: \textit{si hay dos combinaciones ante las que nos mostramos indiferentes, necesariamente eso implica que la que tenga más de un bien tendrá menos del otro bien.}
    \item Unas curvas de indiferencia más altas se corresponden a niveles de utilidad más altos: \textit{a medida que nos movemos hacia arriba y a la derecha en el diagrama, más lejos del origen, nos movemos a combinaciones que tienen más de ambos bienes.}
    \item Las curvas de indiferencia son suaves por lo general: \textit{cambios pequeños en la cantidad de bienes no causan grandes saltos en la utilidad.}
    \item Las curvas de indiferencia no se cruzan.
    \item A medida que te mueves hacia la derecha a lo largo de una curva de indiferencia, esta se vuelve más plana.
\end{itemize}

El último punto hace referencia a la \textit{utilidad decreciente de los bienes}, que establece que mientras más de un bien se tiene (o consume), menor es la utilidad adicional que se le otorga.

\subsection{Tasa marginal de sustitución (TMS)}
Es la cantidad de bien del eje vertical que se está dispuesto a dar por 'un' bien del eje horizontal, su valor viene dado por la tangente en un punto de la curva de indiferencia.
%Es cuanto se está dispuesto a dar por cada bien del eje vertical por algo del eje horizontal, 

En el caso de que la función de producción venga dada por $y = f(x)$ con $U = U(x, y)$ \[TMS = \abs{\frac{\partial U}{\partial x} / \frac{\partial U}{\partial y}} = -\frac{\partial U}{\partial x} / \frac{\partial U}{\partial y}\]

En palabras la fórmula anterior se puede escribir como
\[\text{TMS} = \abs{\frac{\text{utilidad marginal de \textit{x}}}{\text{utilidad marginal de \textit{y}}}}\]

\subsection{Optimización de la decisión}
Siempre se busca maximizar la utilidad sujeta al conjunto factible, este óptimo se encuentra en la \hyperlink{frontera-factible}{frontera del conjunto factible} y/o en la curva de indiferencia de mayor utilidad ('arriba a la derecha').

En el óptimo se cumple que la curva de indiferencia es tangente a la frontera factible y \[TMS = -P_{Mg}\] (Esto es equivalente a que las tangentes de la frontera factible y el producto marginal sean iguales)

\subsubsection{Tasa marginal de transformación (TMT)}
Cantidad de algún bien que debe sacrificarse para adquirir una unidad adicional de otro bien. En cualquier punto, es la pendiente de la frontera factible. Es equivalente a $-P_{Mg} \implies TMT = TMS$ también es válido para obtener óptimos.

\subsection{Efecto neto}
Es la suma del efecto ingreso y el efecto neto.

\subsubsection{Efecto ingreso}
Son los cambios en el consumo debido a una expansión en el conjunto factible, pero manteniendo el mismo trade-off entre cantidad consumida y tiempo libre.

Se deriva de la perdida (o ganancia) del poder adquisitivo, es ajeno a la variación relativa de los precios.
\\

\textit{Se desplaza la frontera factible verticalmente}, puede causar que la cantidad de bien dispuesto a invertir disminuya.\\


\subsubsection{Efecto sustitución}
Es la variación del consumo de bien cuyo precio varía. Se deriva exclusivamente de la variación de los precios, y es por lo tanto, ajeno al cambio de poder adquisitivo.
\\

\textit{Se da únicamente por cambios en el precio o el costo de oportunidad, dado el nuevo nivel de utilidad}, puede causar que la cantidad de bien dispuesto a invertir aumente.

\subsubsection{Como calcular los efectos}
\href{https://www.core-econ.org/the-economy/book/es/text/leibniz-03-07-01.html}{Según CORE.\\}


En un problema se obtienen valores de \textit{x} e \textit{y} en el óptimo, sean estos $x_0$ e $y_0$.

Cuando hay un cambio en el problema/situación planteada para calcular el efecto ingreso hay que, obtener la nueva utilidad óptima y los valores de las variables correspondientes ($x_1$ e $y_1$), e intentar conseguir un ingreso que causaría que la utilidad óptima fuera igual a la del problema modificado ($x_2$ e $y_2$). Luego la diferencia de ''x'' ($x_2 - x_0$) es el efecto ingreso. Si es que la diferencia no es igual a la diferencia total ($x_2 - x_0 \neq x_1 - x_0$), entonces lo faltante es el efecto sustitución. 

\subsection{Problemas de optimización}
En estos problemas se intenta optimizar una función objetivo, para encontrar su mínimo o máximo. Esta está sujeta a restricciones, que ponen límites a las variables presentes en el problema. 
\\

Un ejemplo es maximizar la función utilidad $U(x_1,x_2)$ donde la restricción es que $g(x_1,x_2) \leq w$, con $w$ un valor. Aquí las variables son $x_1$ y $x_2$.
\newpage
\section{Escasez}

\econ{https://www.core-econ.org/the-economy/book/es/text/03.html}{3}
\newline
\newline
Sea $Y = f(h)$ la función de producción, con $h$ el bien gastado en producir (puede ser horas, número de trabajadores, tierra utilizada, tiempo libre, etc). Se tendrá que:

\begin{itemize}
    \item \textbf{Producto medio}: $P_{Me} = \frac{f(h)}{h}$
    \newline 
    \newline Es la cantidad promedio de $Y$ conseguido con una cantidad $h$.
    \item \textbf{Producto marginal}: $P_{Mg} = \frac{df(h)}{dh}$
    \newline
    \newline Es el incremento en $Y$ que se obtiene por usar (o gastar) una cantidad mayor o menor de $h$. \textit{(Literalmente la derivada)}
    \newline También se puede ver como el retorno obtenido al usar distintas cantidades de $h$. \textit{(Cuanto Y cada h te da, que varía al aumentar o disminuir la cantidad de h)}
\end{itemize}

Donde la función de producción determina el conjunto factible de alternativas a elegir.

\subsection{Costo de oportunidad} 
\hyperlink{ejemplo-Costo de oportunidad}{Ejemplo}

Es el costo de la alternativa a la que renunciamos cuando tomamos una determinada decisión, incluyendo los beneficios que podríamos haber obtenido de haber escogido la opción alternativa. 

\subsection{Preferencias}

Para escoger una opción se depende de los beneficios frente a las otras opciones, que está intrínsecamente ligado a las preferencias de la persona.
\newline

Las preferencias de una persona se modelan a través de la función de \hyperlink{utilidad}{utilidad} $U$, que indica la utilidad percibida al consumir cierta cantidad dada del bien.

\subsubsection{Curvas de indiferencia}
Cuando las opciones indiferentes entre sí se marcan, se da origen a una curva de indiferencia. (En esta curva la utilidad siempre posee el mismo valor). 
\\

\textbf{Propiedades de las curvas de indiferencia:}

\begin{itemize}
    \item Las curvas de indiferencia tienen pendiente negativa que refleja las disyuntivas que implican una cierta renuncia: \textit{si hay dos combinaciones ante las que nos mostramos indiferentes, necesariamente eso implica que la que tenga más de un bien tendrá menos del otro bien.}
    \item Unas curvas de indiferencia más altas se corresponden a niveles de utilidad más altos: \textit{a medida que nos movemos hacia arriba y a la derecha en el diagrama, más lejos del origen, nos movemos a combinaciones que tienen más de ambos bienes.}
    \item Las curvas de indiferencia son suaves por lo general: \textit{cambios pequeños en la cantidad de bienes no causan grandes saltos en la utilidad.}
    \item Las curvas de indiferencia no se cruzan.
    \item A medida que te mueves hacia la derecha a lo largo de una curva de indiferencia, esta se vuelve más plana.
\end{itemize}

El último punto hace referencia a la \textit{utilidad decreciente de los bienes}, que establece que mientras más de un bien se tiene (o consume), menor es la utilidad adicional que se le otorga.

\subsection{Tasa marginal de sustitución (TMS)}
Es la cantidad de bien del eje vertical que se está dispuesto a dar por 'un' bien del eje horizontal, su valor viene dado por la tangente en un punto de la curva de indiferencia.
%Es cuanto se está dispuesto a dar por cada bien del eje vertical por algo del eje horizontal, 

En el caso de que la función de producción venga dada por $y = f(x)$ con $U = U(x, y)$ \[TMS = \abs{\frac{\partial U}{\partial x} / \frac{\partial U}{\partial y}} = -\frac{\partial U}{\partial x} / \frac{\partial U}{\partial y}\]

En palabras la fórmula anterior se puede escribir como
\[\text{TMS} = \abs{\frac{\text{utilidad marginal de \textit{x}}}{\text{utilidad marginal de \textit{y}}}}\]

\subsection{Optimización de la decisión}
Siempre se busca maximizar la utilidad sujeta al conjunto factible, este óptimo se encuentra en la \hyperlink{frontera-factible}{frontera del conjunto factible} y/o en la curva de indiferencia de mayor utilidad ('arriba a la derecha').

En el óptimo se cumple que la curva de indiferencia es tangente a la frontera factible y \[TMS = -P_{Mg}\] (Esto es equivalente a que las tangentes de la frontera factible y el producto marginal sean iguales)

\subsubsection{Tasa marginal de transformación (TMT)}
Cantidad de algún bien que debe sacrificarse para adquirir una unidad adicional de otro bien. En cualquier punto, es la pendiente de la frontera factible. Es equivalente a $-P_{Mg} \implies TMT = TMS$ también es válido para obtener óptimos.

\subsection{Efecto neto}
Es la suma del efecto ingreso y el efecto neto.

\subsubsection{Efecto ingreso}
Son los cambios en el consumo debido a una expansión en el conjunto factible, pero manteniendo el mismo trade-off entre cantidad consumida y tiempo libre.

Se deriva de la perdida (o ganancia) del poder adquisitivo, es ajeno a la variación relativa de los precios.
\\

\textit{Se desplaza la frontera factible verticalmente}, puede causar que la cantidad de bien dispuesto a invertir disminuya.\\


\subsubsection{Efecto sustitución}
Es la variación del consumo de bien cuyo precio varía. Se deriva exclusivamente de la variación de los precios, y es por lo tanto, ajeno al cambio de poder adquisitivo.
\\

\textit{Se da únicamente por cambios en el precio o el costo de oportunidad, dado el nuevo nivel de utilidad}, puede causar que la cantidad de bien dispuesto a invertir aumente.

\subsubsection{Como calcular los efectos}
\href{https://www.core-econ.org/the-economy/book/es/text/leibniz-03-07-01.html}{Según CORE.\\}


En un problema se obtienen valores de \textit{x} e \textit{y} en el óptimo, sean estos $x_0$ e $y_0$.

Cuando hay un cambio en el problema/situación planteada para calcular el efecto ingreso hay que, obtener la nueva utilidad óptima y los valores de las variables correspondientes ($x_1$ e $y_1$), e intentar conseguir un ingreso que causaría que la utilidad óptima fuera igual a la del problema modificado ($x_2$ e $y_2$). Luego la diferencia de ''x'' ($x_2 - x_0$) es el efecto ingreso. Si es que la diferencia no es igual a la diferencia total ($x_2 - x_0 \neq x_1 - x_0$), entonces lo faltante es el efecto sustitución. 

\subsection{Problemas de optimización}
En estos problemas se intenta optimizar una función objetivo, para encontrar su mínimo o máximo. Esta está sujeta a restricciones, que ponen límites a las variables presentes en el problema. 
\\

Un ejemplo es maximizar la función utilidad $U(x_1,x_2)$ donde la restricción es que $g(x_1,x_2) \leq w$, con $w$ un valor. Aquí las variables son $x_1$ y $x_2$.
\newpage
\section{Escasez}

\econ{https://www.core-econ.org/the-economy/book/es/text/03.html}{3}
\newline
\newline
Sea $Y = f(h)$ la función de producción, con $h$ el bien gastado en producir (puede ser horas, número de trabajadores, tierra utilizada, tiempo libre, etc). Se tendrá que:

\begin{itemize}
    \item \textbf{Producto medio}: $P_{Me} = \frac{f(h)}{h}$
    \newline 
    \newline Es la cantidad promedio de $Y$ conseguido con una cantidad $h$.
    \item \textbf{Producto marginal}: $P_{Mg} = \frac{df(h)}{dh}$
    \newline
    \newline Es el incremento en $Y$ que se obtiene por usar (o gastar) una cantidad mayor o menor de $h$. \textit{(Literalmente la derivada)}
    \newline También se puede ver como el retorno obtenido al usar distintas cantidades de $h$. \textit{(Cuanto Y cada h te da, que varía al aumentar o disminuir la cantidad de h)}
\end{itemize}

Donde la función de producción determina el conjunto factible de alternativas a elegir.

\subsection{Costo de oportunidad} 
\hyperlink{ejemplo-Costo de oportunidad}{Ejemplo}

Es el costo de la alternativa a la que renunciamos cuando tomamos una determinada decisión, incluyendo los beneficios que podríamos haber obtenido de haber escogido la opción alternativa. 

\subsection{Preferencias}

Para escoger una opción se depende de los beneficios frente a las otras opciones, que está intrínsecamente ligado a las preferencias de la persona.
\newline

Las preferencias de una persona se modelan a través de la función de \hyperlink{utilidad}{utilidad} $U$, que indica la utilidad percibida al consumir cierta cantidad dada del bien.

\subsubsection{Curvas de indiferencia}
Cuando las opciones indiferentes entre sí se marcan, se da origen a una curva de indiferencia. (En esta curva la utilidad siempre posee el mismo valor). 
\\

\textbf{Propiedades de las curvas de indiferencia:}

\begin{itemize}
    \item Las curvas de indiferencia tienen pendiente negativa que refleja las disyuntivas que implican una cierta renuncia: \textit{si hay dos combinaciones ante las que nos mostramos indiferentes, necesariamente eso implica que la que tenga más de un bien tendrá menos del otro bien.}
    \item Unas curvas de indiferencia más altas se corresponden a niveles de utilidad más altos: \textit{a medida que nos movemos hacia arriba y a la derecha en el diagrama, más lejos del origen, nos movemos a combinaciones que tienen más de ambos bienes.}
    \item Las curvas de indiferencia son suaves por lo general: \textit{cambios pequeños en la cantidad de bienes no causan grandes saltos en la utilidad.}
    \item Las curvas de indiferencia no se cruzan.
    \item A medida que te mueves hacia la derecha a lo largo de una curva de indiferencia, esta se vuelve más plana.
\end{itemize}

El último punto hace referencia a la \textit{utilidad decreciente de los bienes}, que establece que mientras más de un bien se tiene (o consume), menor es la utilidad adicional que se le otorga.

\subsection{Tasa marginal de sustitución (TMS)}
Es la cantidad de bien del eje vertical que se está dispuesto a dar por 'un' bien del eje horizontal, su valor viene dado por la tangente en un punto de la curva de indiferencia.
%Es cuanto se está dispuesto a dar por cada bien del eje vertical por algo del eje horizontal, 

En el caso de que la función de producción venga dada por $y = f(x)$ con $U = U(x, y)$ \[TMS = \abs{\frac{\partial U}{\partial x} / \frac{\partial U}{\partial y}} = -\frac{\partial U}{\partial x} / \frac{\partial U}{\partial y}\]

En palabras la fórmula anterior se puede escribir como
\[\text{TMS} = \abs{\frac{\text{utilidad marginal de \textit{x}}}{\text{utilidad marginal de \textit{y}}}}\]

\subsection{Optimización de la decisión}
Siempre se busca maximizar la utilidad sujeta al conjunto factible, este óptimo se encuentra en la \hyperlink{frontera-factible}{frontera del conjunto factible} y/o en la curva de indiferencia de mayor utilidad ('arriba a la derecha').

En el óptimo se cumple que la curva de indiferencia es tangente a la frontera factible y \[TMS = -P_{Mg}\] (Esto es equivalente a que las tangentes de la frontera factible y el producto marginal sean iguales)

\subsubsection{Tasa marginal de transformación (TMT)}
Cantidad de algún bien que debe sacrificarse para adquirir una unidad adicional de otro bien. En cualquier punto, es la pendiente de la frontera factible. Es equivalente a $-P_{Mg} \implies TMT = TMS$ también es válido para obtener óptimos.

\subsection{Efecto neto}
Es la suma del efecto ingreso y el efecto neto.

\subsubsection{Efecto ingreso}
Son los cambios en el consumo debido a una expansión en el conjunto factible, pero manteniendo el mismo trade-off entre cantidad consumida y tiempo libre.

Se deriva de la perdida (o ganancia) del poder adquisitivo, es ajeno a la variación relativa de los precios.
\\

\textit{Se desplaza la frontera factible verticalmente}, puede causar que la cantidad de bien dispuesto a invertir disminuya.\\


\subsubsection{Efecto sustitución}
Es la variación del consumo de bien cuyo precio varía. Se deriva exclusivamente de la variación de los precios, y es por lo tanto, ajeno al cambio de poder adquisitivo.
\\

\textit{Se da únicamente por cambios en el precio o el costo de oportunidad, dado el nuevo nivel de utilidad}, puede causar que la cantidad de bien dispuesto a invertir aumente.

\subsubsection{Como calcular los efectos}
\href{https://www.core-econ.org/the-economy/book/es/text/leibniz-03-07-01.html}{Según CORE.\\}


En un problema se obtienen valores de \textit{x} e \textit{y} en el óptimo, sean estos $x_0$ e $y_0$.

Cuando hay un cambio en el problema/situación planteada para calcular el efecto ingreso hay que, obtener la nueva utilidad óptima y los valores de las variables correspondientes ($x_1$ e $y_1$), e intentar conseguir un ingreso que causaría que la utilidad óptima fuera igual a la del problema modificado ($x_2$ e $y_2$). Luego la diferencia de ''x'' ($x_2 - x_0$) es el efecto ingreso. Si es que la diferencia no es igual a la diferencia total ($x_2 - x_0 \neq x_1 - x_0$), entonces lo faltante es el efecto sustitución. 

\subsection{Problemas de optimización}
En estos problemas se intenta optimizar una función objetivo, para encontrar su mínimo o máximo. Esta está sujeta a restricciones, que ponen límites a las variables presentes en el problema. 
\\

Un ejemplo es maximizar la función utilidad $U(x_1,x_2)$ donde la restricción es que $g(x_1,x_2) \leq w$, con $w$ un valor. Aquí las variables son $x_1$ y $x_2$.
\newpage
\section{Escasez}

\econ{https://www.core-econ.org/the-economy/book/es/text/03.html}{3}
\newline
\newline
Sea $Y = f(h)$ la función de producción, con $h$ el bien gastado en producir (puede ser horas, número de trabajadores, tierra utilizada, tiempo libre, etc). Se tendrá que:

\begin{itemize}
    \item \textbf{Producto medio}: $P_{Me} = \frac{f(h)}{h}$
    \newline 
    \newline Es la cantidad promedio de $Y$ conseguido con una cantidad $h$.
    \item \textbf{Producto marginal}: $P_{Mg} = \frac{df(h)}{dh}$
    \newline
    \newline Es el incremento en $Y$ que se obtiene por usar (o gastar) una cantidad mayor o menor de $h$. \textit{(Literalmente la derivada)}
    \newline También se puede ver como el retorno obtenido al usar distintas cantidades de $h$. \textit{(Cuanto Y cada h te da, que varía al aumentar o disminuir la cantidad de h)}
\end{itemize}

Donde la función de producción determina el conjunto factible de alternativas a elegir.

\subsection{Costo de oportunidad} 
\hyperlink{ejemplo-Costo de oportunidad}{Ejemplo}

Es el costo de la alternativa a la que renunciamos cuando tomamos una determinada decisión, incluyendo los beneficios que podríamos haber obtenido de haber escogido la opción alternativa. 

\subsection{Preferencias}

Para escoger una opción se depende de los beneficios frente a las otras opciones, que está intrínsecamente ligado a las preferencias de la persona.
\newline

Las preferencias de una persona se modelan a través de la función de \hyperlink{utilidad}{utilidad} $U$, que indica la utilidad percibida al consumir cierta cantidad dada del bien.

\subsubsection{Curvas de indiferencia}
Cuando las opciones indiferentes entre sí se marcan, se da origen a una curva de indiferencia. (En esta curva la utilidad siempre posee el mismo valor). 
\\

\textbf{Propiedades de las curvas de indiferencia:}

\begin{itemize}
    \item Las curvas de indiferencia tienen pendiente negativa que refleja las disyuntivas que implican una cierta renuncia: \textit{si hay dos combinaciones ante las que nos mostramos indiferentes, necesariamente eso implica que la que tenga más de un bien tendrá menos del otro bien.}
    \item Unas curvas de indiferencia más altas se corresponden a niveles de utilidad más altos: \textit{a medida que nos movemos hacia arriba y a la derecha en el diagrama, más lejos del origen, nos movemos a combinaciones que tienen más de ambos bienes.}
    \item Las curvas de indiferencia son suaves por lo general: \textit{cambios pequeños en la cantidad de bienes no causan grandes saltos en la utilidad.}
    \item Las curvas de indiferencia no se cruzan.
    \item A medida que te mueves hacia la derecha a lo largo de una curva de indiferencia, esta se vuelve más plana.
\end{itemize}

El último punto hace referencia a la \textit{utilidad decreciente de los bienes}, que establece que mientras más de un bien se tiene (o consume), menor es la utilidad adicional que se le otorga.

\subsection{Tasa marginal de sustitución (TMS)}
Es la cantidad de bien del eje vertical que se está dispuesto a dar por 'un' bien del eje horizontal, su valor viene dado por la tangente en un punto de la curva de indiferencia.
%Es cuanto se está dispuesto a dar por cada bien del eje vertical por algo del eje horizontal, 

En el caso de que la función de producción venga dada por $y = f(x)$ con $U = U(x, y)$ \[TMS = \abs{\frac{\partial U}{\partial x} / \frac{\partial U}{\partial y}} = -\frac{\partial U}{\partial x} / \frac{\partial U}{\partial y}\]

En palabras la fórmula anterior se puede escribir como
\[\text{TMS} = \abs{\frac{\text{utilidad marginal de \textit{x}}}{\text{utilidad marginal de \textit{y}}}}\]

\subsection{Optimización de la decisión}
Siempre se busca maximizar la utilidad sujeta al conjunto factible, este óptimo se encuentra en la \hyperlink{frontera-factible}{frontera del conjunto factible} y/o en la curva de indiferencia de mayor utilidad ('arriba a la derecha').

En el óptimo se cumple que la curva de indiferencia es tangente a la frontera factible y \[TMS = -P_{Mg}\] (Esto es equivalente a que las tangentes de la frontera factible y el producto marginal sean iguales)

\subsubsection{Tasa marginal de transformación (TMT)}
Cantidad de algún bien que debe sacrificarse para adquirir una unidad adicional de otro bien. En cualquier punto, es la pendiente de la frontera factible. Es equivalente a $-P_{Mg} \implies TMT = TMS$ también es válido para obtener óptimos.

\subsection{Efecto neto}
Es la suma del efecto ingreso y el efecto neto.

\subsubsection{Efecto ingreso}
Son los cambios en el consumo debido a una expansión en el conjunto factible, pero manteniendo el mismo trade-off entre cantidad consumida y tiempo libre.

Se deriva de la perdida (o ganancia) del poder adquisitivo, es ajeno a la variación relativa de los precios.
\\

\textit{Se desplaza la frontera factible verticalmente}, puede causar que la cantidad de bien dispuesto a invertir disminuya.\\


\subsubsection{Efecto sustitución}
Es la variación del consumo de bien cuyo precio varía. Se deriva exclusivamente de la variación de los precios, y es por lo tanto, ajeno al cambio de poder adquisitivo.
\\

\textit{Se da únicamente por cambios en el precio o el costo de oportunidad, dado el nuevo nivel de utilidad}, puede causar que la cantidad de bien dispuesto a invertir aumente.

\subsubsection{Como calcular los efectos}
\href{https://www.core-econ.org/the-economy/book/es/text/leibniz-03-07-01.html}{Según CORE.\\}


En un problema se obtienen valores de \textit{x} e \textit{y} en el óptimo, sean estos $x_0$ e $y_0$.

Cuando hay un cambio en el problema/situación planteada para calcular el efecto ingreso hay que, obtener la nueva utilidad óptima y los valores de las variables correspondientes ($x_1$ e $y_1$), e intentar conseguir un ingreso que causaría que la utilidad óptima fuera igual a la del problema modificado ($x_2$ e $y_2$). Luego la diferencia de ''x'' ($x_2 - x_0$) es el efecto ingreso. Si es que la diferencia no es igual a la diferencia total ($x_2 - x_0 \neq x_1 - x_0$), entonces lo faltante es el efecto sustitución. 

\subsection{Problemas de optimización}
En estos problemas se intenta optimizar una función objetivo, para encontrar su mínimo o máximo. Esta está sujeta a restricciones, que ponen límites a las variables presentes en el problema. 
\\

Un ejemplo es maximizar la función utilidad $U(x_1,x_2)$ donde la restricción es que $g(x_1,x_2) \leq w$, con $w$ un valor. Aquí las variables son $x_1$ y $x_2$.
\newpage
\section{Escasez}

\econ{https://www.core-econ.org/the-economy/book/es/text/03.html}{3}
\newline
\newline
Sea $Y = f(h)$ la función de producción, con $h$ el bien gastado en producir (puede ser horas, número de trabajadores, tierra utilizada, tiempo libre, etc). Se tendrá que:

\begin{itemize}
    \item \textbf{Producto medio}: $P_{Me} = \frac{f(h)}{h}$
    \newline 
    \newline Es la cantidad promedio de $Y$ conseguido con una cantidad $h$.
    \item \textbf{Producto marginal}: $P_{Mg} = \frac{df(h)}{dh}$
    \newline
    \newline Es el incremento en $Y$ que se obtiene por usar (o gastar) una cantidad mayor o menor de $h$. \textit{(Literalmente la derivada)}
    \newline También se puede ver como el retorno obtenido al usar distintas cantidades de $h$. \textit{(Cuanto Y cada h te da, que varía al aumentar o disminuir la cantidad de h)}
\end{itemize}

Donde la función de producción determina el conjunto factible de alternativas a elegir.

\subsection{Costo de oportunidad} 
\hyperlink{ejemplo-Costo de oportunidad}{Ejemplo}

Es el costo de la alternativa a la que renunciamos cuando tomamos una determinada decisión, incluyendo los beneficios que podríamos haber obtenido de haber escogido la opción alternativa. 

\subsection{Preferencias}

Para escoger una opción se depende de los beneficios frente a las otras opciones, que está intrínsecamente ligado a las preferencias de la persona.
\newline

Las preferencias de una persona se modelan a través de la función de \hyperlink{utilidad}{utilidad} $U$, que indica la utilidad percibida al consumir cierta cantidad dada del bien.

\subsubsection{Curvas de indiferencia}
Cuando las opciones indiferentes entre sí se marcan, se da origen a una curva de indiferencia. (En esta curva la utilidad siempre posee el mismo valor). 
\\

\textbf{Propiedades de las curvas de indiferencia:}

\begin{itemize}
    \item Las curvas de indiferencia tienen pendiente negativa que refleja las disyuntivas que implican una cierta renuncia: \textit{si hay dos combinaciones ante las que nos mostramos indiferentes, necesariamente eso implica que la que tenga más de un bien tendrá menos del otro bien.}
    \item Unas curvas de indiferencia más altas se corresponden a niveles de utilidad más altos: \textit{a medida que nos movemos hacia arriba y a la derecha en el diagrama, más lejos del origen, nos movemos a combinaciones que tienen más de ambos bienes.}
    \item Las curvas de indiferencia son suaves por lo general: \textit{cambios pequeños en la cantidad de bienes no causan grandes saltos en la utilidad.}
    \item Las curvas de indiferencia no se cruzan.
    \item A medida que te mueves hacia la derecha a lo largo de una curva de indiferencia, esta se vuelve más plana.
\end{itemize}

El último punto hace referencia a la \textit{utilidad decreciente de los bienes}, que establece que mientras más de un bien se tiene (o consume), menor es la utilidad adicional que se le otorga.

\subsection{Tasa marginal de sustitución (TMS)}
Es la cantidad de bien del eje vertical que se está dispuesto a dar por 'un' bien del eje horizontal, su valor viene dado por la tangente en un punto de la curva de indiferencia.
%Es cuanto se está dispuesto a dar por cada bien del eje vertical por algo del eje horizontal, 

En el caso de que la función de producción venga dada por $y = f(x)$ con $U = U(x, y)$ \[TMS = \abs{\frac{\partial U}{\partial x} / \frac{\partial U}{\partial y}} = -\frac{\partial U}{\partial x} / \frac{\partial U}{\partial y}\]

En palabras la fórmula anterior se puede escribir como
\[\text{TMS} = \abs{\frac{\text{utilidad marginal de \textit{x}}}{\text{utilidad marginal de \textit{y}}}}\]

\subsection{Optimización de la decisión}
Siempre se busca maximizar la utilidad sujeta al conjunto factible, este óptimo se encuentra en la \hyperlink{frontera-factible}{frontera del conjunto factible} y/o en la curva de indiferencia de mayor utilidad ('arriba a la derecha').

En el óptimo se cumple que la curva de indiferencia es tangente a la frontera factible y \[TMS = -P_{Mg}\] (Esto es equivalente a que las tangentes de la frontera factible y el producto marginal sean iguales)

\subsubsection{Tasa marginal de transformación (TMT)}
Cantidad de algún bien que debe sacrificarse para adquirir una unidad adicional de otro bien. En cualquier punto, es la pendiente de la frontera factible. Es equivalente a $-P_{Mg} \implies TMT = TMS$ también es válido para obtener óptimos.

\subsection{Efecto neto}
Es la suma del efecto ingreso y el efecto neto.

\subsubsection{Efecto ingreso}
Son los cambios en el consumo debido a una expansión en el conjunto factible, pero manteniendo el mismo trade-off entre cantidad consumida y tiempo libre.

Se deriva de la perdida (o ganancia) del poder adquisitivo, es ajeno a la variación relativa de los precios.
\\

\textit{Se desplaza la frontera factible verticalmente}, puede causar que la cantidad de bien dispuesto a invertir disminuya.\\


\subsubsection{Efecto sustitución}
Es la variación del consumo de bien cuyo precio varía. Se deriva exclusivamente de la variación de los precios, y es por lo tanto, ajeno al cambio de poder adquisitivo.
\\

\textit{Se da únicamente por cambios en el precio o el costo de oportunidad, dado el nuevo nivel de utilidad}, puede causar que la cantidad de bien dispuesto a invertir aumente.

\subsubsection{Como calcular los efectos}
\href{https://www.core-econ.org/the-economy/book/es/text/leibniz-03-07-01.html}{Según CORE.\\}


En un problema se obtienen valores de \textit{x} e \textit{y} en el óptimo, sean estos $x_0$ e $y_0$.

Cuando hay un cambio en el problema/situación planteada para calcular el efecto ingreso hay que, obtener la nueva utilidad óptima y los valores de las variables correspondientes ($x_1$ e $y_1$), e intentar conseguir un ingreso que causaría que la utilidad óptima fuera igual a la del problema modificado ($x_2$ e $y_2$). Luego la diferencia de ''x'' ($x_2 - x_0$) es el efecto ingreso. Si es que la diferencia no es igual a la diferencia total ($x_2 - x_0 \neq x_1 - x_0$), entonces lo faltante es el efecto sustitución. 

\subsection{Problemas de optimización}
En estos problemas se intenta optimizar una función objetivo, para encontrar su mínimo o máximo. Esta está sujeta a restricciones, que ponen límites a las variables presentes en el problema. 
\\

Un ejemplo es maximizar la función utilidad $U(x_1,x_2)$ donde la restricción es que $g(x_1,x_2) \leq w$, con $w$ un valor. Aquí las variables son $x_1$ y $x_2$.
\newpage

\titleformat{\section}
  {\Large\bfseries}{\thesection.}{1px}{}

\section{Videos EOL}

\subsection*{Módulo 0: Introducción a la economía}

\begin{itemize}
    \item[-] \href{https://www.youtube.com/watch?v=Bk_XdL-Yq2E}{Bienestar, crecimiento y desigualdad.}
    
    \item[-] \href{https://www.youtube.com/watch?v=QhNrHly3Kd8}{Especialización del trabajo, intercambio y mercados.}

\end{itemize}

\subsection*{Módulo 1: Uso de modelos matemáticos en la economía}

\begin{itemize}
    \item[-] \href{https://www.youtube.com/watch?v=PhbLaTLzIAk}{Función de producción y Trampa de Malthus.}
    
    \item[-] \href{https://www.youtube.com/watch?v=DXGO7kK1uXI}{Tecnología, costos e innovación.}
    
    \item[-] \href{https://www.youtube.com/watch?v=-bYKV0pErEA}{Tutorial regresión lineal.}
    
\end{itemize}

\subsection*{Módulo 2: Escasez}

\begin{itemize}
    \item[-] \href{https://www.youtube.com/watch?v=P29Zd1-FGxE}{Motivación y función producción.}
    
    \item[-] \href{https://www.youtube.com/watch?v=R7monXl_M4E}{Preferencias y tomas de decisión.}
    
    \item[-] \href{https://www.youtube.com/watch?v=J-MyCczFGZc}{Salario y Efectos.} (Efecto ingreso y sustitución)
    
    \item[-] \href{https://www.youtube.com/watch?v=6gd_bGovblk}{Explicando Diferencias y Conclusión.}
    
\end{itemize}

\subsection*{Módulo 3: Teoría de Juegos}

\begin{itemize}
    \item[-] \href{https://www.youtube.com/watch?v=2B01O0VaJg4}{Presentación del módulo.}
    
    \item[-] \href{https://www.youtube.com/watch?v=Dh2AqD6h9nI}{Motivación y definiciones básicas.}
    
    \item[-] \href{https://www.youtube.com/watch?v=fKQ_5sbB9pI}{Estrategias dominadas.}
    
    \item[-] \href{https://youtu.be/S5mZmz6Q3PE}{Equilibrio de Nash.}
    
\end{itemize}

\subsection*{Módulo 4: Criterios de asignación}

\begin{itemize}
    \item[-] \href{https://youtu.be/VmjpxTykH5s}{Evaluación de Asignaciones.}
    
    \item[-] \href{https://youtu.be/jWcbHlTCPnQ}{Determinación de asignaciones.}
    
    \item[-] \href{https://youtu.be/PnxKqsnTKOM}{Desigualdad.}
    
\end{itemize}

\subsection*{Módulo 5: Monopolios}

\begin{itemize}
    \item[-] \href{https://youtu.be/WADyP3bQrH4}{Motivación}
    
    \item[-] \href{https://youtu.be/5jgzGaegmcY}{El problema del monopolio.}
    
    \item[-] \href{https://youtu.be/aORZTXIdOHw}{Relación entre la demanda y el margen monopólico.}
    
    \item[-] \href{https://youtu.be/VMXx_u8HF00}{Ineficiencia, monopolios naturales y regulación.}
    
\end{itemize}

\subsection*{Módulo 6: Discriminación de precios y tarificación}

\begin{itemize}
    \item[-] \href{https://www.youtube.com/watch?v=NR6JSmtmvkA}{Segmentos de mercado y monopolista}
    
    \item[-] \href{https://www.youtube.com/watch?v=pkvyhRhybI4}{Segmentación y discriminación de precios}
    
    \item[-] \href{https://www.youtube.com/watch?v=HlnLgL_2_jg}{Discriminación por volumen}
    
    \item[-] \href{https://www.youtube.com/watch?v=xHRb9aZVXUA}{Captura de excedentes}
    
\end{itemize}

\subsection*{Módulo 7: Oligopolio}

\begin{itemize}
    \item[-] \href{https://www.youtube.com/watch?v=dAHkFh1iNvk}{Motivación y definición}
    
    \item[-] \href{https://www.youtube.com/watch?v=lz5_zY4xf_Y}{Modelo de Cournot}
    
    \item[-] \href{https://www.youtube.com/watch?v=MC0dFTZrMXk}{Eficiencia, concentración y estructura de mercado}
    
    \item[-] \href{https://www.youtube.com/watch?v=aPzrelUix5w}{Política de competencia} 
    
\end{itemize}

\newcommand{\ytem}[2]{\item[-] \href{#1}{#2}}

\subsection*{Modulo 8: Mercados competitivos y firmas tomadoras de precios}

\begin{itemize}
    \ytem{https://www.youtube.com/watch?v=waAob5x6Cu4}{Introducción}
    
    \ytem{https://www.youtube.com/watch?v=WUn693uV38A}{Oferta, demanda y equilibrio de mercado}
    
    \ytem{https://www.youtube.com/watch?v=9vpN6H6RV2g}{Cambios en la oferta y demanda}
    
    \ytem{https://www.youtube.com/watch?v=ACTfimO_lXA}{Impuestos y subsidios}
    
\end{itemize}

\subsection*{Modulo 9: Fallas de Mercado y Externalidades}

\begin{itemize}
    \ytem{https://www.youtube.com/watch?v=OjlhIaO1YxQ}{Presentación del módulo}
    
    \ytem{https://www.youtube.com/watch?v=WJwVzEfAYmI}{Eficiencia y Externalidades}
    
    \ytem{https://www.youtube.com/watch?v=n8oAss-o59k}{Correción de Fallas}
    
    \ytem{https://www.youtube.com/watch?v=iccZWCPPwhA}{Otras fuentes de ineficiencia}
    
\end{itemize}

\newpage
\section{Glosario de términos}

\textit{Glosario de términos del CORE ECON (Puede buscar con ctrl+f o cmd+f)} \href{https://www.core-econ.org/the-economy/book/es/text/50-02-glossary.html#glossary-econom}{\textbf{IR}}

\begin{itemize}
    
    \item \hypertarget{trade-off}{\textbf{Trade Off}}: decisión tomada en una situación conflictiva en la cual se debe perder, reducir cierta cualidad a cambio de otra cualidad. En economía se lo suele traducir como 'intercambio', destacando entonces que se pierde un beneficio y se gana otro. \textit{[Fuente: Wikipedia]}
    
    \item \hypertarget{equilibrio}{\textbf{Equilibrio}}: Resultado autosostenible de un modelo. En este caso, algo de interés no cambia, a menos que se introduzca una fuerza externa que altere la descripción de la situación que proporciona el modelo. \textit{[Fuente: CORE]}
    
    \item \hypertarget{subsistencia}{\textbf{Nivel de subsistencia}}: Nivel de vida (medido en términos del consumo o el ingreso) al que la población no crece ni decrece. \textit{[Fuente: CORE]}
    
    \item \hypertarget{rendimiento-decreciente}{\textbf{Rendimiento decreciente}}: Situación en la cual el uso de una unidad adicional de un insumo de producción resulta en un menor incremento en el producto, respecto al incremento anterior. \textit{También se conoce como: rendimientos marginales decrecientes de la producción.} \fuente{CORE}
    
    \item \hypertarget{utilidad}{\textbf{Utilidad}}: Indicador numérico de valor que uno asigna a un resultado, de modo que se escojan resultados de mayor valor por encima de otros de menor valor cuando ambos sean factibles. \fuente{CORE}
    
    \item \hypertarget{frontera-factible}{\textbf{Frontera factible}}: Curva de puntos que define la máxima cantidad factible de un bien para una cantidad dada de otro. \fuente{CORE}
    
    \item \hypertarget{poder}{\textbf{Poder}}: capacidad de hacer y obtener las cosas que queremos, en contraposición con las intenciones de los demás.
    
    \item \hypertarget{instituciones}{\textbf{Instituciones}}: son reglas escritas y no escritas que rigen qué hacen las personas cuando interactúan en un proyecto común, y la distribución de los productos resultantes de su esfuerzo conjunto.
    
    \item \hypertarget{ex-ante}{\textbf{Ex-ante}}: significa "antes del suceso". Ex-ante se usa más comúnmente en el mundo comercial, donde los resultados de una acción concreta, o una serie de acciones, se prevén con antelación (o eso se pretende).
    
\end{itemize}

\newpage
\section{Misceláneo}

\subsection{Módulo 6}

\subsubsection{\cita{Mismo bien}}
\label{misc:mismo_precio}
¿Como se define un mismo bien?\\

Diferencias geográficas pueden causar diferencia de costes para un mismo producto, a causa de los costes de producción.\\


¿Si los bienes son distintos, habría discriminación?\\

Se podría discriminar sin que el bien sea exactamente el mismo (calidad del bien/servicio es distinta).\\


Se toma como un \cita{mismo bien}, uno casi indistinguible de otro.

\newpage




\section{Ejemplos}

\subsection{Costo de oportunidad}

\fuente{CORE}

Imagine que se les ha pedido a un contador y a un economista que informen sobre el costo de ir a un concierto \textit{A}, en un teatro, con una entrada cuyo costo asciende a 25 dólares. En un parque cercano hay un concierto \textit{B}, que es gratuito, pero que se celebra al mismo tiempo.
\\

\textbf{contador}:
el costo del concierto \textit{A} es el costo entendido como «lo que sale de su bolsillo»: usted ha pagado 25 dólares por una entrada, por lo tanto, el costo es 25 dólares.
\\

\textbf{economista}
¿Pero a qué tiene que renunciar para ir al concierto \textit{A}? Usted ha dado 25 dólares, más el disfrute del concierto gratuito en el parque. Así que el costo del concierto para usted es el costo en términos de lo que sale de su bolsillo más el costo de oportunidad.
Suponga que lo máximo que hubiera estado dispuesto a pagar para asistir al concierto gratuito en el parque (si no fuera gratuito) fueran 15 dólares. Entonces su beneficio, si es que eligiera su siguiente mejor alternativa al concierto \textit{A}, sería de 15 dólares de disfrute en el parque. Este es el costo de oportunidad de ir al concierto \textit{A}.
\\

Así que el costo económico (costo de bolsillo de una acción + costo de oportunidad) total del concierto \textit{A} es 25 dólares + 15 dólares = 40 dólares. Si anticipa que el goce que experimentará por ir al concierto \textit{A} es 50 dólares, dejará pasar el concierto \textit{B} y comprará la entrada para el teatro, porque 50 dólares es más que 40 dólares. Por otro lado, si anticipa que el goce que experimentará en el concierto \textit{A} es 35 dólares, entonces el costo económico de 40 dólares indica que no escogerá ir al teatro. En términos simples: dado que tiene que pagar 25 dólares por la entrada, optará por el concierto \textit{B} y se guardará los 25 dólares para gastarlos en otras cosas y disfrutar así de un beneficio valorado en 15 dólares resultante de ir al concierto gratuito en el parque.

\subsection{Discriminación primer grado}
\label{ejem:disc_1er_grado}
\begin{itemize}
    \item $1^{\text{er}}$ caso: el médico de un pueblo pequeño
    \item El monopolista fija precios diferentes para cada consumidor y para cada consulta comprada por cada uno de ellos
    \item Información: el monopolista (médico) puede identificar a cada consumidor.
    \item Arbitraje o fuga: no es posible
    \item Precios: diferentes para cada consumidor y unidad
\end{itemize}

\subsection{Discriminación de segundo grado}
\label{ejem:disc_2do_grado}
\begin{itemize}
    \item \textit{A} valora en \$3 la primera unidad, en \$2 la segunda, y \$1 la tercera.
    \item \textit{B}valora en \$4 la primera unidad, en \$3 la segunda, y en \$2 la tercera.
    \item Curva de Demanda: $(\$4, 1)$, $(\$3,3)$, $(\$2,5)$, $(\$1,6)$
    \item Mejor precio ($CMg = 0$)=\$2. Se venden 5 unidades a \$10.
    \item Si se tiene una unidad \$3, dos unidades \$4.8, tres unidades a \$6.5\\
    \textit{A} compra dos unidades, \textit{B} compra 3 unidades (Se auto-segmentan). Se venden 5 unidades en \$11.3
\end{itemize}

\newpage

\end{document}
