\section{Oligopolios}

Son los mercados en donde hay varios actores, por lo que el modelo de monopolios no es adecuado. 

\subsection{Modelo de Cournot lineal}
Sea $N$ la cantidad de firmas que producen un producto homogéneo, el modelo supone que su demanda $P(Q)$ y sus costos $c_i(q_i)$ son lineales, y en el caso del costo homogéneos, es decir

\[P(Q) = a-bQ \quad a,b\geq 0\]
\[c_i(q_i) = c(q_i) = c\cdot q_i \quad 0 \leq c \leq a\]

La competencia en este modelos es en cantidades, siendo esta la distinción con otros modelos de oligopolio, donde cada firma decide el número de unidades a a producir y vender.\\

Y la utilidad que cada firma $i$ maximizará, con $\sum_jq_j=Q^*$ la cantidad total, es

\[\pi_i(q_i,q_j)=q_i(a-b\sum_jq_j)-cq_i\]

La decisión de cuánto vender dependerá de la decisión de los rivales, por lo que este problema define un juego, para el cual se buscará un equilibrio de Nash.\\

En el quilibrio de Nash simétrico, todas las firmas producirán lo mismo, obteniendo las mismos ingresos. Siendo su valor
\[q_i^* = \frac{a-c}{(N+1)b}\]
Donde la cantidad total y el precio vienen dados por
\[Q^* = \frac{N(a-c)}{(N+1)b} = \sum_iq_i\]
\[P^* = \frac{a + Nc}{N+1}\]
El poder de mercado de las firmas tiene márgenes positivos, pues \[a > c \implies P^* > c\]

Notemos que si $N \longrightarrow \infty$, entonces el precio tiende al costo del producto, que corresponde a que las firmas producen a costos marginales. 

\subsection{Ineficiencia del Oligopolio}
La cantidad producida en mercados oligopólicos es menor a la cantidad socialmente eficiente, pero es mayor a la cantidad monopólica.\\

Los excedentes totales aumentan con el número de firmas.

\subsection{Costos sociales de la concentración de mercados}
Se dice que un mercado es \cita{menos concentrado} mientras más firmas participan en él.\\

En mercados menos concentrados los precios son menores, los consumidores tienen más excedente y el excedente social es mayor.\\

Además, en mercados menos concentrados hay otras ganancias sociales en término de desigualdad de ingresos y discriminación.

\subsection{Estructura de mercado}
El número de firmas que hay en un mercado está determinado por los costos de entrada.\\

El costo de entrada $C_e$ modela costos que una firma debe incurrir antes de comenzar su producción, independiente de las unidades que decida producir. Por ejemplo, infraestructura, una inversión inicial o patentes.\\

La condición de entrada determina el número de firmas $N$, donde la cantidad de firmas en equilibrio es

\[N* = \lados{\lfloor}{\frac{a-c}{\sqrt{bC_E}} -1}\]

Donde los corchetes representan que se le aplica la función \textit{suelo} o \textit{parte entera menor}, $c$ son los costos marginales, $a$ y $b$ provienen de la función demanda lineal (ver más arriba) y $C_e$ son los costos de entrada.


\subsubsection{Cantidad de firmas según costos de entrada}

\begin{itemize}
    \item Costos de entrada grandes impiden existencia de dos o más firmas: \textbf{monopolios naturales}. (e.j. distribución eléctrica, sanitarias)
    
    \item Costos de entrada moderados (en comparación a la demanda): \textbf{duopolios u oligopolios con pocas firmas}. (e.j. Airbus-Boeing, telecomunicaciones, periódicos, farmacias, etc)
    
    \item Costos de entrada pequeños: \textbf{muchas firmas}. (e.j. mercados agrícolas)
    
\end{itemize}

\subsection{Modelo de Stackelberg}

Supone dos firmas que compiten a lo Cournot pero una de ellas \cita{juega} primero que la otra. En este caso, si la firma 1 juega primero se tiene que 
\[q_1*=\frac{a-c}{2b}, \quad \quad q_2*=\frac{a-c}{4b}\]

\subsection{Modelo de Bertrand}

Supone que las firmas fijan precios y dejan que los consumidores decidan cuanto quieren comprar a esos precios. También posee los supuestos de que el producto es homogéneo, la demanda y el costo lineal y homogéneo.\\

Se diferencia con otros modelos porque las firmas eligen el precio de manera simultánea (se hace la suposición de que no tienen restricciones de capacidad).\\

Su equilibrio de Nash es $p_1=p_2=c$, es decir, ambas firmas cobran los costos marginales.

\subsubsection{Paradoja de Bertrand}

Bastan dos firmas para que se alcance la competencia perfecta. Y se denomina paradoja por que en muchas situaciones no se observa este comportamiento.


Algunas soluciones a la paradoja son
\begin{itemize}
    \item Restricciones de capacidad
    \item Productos diferenciados
    \item No es una interacción única en el tiempo; las firmas entienden que una guerra de precio diluye las utilidades
\end{itemize}

\subsection{Política de competencia}
Los mercados con bajos niveles de competencia resultan en pérdida de bienestar social, por ello es tan importante fomentarla. Aunque esto conlleva la existencia de dos desafíos esenciales: fusiones y carteles.

\subsubsection{Fusiones}
Las fusiones consisten en que dos o más firmas se juntan para crear una sola, permitiéndoles ser más eficientes y por consecuencia teniendo menos precios.

\subsubsection{Carteles}
Las firmas pueden actuar cooperativamente formando un cartel, subiendo el precio en un mercado y alcanzando utilidades cercanas a las monopólicas. (e.g. Colusión confort, farmacias)\\

La colusión hace que el resultado del mercado se asemeje al monopolio, sin embargo coludirse puede ser difícil ya que se requiere cooperación entre las firmas.\\

La colusión es siempre dañina para la competencia y los consumidores, y por ello es sancionada.

%\subsection{Concentración de mercados en Chile}

\newpage