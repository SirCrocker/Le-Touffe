\section{Mercados competitivos y firmas tomadoras de precios}

Mercados competitivos o perfectamente competitivos son los cuales las firmas son tomadoras de precio. El precio es por una parte una señal para que los agentes consuman y produzcan, y además sirve para balancear la oferta y demanda de bienes y servicios.\\

La competencia perfecta es una idealización, hay muchos mercados en que los agentes pueden manipular precios.


\subsection{Curva de oferta}

Modela la cantidad que las firmas desean vender en función del precio $P$, esta ordena las firmas de izquierda a derecha partiendo por las que poseen menor costo de producción.\\

La \cita{Ley de oferta} se denota por $O(P)$ y es una función creciente con P.\\

También se tiene la notación $P_s(Q)$, que corresponde a $O^{-1}(Q)$.

\subsubsection{Demanda}
Es importante hacer notar que al hablar de oferta se tiene que la demanda ($D$) también será una función dependiente del precio ($P$). Siendo la función demanda decreciente.\\

Esta es la curva de lo que se está dispuesto a consumir para un determinado bien.\\

Depende principalmente del ingreso o renta disponible.\\
Otras variables es si hay bienes sustitutos o complementarios (e.g. si baja el precio de la \textit{Pepsi} compro menos \textit{Coca-Cola}) y las preferencias (e.g. el clima $\to$ distinto tipos de ropa)

Se denota por $D(P)$, y su inversa $D^{-1}(Q)$ se denota por $P_d(Q)$

\subsection{Equilibrio competitivo}

Dada una curva de demanda $D(P)$ y una curva de oferta $O(P)$, el precio de equilibrio competitivo $P^*$ y la cantidad de equilibrio competitivo $Q^*$ es tal que

\[D(P^*) = O(P^*) = Q^*\]

Si $O(P)$ es continua y estrictamente creciente, $D(P)$ es continua y estrictamente decreciente y se cumple

\[D(0) > O(0) \quad \land \quad \lim_{P \to \infty}D(P) < \lim_{P \to \infty} O(P)\]

Entonces el precio de equilibrio competitivo existe y es único.

\subsubsection{Excesos}

Sea $P_0$ un precio cualquiera y $P^*$ el precio de equilibrio competitivo

\begin{itemize}
    \item Si $P_0 > P^*$, hay un exceso de oferta. (O excedente)
    \item Si $P_0 = P^*$, $P_0$ es el precio de equilibrio.
    \item Si $P_0 < P^*$, hay un exceso de demanda. (O escasez)
\end{itemize}

\subsection{Cambios en la demanda}

La curva de demanda puede cambiar por distintas razones, como son cambios en preferencias, el ingreso o en el precio de los bienes, entre otros.\\

Cuando la demanda aumenta, el precio de equilibrio y la cantidad de equilibrio también. Y de manera vice-versa cuando disminuye la demanda.

\subsection{Cambios en la oferta}

Esta puede cambiar por razones como innovaciones tecnológicas, cambios en los precios de los insumos, entre otros.\\

Cuando se contrae la curva de oferta, el precio de equilibrio aumenta y la cantidad de equilibrio disminuye. Y de manera viceversa cuando se expande la oferta.

\subsection{Excedente}

Hay tres \cita{tipos}

\begin{enumerate}[label=\Roman*.]
    \item \textbf{Excedente de los consumidores}, mide la ganancia total que tienen los consumidores cuando adquieren $Q^*$ unidades y pagan $P^*$ por cada una. Se calcula como
    \[E_C=\int_0^{Q^*}(D^{-1}(s)-P^*)\ ds\]
    
    \item \textbf{Excedente de los productores}, mide la ganancia total de los productores que venden $Q^*$ unidades a precio $P^*$. Se calcula como
    \[E_P=\int_0^{Q^*}(P^*-O^{-1}(s))\ ds\]
    
    \item \textbf{Excedente total}, es la suma de ambos excedentes, mide la ganancia total que se produce al transar $Q^*$ unidades. Se calcula como
    \[E_T=\int_0^{Q^*}(D^{-1}(s) - O^{-1}(s))\ ds\]
\end{enumerate}

\subsection{Mano invisible de mercado}

Sea $P^*$ el precio de equilibrio competitivo y $Q^*$ la cantidad de equilibrio competitivo. Entonces, $Q^*$ maximiza el excedente total:

\[Q^* \in \mathrm{arg}\max_{q}\int_{0}^q(D^{-1}(s) - O^{-1}(s))\ ds\]

Si se cumple que 
\begin{itemize}
    \item La competencia es perfecta
    \item Hay información completa
    \item No hay externalidades
\end{itemize}

Entonces el equilibrio maximizará el excedente social.

\subsection{Impuestos y subsidios}

Tienen varios roles, como son, recaudar recursos para las múltiples tareas del Estado e incentivar o desincentivar el consumo o producción de algunos bienes y servicios.\\

El impuesto, $\iota$, provee el servicio de recaudación al Estado, donde su valor es $\iota Q^*$. Con $Q^*$ la cantidad de equilibrio.


Este además genera una contracción de la oferta y una perdida de excedente social.\\


A que lado del mercado afecta más el impuesto depende de las elasticidades de los mercados. Siendo este pagado por tanto consumidores como productores.

\subsubsection{Fijación de precios}
Trae una perdida irrecuperable de eficiencia. Si el precio fijado es menor al precio de equilibrio.

\newpage