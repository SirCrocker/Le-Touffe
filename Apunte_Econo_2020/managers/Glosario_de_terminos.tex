\section{Glosario de términos}

\textit{Glosario de términos del CORE ECON (Puede buscar con ctrl+f o cmd+f)} \href{https://www.core-econ.org/the-economy/book/es/text/50-02-glossary.html#glossary-econom}{\textbf{IR}}

\begin{itemize}
    
    \item \hypertarget{trade-off}{\textbf{Trade Off}}: decisión tomada en una situación conflictiva en la cual se debe perder, reducir cierta cualidad a cambio de otra cualidad. En economía se lo suele traducir como 'intercambio', destacando entonces que se pierde un beneficio y se gana otro. \textit{[Fuente: Wikipedia]}
    
    \item \hypertarget{equilibrio}{\textbf{Equilibrio}}: Resultado autosostenible de un modelo. En este caso, algo de interés no cambia, a menos que se introduzca una fuerza externa que altere la descripción de la situación que proporciona el modelo. \textit{[Fuente: CORE]}
    
    \item \hypertarget{subsistencia}{\textbf{Nivel de subsistencia}}: Nivel de vida (medido en términos del consumo o el ingreso) al que la población no crece ni decrece. \textit{[Fuente: CORE]}
    
    \item \hypertarget{rendimiento-decreciente}{\textbf{Rendimiento decreciente}}: Situación en la cual el uso de una unidad adicional de un insumo de producción resulta en un menor incremento en el producto, respecto al incremento anterior. \textit{También se conoce como: rendimientos marginales decrecientes de la producción.} \fuente{CORE}
    
    \item \hypertarget{utilidad}{\textbf{Utilidad}}: Indicador numérico de valor que uno asigna a un resultado, de modo que se escojan resultados de mayor valor por encima de otros de menor valor cuando ambos sean factibles. \fuente{CORE}
    
    \item \hypertarget{frontera-factible}{\textbf{Frontera factible}}: Curva de puntos que define la máxima cantidad factible de un bien para una cantidad dada de otro. \fuente{CORE}
    
    \item \hypertarget{poder}{\textbf{Poder}}: capacidad de hacer y obtener las cosas que queremos, en contraposición con las intenciones de los demás.
    
    \item \hypertarget{instituciones}{\textbf{Instituciones}}: son reglas escritas y no escritas que rigen qué hacen las personas cuando interactúan en un proyecto común, y la distribución de los productos resultantes de su esfuerzo conjunto.
    
    \item \hypertarget{ex-ante}{\textbf{Ex-ante}}: significa "antes del suceso". Ex-ante se usa más comúnmente en el mundo comercial, donde los resultados de una acción concreta, o una serie de acciones, se prevén con antelación (o eso se pretende).
    
\end{itemize}

\newpage