\section{Ejemplos}

\subsection{Costo de oportunidad}

\fuente{CORE}

Imagine que se les ha pedido a un contador y a un economista que informen sobre el costo de ir a un concierto \textit{A}, en un teatro, con una entrada cuyo costo asciende a 25 dólares. En un parque cercano hay un concierto \textit{B}, que es gratuito, pero que se celebra al mismo tiempo.
\\

\textbf{contador}:
el costo del concierto \textit{A} es el costo entendido como «lo que sale de su bolsillo»: usted ha pagado 25 dólares por una entrada, por lo tanto, el costo es 25 dólares.
\\

\textbf{economista}
¿Pero a qué tiene que renunciar para ir al concierto \textit{A}? Usted ha dado 25 dólares, más el disfrute del concierto gratuito en el parque. Así que el costo del concierto para usted es el costo en términos de lo que sale de su bolsillo más el costo de oportunidad.
Suponga que lo máximo que hubiera estado dispuesto a pagar para asistir al concierto gratuito en el parque (si no fuera gratuito) fueran 15 dólares. Entonces su beneficio, si es que eligiera su siguiente mejor alternativa al concierto \textit{A}, sería de 15 dólares de disfrute en el parque. Este es el costo de oportunidad de ir al concierto \textit{A}.
\\

Así que el costo económico (costo de bolsillo de una acción + costo de oportunidad) total del concierto \textit{A} es 25 dólares + 15 dólares = 40 dólares. Si anticipa que el goce que experimentará por ir al concierto \textit{A} es 50 dólares, dejará pasar el concierto \textit{B} y comprará la entrada para el teatro, porque 50 dólares es más que 40 dólares. Por otro lado, si anticipa que el goce que experimentará en el concierto \textit{A} es 35 dólares, entonces el costo económico de 40 dólares indica que no escogerá ir al teatro. En términos simples: dado que tiene que pagar 25 dólares por la entrada, optará por el concierto \textit{B} y se guardará los 25 dólares para gastarlos en otras cosas y disfrutar así de un beneficio valorado en 15 dólares resultante de ir al concierto gratuito en el parque.

\subsection{Discriminación primer grado}
\label{ejem:disc_1er_grado}
\begin{itemize}
    \item $1^{\text{er}}$ caso: el médico de un pueblo pequeño
    \item El monopolista fija precios diferentes para cada consumidor y para cada consulta comprada por cada uno de ellos
    \item Información: el monopolista (médico) puede identificar a cada consumidor.
    \item Arbitraje o fuga: no es posible
    \item Precios: diferentes para cada consumidor y unidad
\end{itemize}

\subsection{Discriminación de segundo grado}
\label{ejem:disc_2do_grado}
\begin{itemize}
    \item \textit{A} valora en \$3 la primera unidad, en \$2 la segunda, y \$1 la tercera.
    \item \textit{B}valora en \$4 la primera unidad, en \$3 la segunda, y en \$2 la tercera.
    \item Curva de Demanda: $(\$4, 1)$, $(\$3,3)$, $(\$2,5)$, $(\$1,6)$
    \item Mejor precio ($CMg = 0$)=\$2. Se venden 5 unidades a \$10.
    \item Si se tiene una unidad \$3, dos unidades \$4.8, tres unidades a \$6.5\\
    \textit{A} compra dos unidades, \textit{B} compra 3 unidades (Se auto-segmentan). Se venden 5 unidades en \$11.3
\end{itemize}

\newpage