\section{Discriminación de precios y tarificación}
La discriminación de precios es el intento del monopolista de apoderarse de parte (o todo) el excedente del consumidor más la pérdida de eficiencia.


\subsection{Segmentos de mercado y monopolista}

Un monopolio puede verse enfrentado a la demanda de distintos tipos de usuarios/clientes (como estudiantes, adultos mayores, público general, entre otros). Donde cada tipo de usuario puede poseer una curva de demanda distinta.\\

Los tipos de usuarios se denominan \textit{segmentos}.

\subsection{Demanda total}

Es la suma de las demandas de todos los segmentos, pero sumando \underline{cantidades}, no precios. Por ejemplo

\begin{equation}
	\begin{split}
		q_1 &= \text{máx}\{ S_1(P), 0 \}\\
		q_2 &= \text{máx}\{ S_2(P), 0 \}
	\end{split}
	\nonumber
\end{equation}

Donde $S_1(P)$ y $S_2(P)$ pueden ser negativos según $P$. Así sea

\begin{equation}
	\begin{split}
		S_1(P) &= \begin{cases} 
					S_1 < 0 &\quad \text{si } a \leq P \leq b\\ 
					S_1 > 0 &\quad \text{si } b \leq P
					\end{cases}\\
		S_2(P) &= \begin{cases} 
					S_2 < 0 &\quad \text{si } a \leq P \leq c \leq b\\ 
					S_2 > 0 &\quad \text{si } c \leq P
					\end{cases}
	\end{split}
	\nonumber	
\end{equation}

Tenemos que $Q(P) = q_1 + q_2$, con

\[ Q(P) = 
\begin{cases}
	0 &\quad \text{si } P \leq c\\
	S_1(P) &\quad \text{si } c \leq P \leq b\\
	S_2(P) + S_1(P) &\quad \text{si } b \leq P
\end{cases}
\]

\subsection{Problema del monopolista}

El monopolista quiere alcanzar $IMg = CMg$, en algunos casos $CMg$ puede hacerse \cita{infinito} en un punto si ya no se permite seguir vendiendo bienes.\\

El monopolista puede cobrar precios específicos por segmento del mercado, es necesario que no exista fuga de clientes (o arbitraje) para llevar esto a cabo, es decir, que personas de un segmento $i$ puedan hacerse pasar por personas otro segmento $j$ distinto. O, que los bienes puedan ser revendidos, más barato en comparación a los precios de un segmento dado, por alguien de otro segmento.\\

En estos casos la utilidad que recibe el monopolista se debe a la suma de los ingresos de todos los segmentos, y el problema de optimización, haciendo uso del ejemplo anterior se transforma en

\begin{equation}
	\begin{split}
		\text{máx}& \{ I_1(q_1) + I_2(q_2) \}\\
		\text{s.a }& g(q_1,q_2)
	\end{split}
	\nonumber
\end{equation}

El problema se resuelve haciendo uso de multiplicadores de Lagrange, obteniéndose los precios y cantidades óptimas.\\

La utilidad ($\Pi$) puede ser mayor a la que se obtendría en el óptimo solo sumando demanda a un precio para todos, porque al cobrar precios diferentes el monopolio se aprovecha de las disposiciones a pagar del segmento, capturando más consumidores y a mejores precios.\\

Es importante notar que el multiplicador de Lagrange será $\lambda^*$, esto representa la utilidad marginal de poder vender un bien más. Esto se evidencia con que, sea $aq_1 + bq_2 = \alpha$ entonces $IMg_{q_1} = a\lambda^*$ y $IMg_{q_2} = b\lambda^*$.

\subsection{Precios múltiples}

Así como existen segmentos de consumidores con diferentes disposiciones a pagar, también podemos encontrar que existen disposiciones a pagar diferentes por unidades adicionales de un producto.\\

El vendedor monopólico podría diferenciar las \cita{primeras} unidades de las \cita{segundas}, \cita{terceras}, etc. Para ello puede construir curvas de demanda para cantidad a comprar.\\

Estos precios pueden implementarse haciendo descuentos por volumen, siendo una condición que no exista fuga de clientes. 

\subsection{Discriminación de precios}
Se dice que un productor realiza discriminación de precios si 2 unidades del \txtsi{\cita{mismo bien}}{(\ref{misc:mismo_precio})} son vendidas a precios distintos (sea al mismo consumidor o a consumidores distintos).\\

La discriminación de precios permite al monopolista capturar más excedentes de los consumidores, existen 3 tipos o grados de discriminación.

\subsubsection{Discriminación de primer grado}
\txtsi{\textbf{Ejemplo}}{(\ref{ejem:disc_1er_grado})}\\

Si el monopolio pudiera cobrar a cada consumidor su disposición a pagar, entonces capturaría todo el excedente de los consumidores. Esto se conoce como discriminación de primer grado.\\

En la práctica es muy difícil de lograr. Y es un caso socialmente eficiente, ya que la cantidad máxima de personas puede acceder al servicio/bien.

\subsubsection{Discriminación de segundo grado}
\txtsi{\textbf{Ejemplo}}{(\ref{ejem:disc_2do_grado})}\\

Es el caso en que hay información incompleta. El monopolista conoce cuantos \cita{tipos} de consumidores hay, pero no los sabe distinguir \hyperlink{ex-ante}{ex-ante}. Utiliza mecanismos de selección, es decir combinaciones precio-cantidad, precio-calidad de modo que ellos mismos se discriminen.\\


Es conocida también como la discriminación por volumen.

\subsubsection{Discriminación de tercer grado}
Utiliza alguna señal (edad, ocupación, localización) para hacer discriminación, diferenciar entre grupos y cobrar precios distintos (o generar multi-mercados).\\

Es conocida también como discriminación por segmentación de consumidores, y también supone que no hay fuga/arbitraje.\\

\textbf{Ejemplos:} descuentos para estudiantes, descuentos de la 3era edad, precios según localización.


\newpage