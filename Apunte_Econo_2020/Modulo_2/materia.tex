\section{Escasez}

\econ{https://www.core-econ.org/the-economy/book/es/text/03.html}{3}
\newline
\newline
Sea $Y = f(h)$ la función de producción, con $h$ el bien gastado en producir (puede ser horas, número de trabajadores, tierra utilizada, tiempo libre, etc). Se tendrá que:

\begin{itemize}
    \item \textbf{Producto medio}: $P_{Me} = \frac{f(h)}{h}$
    \newline 
    \newline Es la cantidad promedio de $Y$ conseguido con una cantidad $h$.
    \item \textbf{Producto marginal}: $P_{Mg} = \frac{df(h)}{dh}$
    \newline
    \newline Es el incremento en $Y$ que se obtiene por usar (o gastar) una cantidad mayor o menor de $h$. \textit{(Literalmente la derivada)}
    \newline También se puede ver como el retorno obtenido al usar distintas cantidades de $h$. \textit{(Cuanto Y cada h te da, que varía al aumentar o disminuir la cantidad de h)}
\end{itemize}

Donde la función de producción determina el conjunto factible de alternativas a elegir.

\subsection{Costo de oportunidad} 
\hyperlink{ejemplo-Costo de oportunidad}{Ejemplo}

Es el costo de la alternativa a la que renunciamos cuando tomamos una determinada decisión, incluyendo los beneficios que podríamos haber obtenido de haber escogido la opción alternativa. 

\subsection{Preferencias}

Para escoger una opción se depende de los beneficios frente a las otras opciones, que está intrínsecamente ligado a las preferencias de la persona.
\newline

Las preferencias de una persona se modelan a través de la función de \hyperlink{utilidad}{utilidad} $U$, que indica la utilidad percibida al consumir cierta cantidad dada del bien.

\subsubsection{Curvas de indiferencia}
Cuando las opciones indiferentes entre sí se marcan, se da origen a una curva de indiferencia. (En esta curva la utilidad siempre posee el mismo valor). 
\\

\textbf{Propiedades de las curvas de indiferencia:}

\begin{itemize}
    \item Las curvas de indiferencia tienen pendiente negativa que refleja las disyuntivas que implican una cierta renuncia: \textit{si hay dos combinaciones ante las que nos mostramos indiferentes, necesariamente eso implica que la que tenga más de un bien tendrá menos del otro bien.}
    \item Unas curvas de indiferencia más altas se corresponden a niveles de utilidad más altos: \textit{a medida que nos movemos hacia arriba y a la derecha en el diagrama, más lejos del origen, nos movemos a combinaciones que tienen más de ambos bienes.}
    \item Las curvas de indiferencia son suaves por lo general: \textit{cambios pequeños en la cantidad de bienes no causan grandes saltos en la utilidad.}
    \item Las curvas de indiferencia no se cruzan.
    \item A medida que te mueves hacia la derecha a lo largo de una curva de indiferencia, esta se vuelve más plana.
\end{itemize}

El último punto hace referencia a la \textit{utilidad decreciente de los bienes}, que establece que mientras más de un bien se tiene (o consume), menor es la utilidad adicional que se le otorga.

\subsection{Tasa marginal de sustitución (TMS)}
Es la cantidad de bien del eje vertical que se está dispuesto a dar por 'un' bien del eje horizontal, su valor viene dado por la tangente en un punto de la curva de indiferencia.
%Es cuanto se está dispuesto a dar por cada bien del eje vertical por algo del eje horizontal, 

En el caso de que la función de producción venga dada por $y = f(x)$ con $U = U(x, y)$ \[TMS = \abs{\frac{\partial U}{\partial x} / \frac{\partial U}{\partial y}} = -\frac{\partial U}{\partial x} / \frac{\partial U}{\partial y}\]

En palabras la fórmula anterior se puede escribir como
\[\text{TMS} = \abs{\frac{\text{utilidad marginal de \textit{x}}}{\text{utilidad marginal de \textit{y}}}}\]

\subsection{Optimización de la decisión}
Siempre se busca maximizar la utilidad sujeta al conjunto factible, este óptimo se encuentra en la \hyperlink{frontera-factible}{frontera del conjunto factible} y/o en la curva de indiferencia de mayor utilidad ('arriba a la derecha').

En el óptimo se cumple que la curva de indiferencia es tangente a la frontera factible y \[TMS = -P_{Mg}\] (Esto es equivalente a que las tangentes de la frontera factible y el producto marginal sean iguales)

\subsubsection{Tasa marginal de transformación (TMT)}
Cantidad de algún bien que debe sacrificarse para adquirir una unidad adicional de otro bien. En cualquier punto, es la pendiente de la frontera factible. Es equivalente a $-P_{Mg} \implies TMT = TMS$ también es válido para obtener óptimos.

\subsection{Efecto neto}
Es la suma del efecto ingreso y el efecto neto.

\subsubsection{Efecto ingreso}
Son los cambios en el consumo debido a una expansión en el conjunto factible, pero manteniendo el mismo trade-off entre cantidad consumida y tiempo libre.

Se deriva de la perdida (o ganancia) del poder adquisitivo, es ajeno a la variación relativa de los precios.
\\

\textit{Se desplaza la frontera factible verticalmente}, puede causar que la cantidad de bien dispuesto a invertir disminuya.\\


\subsubsection{Efecto sustitución}
Es la variación del consumo de bien cuyo precio varía. Se deriva exclusivamente de la variación de los precios, y es por lo tanto, ajeno al cambio de poder adquisitivo.
\\

\textit{Se da únicamente por cambios en el precio o el costo de oportunidad, dado el nuevo nivel de utilidad}, puede causar que la cantidad de bien dispuesto a invertir aumente.

\subsubsection{Como calcular los efectos}
\href{https://www.core-econ.org/the-economy/book/es/text/leibniz-03-07-01.html}{Según CORE.\\}


En un problema se obtienen valores de \textit{x} e \textit{y} en el óptimo, sean estos $x_0$ e $y_0$.

Cuando hay un cambio en el problema/situación planteada para calcular el efecto ingreso hay que, obtener la nueva utilidad óptima y los valores de las variables correspondientes ($x_1$ e $y_1$), e intentar conseguir un ingreso que causaría que la utilidad óptima fuera igual a la del problema modificado ($x_2$ e $y_2$). Luego la diferencia de ''x'' ($x_2 - x_0$) es el efecto ingreso. Si es que la diferencia no es igual a la diferencia total ($x_2 - x_0 \neq x_1 - x_0$), entonces lo faltante es el efecto sustitución. 

\subsection{Problemas de optimización}
En estos problemas se intenta optimizar una función objetivo, para encontrar su mínimo o máximo. Esta está sujeta a restricciones, que ponen límites a las variables presentes en el problema. 
\\

Un ejemplo es maximizar la función utilidad $U(x_1,x_2)$ donde la restricción es que $g(x_1,x_2) \leq w$, con $w$ un valor. Aquí las variables son $x_1$ y $x_2$.
\newpage